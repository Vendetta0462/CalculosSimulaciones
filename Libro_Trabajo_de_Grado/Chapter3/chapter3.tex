% Marco teórico - Microfísica
%\chapter{Microfísica: Ecuaciones de Estado}
%\thispagestyle{fancy}

% Introducción similar al capítulo sobre macrofísica. Aquí queremos mostrar brevemente la motivación de el estudio de ecuaciones de estado para entender la física de la materia en entornos tan densos y de tan alta energía.

%\section{Ecuaciones de Estado y Ejemplos}

%Definir las ecuaciones de estado (en este caso barótropas) y mostrar brevemente modelos y gráficas de modelos más simples: gas degenerado de neutrones, protones y electrones libres

%\section{Teoría Relativista de Campo Medio}

% Introducir la RMFT y mostrar las ventajas de una teoría relativista efectiva de campos mesónicos que se toman como uniformes mediante su valor medio en el estado base, frente a otros métodos como QCD y los Scrhodinger-based. Hablar sobre las simetrías y sus cantidades conservadas para explicar como se obtiene la expresión para la densidad de energía y de presión en una teoría de este estilo, construida sobre un lagrangeano.

%\section{Modelo del Estudio}

% Introducir el lagrangiano para un modelo que contenga neutrones, protones, electrones (campos esponoriales); un mesón escalar neutro sigma con autointeracciones de hasta 4to orden acoplado a la densidad escalar de nucleones, un mesón vectorial neutro omega acoplado a la corriente vectorial de nucleones y un mesón vectorial isovectorial neutro rho acoplado a la corriente de isospín de nucleones. Luego hallar las ecuaciones de movimiento para los campos, para finalmente llegar a la expresión de densidad de energía y presión. Hablar sobre los parámetros libres de este modelo.

%\subsection{Materia en Saturación Nuclear}

% Definir, explicar (importancia) y mostrar las mediciones de densidad de saturación, energía de enlace por nucleón, modulo de compresión, coeficiente de energía de simetría y pendiente del coeficiente de energía de simetría. Tomar las mismas mediciones de la propuesta.

\thispagestyle{fancy}
\chapter{Microfísica: Ecuaciones de Estado}
\label{chap:microfisica}

La descripción microscópica de la materia en entornos extremos de densidad es un problema de gran complejidad en la física moderna. En estas condiciones, con densidades que exceden significativamente la densidad nuclear de saturación, la materia exhibe comportamientos que requieren marcos teóricos que incorporen efectos relativistas y de muchos cuerpos, asegurando causalidad en el fluido y modelando las interacciones nucleares relevantes. Los modelos que describen esta materia establecen la conexión directa entre la física microscópica de las interacciones nucleares y las propiedades macroscópicas observables de las estrellas de neutrones \cite{oppenheimerMassiveNeutronCores1939}.

Diferentes teorías han sido propuestas para describir la materia en este régimen, desde enfoques no-relativistas basados en potenciales nucleares ajustados a datos experimentales \cite{myersNuclearPropertiesAccording1996, sakuragiSaturationNuclearMatter2016}, hasta teorías fundamentales como la cromodinámica cuántica (QCD) y sus extensiones efectivas \cite{drischlerGroundingNuclearPhysics2021}. Sin embargo, para obtener una extrapolación adecuada a densidades extremas, es necesario contar con un marco teórico que respete la causalidad y cuya complejidad computacional sea manejable. Es así como la teoría relativista de campo medio surge como una herramienta particularmente adecuada para abordar este régimen, brindando un tratamiento consistente que respeta la causalidad mientras incorpora las interacciones nucleares fuertes \cite{glendenningCompactStarsNuclear2000}. Este formalismo permite extrapolar desde las propiedades conocidas de materia nuclear simétrica hacia las condiciones asimétricas y de alta densidad relevantes para objetos compactos
\clearpage

\section{Ecuaciones de Estado y Ejemplos}

Una ecuación de estado define la relación termodinámica entre las variables que caracterizan el estado de equilibrio de un sistema físico. Para materia estelar a temperatura cero, consideramos ecuaciones de estado barotrópicas que relacionan la presión $P$ con la densidad de energía $\rho$. Esta relación contiene toda la información termodinámica necesaria para determinar la estructura de equilibrio hidrostático de estrellas de neutrones a través de las ecuaciones de Tolman-Oppenheimer-Volkoff (\sistemaTOV). La aproximación barotrópica es válida cuando los tiempos y escalas característicos de los procesos térmicos son despreciables comparados con las escalas hidrodinámicas y gravitacionales que determinan la estructura estelar. Se desprecia la temperatura debido a que la energía térmica y sus efectos son varios órdenes de magnitud inferiores a las energías internas de la materia en estrellas de neutrones \cite{shapiroBlackHolesWhite2008}.

\subsection{Ecuación Politrópica}

La ecuación de estado politrópica es uno de los modelos más sencillos para describir materia estelar, estableciendo una relación de ley de potencias entre la presión $P$ y la densidad de masa $\rho_m$:

\begin{equation}
	P = K (\rho_m)^{\gamma},
	\label{eq:eos_politropica}
\end{equation}

donde $K$ es una constante politrópica y $\gamma$ es el índice adiabático. Esta forma funcional, aunque fenomenológica, captura comportamientos asintóticos importantes de sistemas físicos más complejos y brinda soluciones analíticas o semi-analíticas para las ecuaciones de estructura estelar. El índice politrópico $n = 1/(\gamma - 1)$ determina las características de compresibilidad del material: valores bajos de $n$ corresponden a materia incompresible, mientras que valores altos describen sistemas altamente compresibles. Para materia ultra-relativista, el índice adiabátoco es $\gamma = 4/3$ ($n = 3$), mientras que para materia no-relativista degenerada, es $\gamma = 5/3$ ($n = 3/2$) \cite{chandrasekharIntroductionStudyStellar1970}. Este modelo de ecuación de estado es ampliamente utilizada en su versión politrópica a trozos (Piecewise Polytropic) para aproximar ecuaciones de estado más complejas mediante segmentos con diferentes índices politrópicos, lo que facilita su implementación en simulaciones numéricas \cite{becerraRealisticAnisotropicNeutron2024, raaijmakersConstraintsDenseMatter2021, chatziioannouNeutronStarsDense2024, choudhuryNICERViewNearest2024b}. En particular, es empleada para realizar estimaciones sobre el impacto de las mediciones astrofísicas de estrellas de neutrones en la ecuación de estado de la materia nuclear densa \cite{raaijmakersConstraintsDenseMatter2021}.

\subsection{Gas Ideal Degenerado}
\label{sec:gasnpe}

Para densidades suficientemente bajas tal que se pueda despreciar la interacción nuclear, un modelo más realista consiste en gases degenerados de fermiones, el modelo físico más simple frecuentemente empleado como referencia. Para materia nuclear compuesta por neutrones, protones y electrones, la ecuación de estado completa debe incluir las contribuciones de todas las especies presentes \cite{shapiroBlackHolesWhite2008}:

\begin{align}
	\rho &= \rho_n + \rho_p + \rho_e, \label{eq:densidad_total} \\
	P &= P_n + P_p + P_e, \label{eq:presion_total}
\end{align}

donde cada componente fermiónica contribuye según:

\begin{align}
	\rho_i &= \frac{g_i}{8\pi^2} \int_0^{p_{Fi}} p^2\sqrt{p^2 + m_i^2} \, dp, \label{eq:densidad_fermi_general} \\
	P_i &= \frac{g_i}{24\pi^2} \int_0^{p_{Fi}} \frac{p^4}{\sqrt{p^2 + m_i^2}} \, dp, \label{eq:presion_fermi_general}
\end{align}

con masas $m_n = 939.6$ MeV, $m_p = 938.3$ MeV, $m_e = 0.511$ MeV para los neutrones, protones y electrones respectivamente, y degeneraciones estadísticas (de espín) $g_i = 2$ para todas las especies. Los momentos de Fermi $p_{Fi}$ están determinados por las densidades de número mediante $n_i = \frac{g_i p_{Fi}^3}{6\pi^2}$. Para resolver el sistema, se imponen restricciones adicionales sobre la composición de la materia garantizando el equilibrio termodinámico: la neutralidad de carga eléctrica 

\begin{equation}
	n_p = n_e,
	\label{eq:neutralidad_carga}
\end{equation}

y el equilibrio beta débil 

\begin{equation}
	n \rightleftharpoons p + e^- + \bar{\nu}_e \implies \mu_n = \mu_p + \mu_e,
	\label{eq:equilibrio_beta}
\end{equation}

entre los potenciales químicos $\mu_i$, suponiendo que los neutrinos escapan del sistema sin alterar su energía. Estas restricciones permiten expresar todas las densidades en función de un parámetro libre como el momento de Fermi del electrón $p_{Fe}$, lo que reduce el sistema a una parametrización unidimensional.

La figura \ref{fig:eosnpe} muestra la ecuación de estado adimensionalizada para un gas ideal de neutrones, protones y electrones libres, obtenida tras resolver el sistema de integrales (\ref{eq:densidad_fermi_general}) y (\ref{eq:presion_fermi_general}) e interpolar, verificando que es causal ($c_s^2 < 1$). Adicionalmente, la figura \ref{fig:gas_npe_fraccion} muestra la fracción de neutrones respecto al número total de nucleones (bariones) $n_n/n_B = n_n/(n_n+n_p)$ como función de la densidad bariónica $n_B$. Se muestra la fracción de neutrones en el rango de densidades de 7.0$\times 10^{12}$ g/cm$^3$ a 1.6$\times 10^{16}$ g/cm$^3$. La fracción de neutrones en este rango de densidades inicia muy cercana al 100\%, disminuyendo lentamente a medida que la densidad aumenta, alcanzando aproximadamente un 94.5\% al llegar a $10^{16}$ g/cm$^3$, consistente con lo esperado para materia nuclear en equilibrio beta \cite{shapiroBlackHolesWhite2008}.

\begin{figure}[h]
	\centering
	\begin{subfigure}{0.42\linewidth}
		\centering
		\includegraphics[width=\linewidth]{Figuras/gas_npe}
		\caption{}
		\label{fig:eosnpe}
	\end{subfigure}
	\hfill
	\begin{subfigure}{0.57\linewidth}
		\centering
		\includegraphics[width=\linewidth]{Figuras/gas_npe_fraccion}
		\caption{}
		\label{fig:gas_npe_fraccion}
	\end{subfigure}
	\caption[Propiedades del gas ideal degenerado]{Propiedades del gas ideal degenerado de neutrones, protones y electrones. (a) Ecuación de estado normalizada. La ecuación de estado es causal ($c_s^2 = dP/d\rho < 1$). $\rho_0$ es el parámetro de adimensionalización y corresponde a la densidad de energía a $\rho_m=10^{18}$ g/cm$^3$. (b) Fracción de neutrones respecto al número de nucleones $n_n/n_B$ como función de la densidad bariónica $n_B$.}
	\label{fig:gas_npe_propiedades}
\end{figure}

Luego de obtener la ecuación de estado e integrando las ecuaciones TOV (\sistemaTOV) obtenemos las masas y radios de estrellas construidas con este material, como se muestra en la figura \ref{fig:mrnpe}. Este modelo sencillo predice una masa máxima de apenas 0.7$\, \masasol$, muy inferior a las masas que se han observado de estrellas de neutrones (listadas en la sección \ref{sec:obsNS}), motivando la búsqueda de modelos más realistas que incluyan las interacciones nucleares para lograr replicar las observaciones.

\begin{figure}[h]
	\centering
	\includegraphics[width=0.9\linewidth]{Figuras/gas_npe_MR}
	\caption[Relaciones masa-radio de gas ideal degenerado]{Relaciones de masa - radio (izquierda) y masa - densidad central de masa (derecha) de estrellas de neutrones constituidas por un gas ideal de neutrones, protones y electrones libres.}
	\label{fig:mrnpe}
\end{figure}

\section{Teoría Relativista de Campo Medio}
\label{sec:rmft}

La teoría relativista de campo medio, formulada por Walecka en 1974 \cite{waleckaTheoryHighlyCondensed1974}, es un marco teórico que describe las interacciones nucleares mediante el intercambio de mesones efectivos, extendiendo el modelo de gas degenerado (discutido en \ref{sec:gasnpe}) e incorporando naturalmente los efectos relativistas cuando los nucleones alcanzan velocidades comparables a la velocidad de la luz en condiciones de alta densidad. Esta aproximación ofrece ventajas significativas respecto a enfoques no-relativistas basados en potenciales fenomenológicos, ya que es una teoría covariante y reproduce simultáneamente las propiedades de saturación nuclear, el comportamiento asintótico a alta densidad, y la consistencia causal relativista. Adicionalmente, al ser una teoría efectiva, requiere de un menor esfuerzo computacional respecto a teorías más fundamentales como QCD.

El formalismo se construye a partir de un lagrangiano que describe nucleones interactuando a través de campos mesónicos: los campos fermiónicos representan los grados de libertad nucleónicos y leptónicos, mientras que los campos bosónicos representan mesones que median las interacciones fuertes. La aproximación de campo medio consiste en reemplazar los operadores de campo mesónicos por sus valores esperados en el estado base degenerado:

\begin{equation}
	\langle \phi_i(x^\mu) \rangle = \phi_i^0 \equiv \text{constante},
	\label{eq:campo_medio}
\end{equation}

donde $\phi_i$ denota los diferentes campos mesónicos del modelo. Esta aproximación es válida cuando las fluctuaciones cuánticas son pequeñas comparadas con los valores esperados de los campos, condición que se satisface para materia nuclear densa cuanto mayor es la densidad del sistema \cite{waleckaRelativisticNuclearManyBody1986}.

\subsection{Simetrías y Conservaciones}

El formalismo de la teoría relativista de campo medio se construye sobre la teoría cuántica de campos. La densidad lagrangiana $\mathcal{L}(\psi, \partial_\mu \psi, \phi_a, \partial_\mu \phi_a)$ describe las interacciones entre campos fermiónicos $\psi$ y campos bosónicos $\phi_a$ junto con sus términos libres. Esta densidad lagrangiana debe satisfacer los requerimientos de localidad, covariancia de Lorentz, y las simetrías internas relevantes para las interacciones nucleares fuertes \cite{glendenningCompactStarsNuclear2000}. La acción del sistema se define como la integral de la densidad lagrangiana sobre el volumen espaciotemporal:

\begin{equation}
	S[\psi, \phi_a] = \int d^4x \, \mathcal{L}(\psi, \partial_\mu \psi, \phi_a, \partial_\mu \phi_a),
	\label{eq:accion}
\end{equation}

donde $d^4x$ es el elemento de volumen en coordenadas de Minkowski. El principio de acción estacionaria establece que las configuraciones físicas de los campos corresponden a los extremos de este funcional, lo que conduce a las ecuaciones de Euler-Lagrange para todos los campos:

\begin{equation}
	\frac{\partial \mathcal{L}}{\partial \varphi_b} - \partial_\mu \left( \frac{\partial \mathcal{L}}{\partial (\partial_\mu \varphi_b)} \right) = 0,
	\label{eq:euler_lagrange}
\end{equation}

donde $\varphi_b$ representa todos los campos presentes. La teoría presenta simetrías externas e internas que determinan sus leyes de conservación y sus consecuencias físicas. Las simetrías externas son las transformaciones del grupo de Poincaré: traslaciones espaciotemporales $x^\mu \mapsto x'^\mu = x^\mu + a^\mu$ y boosts de Lorentz $x^\mu \mapsto x'^\mu = \Lambda^\mu{}_\nu x^\nu$. Las simetrías internas relevantes incluyen la simetría de gauge global $U(1)$ $\psi \mapsto e^{-i\lambda}\psi$ asociada con la conservación del número bariónico, y la simetría de isospín $SU(2)$ $\psi \mapsto e^{-i\boldsymbol{\tau}\cdot\boldsymbol{\lambda}}\psi$ asociada con la relación entre neutrones y protones, en la aproximación de iguales masas.

El teorema de Noether establece una correspondencia entre simetrías continuas del lagrangiano y cantidades conservadas. Para cada simetría continua existe una corriente conservada correspondiente que satisface una ecuación de continuidad. El tensor de energía-momento, asociado a la simetría externa, se define como:

\begin{equation}
	T^{\mu\nu} = \frac{\partial \mathcal{L}}{\partial (\partial_\mu \varphi_j)} \partial^\nu \varphi_j - \eta^{\mu\nu} \mathcal{L} \implies \partial_\mu T^{\mu\nu} = 0,
	\label{eq:tensor_energia_momento}
\end{equation}

garantizando que la energía total $\int d^3x T^{00}$ del sistema se conserve en el tiempo en ausencia de fuerzas externas (flujos de momento en la frontera).

La corriente bariónica, asociada a la simetría interna $U(1)$, se define como:

\begin{equation}
	J_B^\mu = \sum_N \bar{\psi}_N \gamma^\mu \psi_N \implies \partial_\mu J_B^\mu = 0,
	\label{eq:corriente_barionica}
\end{equation}

donde $\gamma^\mu$ son las matrices de Dirac y la suma se extiende sobre todas las especies de bariones presentes. Esto implica que el número bariónico $\int d^3x \, J_B^0$ integrado sobre el volumen total del sistema permanece constante en el tiempo, reflejando el hecho experimental de que los bariones no se crean ni se destruyen en interacciones fuertes como se ha evidenciado por ALICE en el LHC \cite{acharyaGlobalBaryonNumber2020} y se ha probado desde su postulación en 1940 \cite{gurrExperimentalTestBaryon1967}.

La corriente de isospín, asociada a la simetría interna $SU(2)$, se define como:

\begin{equation}
	\boldsymbol{J}_I^{\mu} = \sum_N \bar{\psi}_N \gamma^\mu \frac{\boldsymbol{\tau}}{2} \psi_N \implies \partial_\mu J_I^{\mu a} = 0,
	\label{eq:corriente_isospin}
\end{equation}

donde $\boldsymbol{\tau} = \tau^a$ ($a = 1, 2, 3$) son las matrices de Pauli en el espacio de isospín. La simetría de isospín ha sido ampliamente utilizada para describir las interacciones nucleares. En la realidad física esta simetría está ligeramente rota por las diferencias de masa entre neutrones y protones, y por las interacciones electromagnéticas. Sin embargo, dado que estas violaciones son pequeñas en la escala de interacciones fuertes, pueden tratarse como pequeñas correcciones e ignorarse en una primera aproximación \cite{baczykIsospinsymmetryBreakingMasses2018}. La conservación aproximada de isospín justifica el tratamiento unificado de neutrones y protones en modelos de materia nuclear.

% Estas leyes de conservación derivadas del teorema de Noether añaden restricciones importantes sobre la dinámica del sistema y establecen conexiones directas entre las simetrías fundamentales de la teoría y las cantidades físicamente observables.

%\vspace{-15pt}

\section{Modelo del Estudio}
\label{sec:modelo}

El modelo empleado en este trabajo incorpora nucleones (protones y neutrones) interactuando mediante campos mesónicos, y electrones libres. El lagrangiano total incluye términos para nucleones acoplados a los mesones, términos libres para electrones, un mesón escalar neutro $\sigma$ con autointeracciones no-lineales hasta cuarto orden, un mesón vectorial neutro $\omega^\mu$, y un mesón vectorial isovectorial $\boldsymbol{\rho^\mu}$ \cite{glendenningCompactStarsNuclear2000}:

\begin{equation}
	\begin{aligned}
		\mathcal{L} = &\bar{\psi} \left[ \gamma^\mu \left( i\partial_\mu - g_{\omega} \omega_\mu - \half g_{\rho} \boldsymbol{\tau} \cdot \rhomeson_\mu \right) - (m - g_{\sigma} \sigma) \right] \psi  \\
		&+ \frac{1}{2} \left( \partial_\mu \sigma \partial^\mu \sigma - m_\sigma^2 \sigma^2 \right) - \frac{1}{3}bm (g_\sigma\sigma)^3 - \frac{1}{4}c(g_\sigma\sigma)^4  \\
		&- \frac{1}{4} \omega_{\mu\nu} \omega^{\mu\nu} + \frac{1}{2} m_\omega^2 \omega_\mu \omega^\mu  \\
		&- \frac{1}{4} \rhomeson_{\mu\nu} \cdot \rhomeson^{\mu\nu} + \frac{1}{2} m_\rho^2 \rhomeson_\mu \cdot \rhomeson^\mu \\
		&- g_\rho \rhomeson_\mu \cdot [\rhomeson_\nu \cross \rhomeson^{\nu\mu} + 2g_\rho(\rhomeson^\mu \cross \rhomeson^\nu) \cross \rhomeson_\nu] \\
		&+ \bar{\psi}_e \left( i\gamma^\mu \partial_\mu - m_e \right) \psi_e,
		\label{eq:lagrangiano}
	\end{aligned}
\end{equation}

donde $\psi$ es una representación conveniente de los campos nucleónicos como un espinor de ocho componentes, $\psi_e$ es el campo del electrón, $\omega_{\mu\nu} = \partial_\mu \omega_\nu - \partial_\nu \omega_\mu$ y $\rhomeson_{\mu\nu} = \partial_\mu \rhomeson_\nu - \partial_\nu \rhomeson_\mu$ son los tensores antisimétricos de campo asociados a los mesones $\omega$ y $\rhomeson$ respectivamente, $m \approx 938.92 \text{MeV}$ es la masa de un nucleón considerada igual para protones y neutrones, y $m_i$ es la masa de la especie $i$. Los parámetros adimensionales de acoplamiento $g_{\sigma}$, $g_{\omega}$, $g_{\rho}$ cuantifican la intensidad de las interacciones nucleón-mesón, y los parámetros adimensionales $b$, $c$ regulan la autointeracción del mesón sigma. En la construcción de este lagrangiano se acopla el campo escalar con la densidad escalar, el campo vectorial con la corriente bariónica, y el campo isovectorial con la 3-corriente de isospín. Esta última corriente contiene no solo una contribución por los nucleones, sino también una contribución por la corriente propia del campo $\rhomeson$ y otra por la interacción del mesón con su propia 3-corriente.


Las ecuaciones de movimiento para los campos se obtienen mediante las ecuaciones de Euler-Lagrange (\ref{eq:euler_lagrange}). Para los campos mesónicos, se satisfacen:

\begin{gather}
	\left(\square + m_\sigma^2\right) \sigma = g_\sigma\left[\bar{\psi} \psi - bm(g_\sigma\sigma)^2 - c(g_\sigma\sigma)^3\right], \label{eq:eom_sigma_full} \\
	(\square + m_\omega^2) \omega^\mu - \partial^\mu \partial_\nu \omega^\nu = g_{\omega} \bar{\psi} \gamma^\mu \psi, \label{eq:eom_omega_full} \\
	\begin{aligned}
		(\square + m_\rho^2) \rhomeson^\mu - \partial^\mu \partial_\nu \rhomeson^\nu = \half g_{\rho} \bar{\psi} \gamma^\mu \boldsymbol{\tau} \psi - 2g_\rho&[\rhomeson_\nu \cross \rhomeson^{\nu\mu} + \partial_\nu(\rhomeson^\mu \cross \rhomeson^\nu) \\ 
		&+ 4g_\rho[(\rhomeson^\mu \cdot \rhomeson_\nu)\rhomeson^\nu - (\rhomeson_\nu \cdot \rhomeson^\nu)\rhomeson^\mu]]. \label{eq:eom_rho_full}
	\end{aligned}
\end{gather}

Para los campos fermiónicos se obtienen las ecuaciones de Dirac, acoplada a los campos mesónicos para los nucleones y libre para los electrones:

\begin{gather*}
	\left[ \gamma^\mu \left( i\partial_\mu - g_{\omega} \omega_\mu - \half g_{\rho} \boldsymbol{\tau} \cdot \rhomeson_\mu \right) - (m - g_{\sigma} \sigma) \right] \psi = 0, \\
	(i\gamma^\mu \partial_\mu - m_e) \psi_e = 0.
\end{gather*}

Hasta el momento tenemos un conjunto de ecuaciones de movimiento diferenciales, no lineales y acopladas que describen la dinámica completa del sistema. Aplicamos la aproximación de campo medio descrita en la sección \ref{sec:rmft}, considerando que tenemos materia estática y uniforme en su estado base, de modo que los valores esperados de los campos mesónicos son constantes en el espacio y el tiempo. En este caso, las derivadas espaciales y temporales de los campos mesónicos se anulan, simplificando las ecuaciones de movimiento a un sistema algebraico acoplado. Mantendremos las etiquetas para los campos, recordado que ahora son el valor esperado en el estado base. Para los mesones (\ref{eq:eom_sigma_full} - \ref{eq:eom_rho_full}), sustituyendo consistentemente las fuentes de corriente por su valor esperado, se obtiene:
% \begin{equation*}
% 	\begin{aligned}
% 		m_\sigma^2 \sigma &= g_\sigma \left[ \langle \bar{\psi} \psi \rangle - bm(g_\sigma\sigma)^2 - c(g_\sigma\sigma)^3\right], \\
% 		m_\omega^2 \omega^\mu &= g_{\omega} \langle \bar{\psi} \gamma^\mu \psi \rangle, \\
% 		m_\rho^2 \rhomeson^\mu &= \half g_{\rho} \langle \bar{\psi} \gamma^\mu \boldsymbol{\tau} \psi \rangle - 8g_\rho^2( (\rhomeson^\mu \cdot \rhomeson_\nu)\rhomeson^\nu - (\rhomeson_\nu \cdot \rhomeson^\nu)\rhomeson^\mu).
% 	\end{aligned}
% \end{equation*}

\begin{equation}
	\begin{aligned}
		m_\sigma^2 \sigma &= g_\sigma \left[\langle \bar{\psi}_p \psi_p \rangle + \langle \bar{\psi}_n \psi_n \rangle - bm(g_\sigma\sigma)^2 - c(g_\sigma\sigma)^3\right], \\
		m_\omega^2 \omega^\mu &= g_\omega \left[\langle \bar{\psi}_p \gamma^\mu \psi_p \rangle + \langle \bar{\psi}_n \gamma^\mu \psi_n \rangle \right], \\
		m_\rho^2 \rho_3^\mu &= g_\rho \left[\half \langle \bar{\psi}_p \gamma^\mu \psi_p \rangle - \half \langle \bar{\psi}_n \gamma^\mu \psi_n \rangle \right],
		\label{eq:eom_meson_mf}
	\end{aligned}
\end{equation}

donde $\psi_i$ representa el campo fermiónico para la especie $i = \{p, n\}$, y las primeras dos componentes del mesón $\rhomeson^\mu$ son escritas en términos de los operadores de creación y aniquilación para mesones rho cargados $\rho_\pm^\mu = \frac{1}{\sqrt{2}}(\rho_1^\mu\pm i\rho_2^\mu)$, luego su valor esperado se anula en el estado base del sistema \cite{glendenningCompactStarsNuclear2000}, anulando el término fuente de la corriente propia del campo.

Para los campos fermiónicos se tienen ecuaciones sin dependencia de las coordenadas de espacio-tiempo, siendo estados propios de momento:

\begin{equation}
	\begin{gathered}
		\left[ \gamma^\mu \left( p_\mu - g_{\omega} \omega_\mu - g_{\rho} I_3 \rho_{3\mu} \right) - (m - g_{\sigma} \sigma) \right] \psi(p^\nu) = 0, \\
		(\gamma^\mu p_\mu - m_e) \psi_e(p^\nu) = 0,
		\label{eq:eom_fermion_mf}
	\end{gathered}
\end{equation}

donde $I_3 = \{+\half \; \text{para protones}, -\half \; \text{para neutrones}\}$ es el isospín de la partícula. En analogía con la ecuación de Dirac libre, definimos las cantidades de 4-momento y masa efectivas para nucleones:
%\vspace{-8pt}
\begin{equation}
	\begin{gathered}
		P^\mu = p^\mu - g_{\omega} \omega^\mu - g_{\rho} I_3 \rho_{3}^\mu,\\
		m^* = m - g_\sigma \sigma,
	\end{gathered}	
\end{equation}

obteniendo entonces la relación de dispersión relativista:

\begin{equation}
	\left(P_\mu P^\mu - m^{*2}\right)\psi(P^\nu) = 0,
\end{equation}

luego los valores propios de energía para los nucleones son:

\begin{equation}
	\epsilon(\Vec{p})_{I_3} = \sqrt{(\Vec{p}-g_\omega \Vec{\omega}-g_\rho I_3\Vec{\rho}_{3})^2+(m-g_\sigma\sigma)^2} + g_\omega\omega_0 + g_\rho I_3\rho_{30}.
	\label{eq:particleenergy} 
\end{equation}

% Aunque todas las integrales anteriores son analíticas, la ecuación de movimiento para el campo escalar (\ref{eq:eom_meson_mf}) es una ecuación no lineal en $\sigma$ que debe resolverse autoconsistentemente con métodos numéricos. Además, resolviendo el sistema con $g_\sigma \sigma$, $g_\omega \omega_0$ y $g_\rho \rho_{30}$ como variables, el modelo tiene como parámetros libres unicamente los cocientes $g_\sigma/m_\sigma$, $g_\omega/m_\omega$ y $g_\rho/m_\rho$, junto con los parámetros de autointeracción escalar.

Ahora, los valores esperados de los operadores en (\ref{eq:eom_meson_mf}) se calculan con el método descrito en el Apéndice \ref{apendice:valores_esperados}. Luego, las ecuaciones de movimiento quedan de la siguiente forma:

\begin{equation}
	\begin{aligned}
		m_\sigma^2 \sigma &= g_\sigma \sum_N \frac{1}{\pi^2}\int_0^{p_{FN}} \frac{p^2 (m-g_\sigma\sigma) dp}{\sqrt{p^2 + (m-g_\sigma\sigma)^{2}}} - g_\sigma bm(g_\sigma\sigma)^2 - g_\sigma c(g_\sigma\sigma)^3, \\
		m_\omega^2 \omega_0 &= \frac{1}{3\pi^2} g_\omega(p_{Fp}^3 + p_{Fn}^3), \\
		m_\rho^2 \rho_{30} &= \frac{1}{6\pi^2} g_\rho(p_{Fp}^3 - p_{Fn}^3),
	\end{aligned}
	\label{eq:eom_meson_mf_sustituidas}
\end{equation}

donde $N$ indica la suma sobre protones y neutrones. Observamos que las componentes espaciales de los campos vectoriales se anulan debido a la isotropía del sistema en su estado base. 

Para hallar la ecuación de estado, calculamos el tensor de energía-momento canónico (\ref{eq:tensor_energia_momento}) en el marco de referencia del fluido en reposo e isotrópico, utilizando el lagrangiano (\ref{eq:lagrangiano}). Luego, hallamos el valor esperado de sus componentes empleando el mismo método del Apéndice \ref{apendice:valores_esperados} y las identificamos con las cantidades termodinámicas de densidad de energía y presión para un fluido perfecto (\ref{eq:tensor_fluido_perfecto}). Las expresiones para densidad de energía y presión son:

\begin{gather}
	\begin{aligned}
		\rho = \langle T^{00} \rangle &= \frac{1}{2}m_\sigma^2\sigma^2 + \frac{1}{3}bm(g_\sigma\sigma)^3 + \frac{1}{4}c(g_\sigma\sigma)^4 + \frac{1}{2}m_\omega^2\omega_0^2 + \frac{1}{2}m_\rho^2\rho_{30}^2 \\
		&+ \sum_N \frac{1}{\pi^2}\int_0^{p_{FN}} p^2 dp \sqrt{p^2 + (m-g_\sigma\sigma)^2} + \frac{1}{\pi^2}\int_0^{p_{Fe}} p^2 dp \sqrt{p^2 + m_e^2},
		\label{eq:densidad_energia}
	\end{aligned} \\
	\begin{aligned}
		P = \frac{1}{3}\sum_i \langle T^{ii} \rangle &= - \frac{1}{2}m_\sigma^2\sigma^2 - \frac{1}{3}bm(g_\sigma\sigma)^3 - \frac{1}{4}c(g_\sigma\sigma)^4 + \frac{1}{2}m_\omega^2\omega_0^2 + \frac{1}{2}m_\rho^2\rho_{30}^2 \\
		&+ \sum_N \frac{1}{3\pi^2}\int_0^{p_{FN}} \frac{p^4 dp}{\sqrt{p^2 + (m-g_\sigma\sigma)^2}} + \frac{1}{3\pi^2}\int_0^{p_{Fe}} \frac{p^4 dp}{\sqrt{p^2 + m_e^2}}.
		\label{eq:presion}
	\end{aligned}
\end{gather}

Ahora, con el fin de resolver numéricamente el sistema, nos interesa reescribir las ecuaciones necesarias en términos de variables adimensionales. Definimos entonces las siguientes cantidades:

\begin{equation}
	\begin{aligned}
		&x = \frac{p}{m}, & &\xsigma = 1 - \frac{g_\sigma \sigma}{m} = \frac{m^*}{m}, \\
		&A_i = \left(\frac{g_i}{m_i}\right)^2 m^2, & &\tilde{n} = \frac{n_B}{m^3} = \frac{1}{3\pi^2}(x_{Fp}^3+x_{Fn}^3), \\
		& \tilde{\rho} = \frac{2\rho}{m^4}, & & \tilde{P} = \frac{2P}{m^4},
	\end{aligned}
	\label{eq:variables_adimensionales}
\end{equation}

donde $i = \{\sigma, \omega, \rho\}$. Usando estas variables, la ecuación de movimiento para el campo escalar (\ref{eq:eom_meson_mf_sustituidas}) se escribe como:

\begin{equation}
	(1 - \xsigma) - A_\sigma \left[ \frac{1}{\pi} \sum_N \int_0^{x_{FN}} \frac{\xsigma x^2 dx}{\sqrt{x^2 + \xsigma^2}} - b(1-\xsigma)^2 - c(1-\xsigma)^3 \right] = 0,
	\label{eq:eom_sigma_adimensional}
\end{equation}

donde $x_{Fi} = p_{Fi}/m$ son los momentos de Fermi adimensionales. Las expresiones para densidad de energía (\ref{eq:densidad_energia}) y presión (\ref{eq:presion}), utilizando las expresiones para la densidad bariónica (\ref{eq:densidad_barionica}) y la densidad de 3-isospín (\ref{eq:densidad_isospin}), quedan:

\begin{gather}
	\begin{aligned}
		\tilde{\rho} = &\frac{1}{A_\sigma}(1-\xsigma)^2 + \frac{2}{3}b(1-\xsigma)^3 + \frac{1}{2}c(1-\xsigma)^4 + A_\omega \tilde{n}^2 + \frac{1}{36 \pi^4}A_\rho(x_{Fp}^3-x_{Fn}^3)^2 \\
		&+ \sum_N \frac{2}{\pi^2}\int_0^{x_{FN}} \sqrt{x^2 + \xsigma^2} x^2 dx + \frac{2}{\pi^2}\int_0^{x_{Fe}} \sqrt{x^2 + \left(\frac{m_e}{m}\right)^2} x^2 dx,
		\label{eq:densidad_energia_adimensional}
	\end{aligned} \\
	\begin{aligned}
		\tilde{P} = &- \frac{1}{A_\sigma}(1-\xsigma)^2 - \frac{2}{3}b(1-\xsigma)^3 - \frac{1}{2}c(1-\xsigma)^4 + A_\omega \tilde{n}^2 + \frac{1}{36 \pi^4}A_\rho(x_{Fp}^3-x_{Fn}^3)^2 \\
		&+ \sum_N \frac{2}{3\pi^2}\int_0^{x_{FN}} \frac{x^4 dx}{\sqrt{x^2 + \xsigma^2}} + \frac{2}{3\pi^2}\int_0^{x_{Fe}} \frac{x^4 dx}{\sqrt{x^2 + \left(\frac{m_e}{m}\right)^2}}.
		\label{eq:presion_adimensional}
	\end{aligned}
\end{gather}

Estas expresiones son funciones de la densidad de número de protones, neutrones y electrones. Si queremos describir la materia en el interior de estrellas de neutrones, debemos considerar ligaduras que nos permitirán cerrar el sistema de ecuaciones. Es necesario imponer las mismas condiciones de neutralidad local de carga (\ref{eq:neutralidad_carga}) y equilibrio beta (\ref{eq:equilibrio_beta}) discutidas para el gas ideal degenerado en la sección \ref{sec:gasnpe}, así como la conservación del número de bariones $n_B = n_p+n_n$, obteniendo:

\begin{equation}
	\begin{gathered}
		x_{Fp} = x_{Fe}, \\
		\sqrt{x_{Fn}^2+\xsigma^2} - \sqrt{x_{Fp}^2+\xsigma^2} - \frac{1}{6\pi^2}A_\rho - \sqrt{x_{Fe}^2+\left(\frac{m_e}{m}\right)^2} = 0,\\
		x_{Fp} = (3\pi^2\tilde{n} - x_{Fn}^3)^{1/3}.
	\end{gathered}
	\label{eq:ligaduras_adimensionales}
\end{equation}

De las expresiones adimensionalizadas (\ref{eq:eom_sigma_adimensional} - \ref{eq:ligaduras_adimensionales}) podemos entender el sistema físico descrito por el modelo. Primero, es importante notar que, si bien definimos ocho constantes inicialmente ($g_i$, $m_i$ con $i=\{\sigma, \omega, \rho\}$, $b$, $c$), las ecuaciones solo dependen de cinco combinaciones adimensionales de estas constantes: los cocientes $A_i$, y los parámetros de autointeracción escalar $b$ y $c$. Por lo tanto, el modelo tiene cinco parámetros libres que deben ser ajustados para reproducir datos experimentales o teóricos adicionales. En segundo lugar, podemos deducir de las integrales en las ecuaciones para el campo escalar, la densidad de energía y la presión que el campo escalar $\sigma$ actúa para reducir la masa efectiva de los nucleones disminuyendo la energía por partícula a mayores densidades, mientras que el campo vectorial $\omega_0$ aumenta la energía por partícula debido a su acoplamiento con la densidad bariónica total. Esto indica que el mesón $\sigma$ genera una interacción atractiva entre nucleones, mientras que el mesón $\omega$ genera una interacción repulsiva. Finalmente, el mesón $\rho$ actúa para ajustar la diferencia entre las densidades de protones y neutrones con un efecto ``repulsivo'' ante la asimetría del sistema.

\begin{figure}[h]
	\centering
	\includegraphics[width=\linewidth]{Figuras/materia_estelar_base}
	\caption[Ecuación de estado para el núcleo de estrellas de neutrones.]{Ecuación de estado (izquierda) y variables termodinámicas en función de la densidad de masa (derecha) para el núcleo de estrellas de neutrones empleando teoría relativista de campo medio. $\rho_0=m^4/2$ es el factor de adimensionalización. Se usaron los parámetros $\left(\frac{g_\sigma}{m_\sigma}\right)^2=12.684\,\text{fm}^2$, $\left(\frac{g_\omega}{m_\omega}\right)^2=7.148\,\text{fm}^2$, $\left(\frac{g_\rho}{m_\rho}\right)^2=4.410\,\text{fm}^2$, $b=5.610\times10^{-3}$ y $c=-6.986\times10^{-3}$ como referencia.}
	\label{fig:materiaestelarbase}
\end{figure}


Dados los parámetros libres del modelo, tenemos un sistema cerrado que podemos resolver numéricamente para cada valor de $n_B$, obteniendo la ecuación de estado $\tilde{\rho}(\tilde{P})$, como se muestra en la figura \ref{fig:materiaestelarbase}. Sin embargo, aún es necesario restringir los parámetros que hacen al modelo físicamente consistente. Para ello, acudimos a las propiedades empíricas de la materia nuclear en saturación, descritas a continuación.


\subsection{Materia en Saturación Nuclear}
\label{sec:saturacion}

Si suprimimos los electrones libres del modelo (\ref{eq:lagrangiano}) y consideramos materia nuclear simétrica ($n_p = n_n$), el sistema se reduce a un fluido de nucleones interactuando mediante los campos mesónicos. Este sistema debe reproducir las propiedades de la materia nuclear en saturación, caracterizada por parámetros empíricos que añaden restricciones obligatorias para cualquier teoría microscópica válida \cite{kumarTheoreticalExperimentalConstraints2024}. De estos parámetros, consideraremos cinco que pueden determinarse experimentalmente mediante mediciones en laboratorios terrestres, siendo una herramienta de calibración para los modelos teóricos nucleares.

La densidad de saturación nuclear, $n_0$, define la densidad a la cual la materia nuclear simétrica alcanza su estado de mínima energía de enlace por nucleón, $\frac{B}{A}$. Tras aplicar un análisis bayesiano a una colección de ligaduras de la Teoría Funcional de la Densidad, que incluyen modelos relativistas de campo medio y de Skyrme, se obtienen valores \cite{drischlerBayesianMixtureModel2024}:

\begin{align}
	&n_0 = 0.157 \pm 0.010 \, \text{fm}^{-3} \label{eq:densidad_saturacion}, \\
	&\frac{B}{A} = \frac{\rho(n_0)}{n_0} - m = -15.97 \pm 0.40 \, \text{MeV}. \label{eq:energia_enlace_saturacion}
\end{align}

Este resultado empírico es consistente con estimaciones realizadas a partir de mediciones de dispersión de electrones con violación de paridad en $^{208}$Pb en el experimento PREX \cite{horowitzInsightsNuclearSaturation2020} y ajustes de 1654 núcleos atómicos a un modelo de gota líquida \cite{kumarTheoreticalExperimentalConstraints2024, myersNuclearPropertiesAccording1996}. En nuestro modelo (\ref{eq:lagrangiano}), estas propiedades de saturación se determinan hallando el mínimo de la función $\tfrac{B}{A}(n_B)$ en ausencia de electrones, como se muestra de ejemplo en la figura \ref{fig:saturacionbase}.

\begin{figure}[h]
	\centering
	\includegraphics[width=0.7\linewidth]{Figuras/saturacion_base}
	\caption[Energía de enlace por nucleón en saturación nuclear]{Energía de enlace por nucleón en función de la densidad bariónica para materia nuclear simétrica sin electrones. La densidad de saturación $n_0$ y la energía de enlace por nucleón en saturación $\frac{B}{A}$ se señalan en el mínimo. Se usaron los parámetros $\left(\frac{g_\sigma}{m_\sigma}\right)^2=12.684\,\text{fm}^2$, $\left(\frac{g_\omega}{m_\omega}\right)^2=7.148\,\text{fm}^2$, $\left(\frac{g_\rho}{m_\rho}\right)^2=4.410\,\text{fm}^2$, $b=5.610\times10^{-3}$, $c=-6.986\times10^{-3}$.}
	\label{fig:saturacionbase}
\end{figure}

El módulo de compresibilidad nuclear, $K_0$, caracteriza la rigidez de la materia nuclear ante compresiones alrededor del punto de saturación. Este parámetro es determinante para la extrapolación de la ecuación de estado a densidades supranucleares, y es estimado mediante las resonancias monopolares en núcleos pesados. Considerando un modelo de cadena de núcleos para esta resonancia en $^{208}$Pb, se obtiene \cite{kumarTheoreticalExperimentalConstraints2024, khanConstrainingNuclearEquation2012}:

\begin{equation}
	K_0 = 9n_0^2 \frac{\partial^2}{\partial n^2}\left(\frac{\rho}{n}\right)\bigg|_{n=n_0} = 230 \pm 40 \, \text{MeV}.
	\label{eq:modulo_compresibilidad_empirico}
\end{equation}

Esta cantidad está relacionada algebraicamente con cantidades de nuestro modelo de materia simétrica en saturación mediante:

\begin{equation}
	\begin{gathered}
		\frac{K_0}{3m} = \frac{2}{\pi^2}A_\omega x_F^3 + \frac{x_F^2}{\sqrt{x_F^2 + \xsigma^2}} - \frac{2A_\sigma}{\pi^2}\frac{x_F^3\xsigma^2}{x_F^2+\xsigma^2}F^{-1}, \\
		F = 1 + A_\sigma (1-\xsigma)[2b+3c(1-\xsigma)]+\frac{2A_\sigma}{\pi^2}\int_0^{x_F} \frac{x^4 dx}{(x^2+\xsigma^2)^{3/2}},
	\end{gathered}
	\label{eq:modulo_compresibilidad}
\end{equation}

donde $x_Fp=x_Fn=x_F$.

La energía de simetría, $a_\text{sym}$, cuantifica el costo energético de desviarse de la composición simétrica. Esta cantidad determina la composición de protones y neutrones en materia nuclear densa, afectando directamente las propiedades de las estrellas de neutrones. La pendiente de la energía de simetría, $L_0$, describe la dependencia de la energía de simetría con la densidad de bariones alrededor de la densidad de saturación.

Para estas dos cantidades, tras una revisión de 28 estimaciones tanto de experimentos terrestres como de observaciones astronómicas de estrellas de neutrones, se estiman valores representativos \cite{kumarTheoreticalExperimentalConstraints2024, liUnderstandingAstrophysicalEffects2019}:

\begin{align}
	&a_\text{sym} = \frac{1}{2} \frac{\partial^2}{\partial t^2}\left(\frac{E}{A}\right)\bigg|_{t=0} = 31.6 \pm 2.7 \, \text{MeV}, \label{eq:energia_simetria_empirico} \\
	&L_0 = 3n_0 \frac{\partial a_\text{sym}}{\partial n}\bigg|_{n=n_0} = 58.9 \pm 16 \, \text{MeV}. \label{eq:pendiente_simetria_empirico},
\end{align}

donde $t = (n_n - n_p)/n_B$ es el parámetro de asimetría de isospín. Es de notar que, si bien el valor fiduciario tiene una baja incertidumbre, las estimaciones individuales de $L_0$ varían ampliamente con intervalos de confianza que se estiman entre 20 y 120 MeV debido a la dificultad para acceder a materia altamente asimétrica en el laboratorio \cite{esteeProbingSymmetryEnergy2021}. En nuestro modelo, estas cantidades pueden calcularse como:

\begin{align}
	\frac{a_\text{sym}}{m} &= \frac{x_F^2}{6\sqrt{x_F^2 + \xsigma^2}} + \frac{1}{12\pi^2}A_\rho x_F^3, \label{eq:energia_simetria} \\
	\frac{L_0}{m} &= \frac{x_F^2}{6\sqrt{x_F^2 + \xsigma^2}}\left(1 + \frac{\xsigma^2}{x_F^2 + \xsigma^2}\right) + \frac{1}{4\pi^2}A_\rho x_F^3. \label{eq:pendiente_simetria}
\end{align}

Estas cinco propiedades empíricas ($n_0$, $\frac{B}{A}$, $K_0$, $a_\text{sym}$, $L_0$) definen restricciones que cualquier ecuación de estado microscópica debe satisfacer para ser físicamente viable. En el contexto de la teoría relativista de campo medio, estas propiedades y sus incertidumbres se utilizan para determinar las regiones en el espacio de parámetros que producen ecuaciones de estado físicamente aceptables frente a mediciones nucleares experimentales.

\section{Corteza de la Estrella}
\label{sec:corteza}

El modelo de teoría relativista de campo medio desarrollado es una descripción de la materia nuclear en el régimen de alta densidad, específicamente para densidades bariónicas superiores a aproximadamente $0.1$ fm$^{-3}$ o $\rho_m \gtrsim 1.7 \times 10^{14}$ g/cm$^3$, donde las interacciones nucleón-nucleón dominan la dinámica del sistema y la aproximación de campo medio tiene sentido. Sin embargo, las estrellas de neutrones presentan una estructura estratificada que incluye regiones de menor densidad donde esta aproximación deja de ser aplicable. La corteza de la estrella, que se extiende desde la superficie hasta el núcleo denso, abarca un rango de densidades de varios órdenes de magnitud y requiere tratamientos teóricos distintos según el régimen de densidad considerado \cite{shapiroBlackHolesWhite2008}.

El alcance del presente estudio se concentra en la descripción microscópica de la materia nuclear en el régimen de altas energías en las estrellas de neutrones, donde las densidades superan varias veces la densidad de saturación $n_0$. Para la descripción de la corteza utilizamos las ecuaciones de estado estándar BPS y BBP. Para densidades inferiores al punto de goteo de neutrones ($n_{\text{drip}} \approx 2.4\times10^{-4}$ fm$^{-3}$ o $\rho_m \approx 4 \times 10^{11}$ g/cm$^3$), la materia se compone de una red cristalina de núcleos pesados embebidos en un gas degenerado de electrones, configuración conocida como corteza externa. En este régimen, la ecuación de estado BPS (Baym, Pethick y Sutherland, 1971) \cite{baymGroundStateMatter1971} es una descripción consistentemente basada en la minimización de la energía del estado base de núcleos atómicos para cada densidad, junto con la contribución del gas de electrones relativista. El modelo BPS determina la composición nuclear óptima mediante la competencia entre la energía de masa nuclear, la energía de Coulomb de la red, y la presión del gas electrónico. Por encima de la densidad de goteo de neutrones y hasta densidades del orden de $0.2$ fm$^{-3}$ o $\rho_m \approx 3.2 \times 10^{14}$ g/cm$^3$ se encuentra la corteza interna, donde los núcleos coexisten con un gas de neutrones libres además del gas de electrones. En este régimen, la ecuación de estado BBP (Baym, Bethe y Pethick, 1971) \cite{baymNeutronStarMatter1971} describe la materia mediante un modelo de gota líquida que considera las contribuciones energéticas de los núcleos, el gas de neutrones libres, y las interacciones nucleares efectivas de la red. Este modelo determina autoconsistentemente la composición nuclear y la fracción de neutrones libres mediante la minimización de la energía total del sistema, que incluye términos de energía de superficie, energía de Coulomb, y energía de simetría nuclear.

Para construir una ecuación de estado unificada que abarque todo el rango de densidades presente en la estrella, desde la superficie hasta el núcleo, empleamos el método de interpolación \href{https://es.wikipedia.org/wiki/Interpolador_c\%C3\%BAbico_de_Hermite}{PCHIP} (Piecewise Cubic Hermite Interpolating Polynomial), que garantiza una transición suave y monótona entre las diferentes ecuaciones de estado. Este método de interpolación preserva la forma de los datos y evita oscilaciones no físicas que podrían introducir otros métodos de interpolación polinómica. Definimos la densidad de empalme entre la ecuación de estado BBP y nuestro modelo de teoría relativista de campo medio entre $n_B = 0.043$ fm$^{-3}$ y $n_B = 0.063$ fm$^{-3}$, correspondiente a una densidad de masa de entre $7.20 \times 10^{13}$ g/cm$^3$ y $1.05 \times 10^{14}$ g/cm$^3$. Esta elección se fundamenta en la necesidad de garantizar la causalidad del fluido en toda la estrella, es decir, que la velocidad del sonido $c^2_s = dP/d\rho$ no exceda la velocidad de la luz $c$ en ningún punto. El resultado de esta interpolación para un conjunto de parámetros de ejemplo y su comprobación de causalidad se muestra en la figura \ref{fig:causalidadmateriabase}.

\begin{figure}[h]
	\centering
	\includegraphics[width=0.7\linewidth]{Figuras/causalidad_materia_estelar_completa}
	\caption[Ecuación de estado unificada y causalidad.]{Ecuación de estado unificada para la estrella de neutrones, construida mediante interpolación PCHIP entre las ecuaciones de estado BPS, BBP y el modelo TRCM. Las X marcan los límites de la región de empalme. El panel inferior muestra el cuadrado de la velocidad del sonido $c_s^2$ para verificar la condición de causalidad. Se usaron los parámetros $\left(\frac{g_\sigma}{m_\sigma}\right)^2=12.684\,\text{fm}^2$, $\left(\frac{g_\omega}{m_\omega}\right)^2=7.148\,\text{fm}^2$, $\left(\frac{g_\rho}{m_\rho}\right)^2=4.410\,\text{fm}^2$, $b=5.610\times10^{-3}$, $c=-6.986\times10^{-3}$.}
	\label{fig:causalidadmateriabase}
\end{figure}

% Figura: Interpolación PCHIP
% [Espacio para figura mostrando P vs rho para BPS, BBP, interpolación PCHIP, y modelo RMFT, 
% con énfasis en la región de empalme en n_B = 0.062 fm^{-3}. Incluir panel secundario 
% mostrando la velocidad del sonido v_s/c para verificar causalidad (v_s < c) en toda la región.]
