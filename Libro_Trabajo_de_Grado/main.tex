\documentclass[8pt, letterpaper]{report}

% Paquetes --------------------------------------------------------------------------
\usepackage{bm}
\usepackage{setspace}
\usepackage{titlesec}
\usepackage[colorlinks=true,linkcolor=blue,citecolor=blue,urlcolor=blue]{hyperref}
\usepackage{fancyhdr}
\usepackage{array}
\usepackage{amsfonts}
\usepackage{mathtools}
\usepackage{float}
\usepackage{caption}
\usepackage{fixmath}
%\captionsetup[figure]{name=Figura}
\counterwithout{figure}{chapter}
\counterwithout{table}{chapter}
\usepackage{tocloft}
\usepackage{amssymb,amsmath,amsthm} % Símbolos matemáticos
\usepackage[spanish]{babel} % Idioma
\usepackage[table,xcdraw]{xcolor}
\usepackage[utf8]{inputenc} % Acentos y otros símbolos 
\usepackage{enumerate} % Páginas
\usepackage{graphicx}
\usepackage{multirow}
\usepackage{physics}
\usepackage{subcaption}
\usepackage{multicol}
\usepackage{siunitx}
\usepackage{color}
\usepackage{braket}
\usepackage[nottoc]{tocbibind}
%\usepackage{natbib}
%\usepackage[square,sort,comma,numbers]{natbib} %Para ver solo números
%\usepackage[square,sort,comma]{natbib}
\usepackage[style=ieee, backend=biber, maxnames=2]{biblatex}
\addbibresource{PropuestadeTesis.bib}
\setlength{\parindent}{0pt} %Quitar indentación
\decimalpoint

%Definiciones paras las ecuaciones y las figuras --------------------------------------

% Redefinir el comando \theequation para incluir el número de sección y subsección
\renewcommand{\theequation}{\thechapter.\arabic{equation}}

\DeclareCaptionFormat{myformat}{\textbf{#1#2} #3}
\captionsetup[figure]{format=myformat}

\addto\captionsspanish{\renewcommand*\contentsname{Tabla de contenido}}
\addto\captionsspanish{\renewcommand*\listtablename{Lista de tablas}}
\addto\captionsspanish{\renewcommand*\tablename{Tabla}}
\addto\captionsspanish{\renewcommand*\listfigurename{Lista de figuras}}

\renewcommand{\cftfigpresnum}{Figura }
\setlength{\cftfignumwidth}{5em}
\renewcommand{\cfttabpresnum}{Tabla }
\setlength{\cfttabnumwidth}{5em}

\tocloftpagestyle{fancy}
\setlength\cftbeforetoctitleskip{1cm}
\setlength\cftbeforeloftitleskip{1cm}
\setlength\cftbeforelottitleskip{1cm}


% Reiniciar el contador de ecuaciones al principio de cada subsección -------------------------

\makeatletter
\@addtoreset{equation}{section}
\makeatother

% Configuración de márgenes
\usepackage[left=3cm, right=2.5cm, top=3cm, bottom=3cm]{geometry}

% Configuración del interlineado
\renewcommand{\baselinestretch}{1.5}

% Configuración de títulos de secciones
\titleformat{\chapter}[display]
  {\normalfont\huge\bfseries}{\chaptertitlename\ \thechapter}{20pt}{\Huge}
\titlespacing*{\chapter}{0pt}{-50pt}{40pt}

% Título de la tesis
\title{Ecuaciones de Estado de Estrellas de Neutrones mediante la Teoría Relativista de Campo Medio}
\author{Nicolas Mantilla Molina}
%\date{Fecha}

% Configuración de puntos en la tabla de contenidos
\renewcommand{\cftchapleader}{\cftdotfill{\cftsecdotsep}}
\renewcommand{\cftchapdotsep}{\cftdotsep}

% Configuración de encabezado y pie de página
\pagestyle{fancy}
\fancyhf{} % Limpia los encabezados y pies de página predeterminados
\fancyhead[L]{\leftmark}
\fancyhead[R]{\thepage}

% Redefinir el estilo de página de la primera página de cada capítulo
\fancypagestyle{plain}{%
  \fancyhf{} % Limpia los encabezados y pies de página predeterminados
  \fancyhead[L]{}
  \fancyhead[C]{Universidad Industrial de Santander}
  \fancyhead[R]{\thepage}
}
\titlespacing*{\chapter}{0pt}{50pt}{20pt} %Cambio los valores del espacio entre el título del cap y el encabezado

\begin{document}

% Portada
\begin{titlepage}
    \centering
    %\begin{minipage}[b]{0.45\textwidth}
    %    \includegraphics[width=\textwidth]{Figuras/logouis.png}\par\vspace{1cm}
    %\end{minipage}
    %\hfill
    %\begin{minipage}[b]{0.45\textwidth}
    %    \includegraphics[width=\textwidth]{Figuras/loGOTS.png}\par\vspace{1cm}
    %\end{minipage}
    \vspace{5cm}
    \rule{\textwidth}{2pt}
    {\LARGE\bfseries ECUACIONES DE ESTADO DE ESTRELLAS DE NEUTRONES MEDIANTE LA TEORÍA RELATIVISTA DE CAMPO MEDIO\par}
    \rule{\textwidth}{2pt}
    \vfill
    {\Large\itshape Nicolas Mantilla Molina\par}
    \vspace{1cm}
    {\Large\itshape Trabajo de grado para optar por el título de:  \\
    \textbf{Físico} \par}
    
    \vspace{.8 cm}
    {\Large\itshape Directora: \par}
    {\Large\itshape Laura Marcela Becerra Bayona, PhD \par}
    %\vfill
    \vspace{0.5cm}
    {\Large\itshape Co-directores: \par}
    {\Large\itshape José Fernando Rodriguez Ruiz, PhD \par}
     {\Large\itshape Luis Alberto Núñez de Villavicencio, PhD \par}
    \vfill
    %\vspace{.8cm}
    {\scshape\Large Universidad  Industrial De Santander \par}
    {\scshape\Large Facultad de Ciencias \par}
    {\scshape\Large Escuela de Física \par}
    {\scshape\Large Bucaramanga \par}
    {\scshape\Large 2025 \par}
\end{titlepage}


% Dedicatoria
\clearpage
\thispagestyle{empty}
\vspace*{\fill}
\begin{flushright}
    {\Large\itshape Dedicatoria\par}
    \vspace{0.5cm}
	Cuerpo de la dediatoria
    
\end{flushright}
\vspace*{\fill}
\clearpage

% Agradecimientos
\clearpage
\thispagestyle{empty}
\vspace*{\fill}
\begin{flushleft}
    {\huge\bfseries Agradecimientos\par}
    \vspace{0.5cm}
	Cuerpo de los agradecimientos
    
\end{flushleft}
\vspace*{\fill}
\clearpage



% Tabla de contenido
\pagestyle{empty}
\tableofcontents


% Lista de figuras
\clearpage
\listoffigures
\clearpage

% Página de resumen
\clearpage
%\thispagestyle{empty}
\vspace*{1cm} % Ajusta este valor según tu preferencia
\begin{flushleft}
    {\huge\bfseries Resumen\par}
    \vspace{0.5cm}
    En el presente trabajo se propone obtener Ecuaciones de Estado (EdE) de una Estrella de Neutrones (EN) mediante la Teoría Relativista de Campo Medio (TRCM). Las EN son objetos cuyas condiciones extremas permiten estudiar la materia nuclear a altas densidades, inalcanzables en laboratorio con la tecnología actual. A través de las EdE, se busca modelar la relación entre  la presión y la densidad de energía en el núcleo estelar, lo cual resulta fundamental para describir propiedades observables de la EN como su masa,  radio y  deformación de marea. Este estudio responde a la necesidad de conectar modelos teóricos con datos experimentales y observacionales, especialmente  aquellos obtenidos por misiones como LIGO, VIRGO y NICER.
    
    Las EdE construidas contienen parámetros libres que describen las interacciones y propiedades microscópicas de la materia. Este proyecto tiene como objetivo construir y ajustar modelos de EdE utilizando datos astrofísicos recientes. Adicionalmente, se investigará la capacidad de la EdE hallada para predecir objetos en el Mass-Gap, un rango de valores entre estrellas de neutrones masivas y agujeros negros ligeros, cuya composición sigue siendo debatida. Este enfoque permitirá modelar la estructura de las EN y contribuir al conocimiento de la materia nuclear en condiciones extremas.

    
\end{flushleft}



% Añadir la página de resumen al PDF pero no a la tabla de contenidos
%\addcontentsline{toc}{chapter}{Resumen}


%Capítulo 1
\chapter{Introducción}
\thispagestyle{fancy}

% Introducción

La materia en las estrellas de neutrones (EN) alcanza uno de los estados más extremos y densos conocidos en el universo observable. Estos objetos compactos, resultado del colapso gravitacional de estrellas masivas (sugerido por primera vez en 1934 por Baade y Zwicky \cite{baadeRemarksSuperNovaeCosmic1934}), comprimen la materia a densidades que superan las condiciones terrestres \cite{glendenningCompactStarsNuclear2000}. Las propiedades y características de la materia nuclear a densidades tan altas son objeto de estudio en la actualidad. Estudiar estos objetos estelares abre las puertas a la comprensión de la materia densa, ofreciendo un laboratorio natural donde se pueden contrastar modelos teóricos mediante observaciones astrofísicas. En adición, es posible corroborar indirectamente la validez de teorías nucleares y subatómicas bajo condiciones actualmente inalcanzables en el laboratorio \cite{raaijmakersConstraintsDenseMatter2021}.

Para este trabajo, el estudio de las EN se centra en las Ecuaciones de Estado (EdE), que describen la relación entre presión, densidad de energía y temperatura, siendo clave para resolver las ecuaciones de Tolman-Oppenheimer-Volkoff (TOV). Estas ecuaciones determinan características macroscópicas de las EN, como la relación masa-radio y las propiedades de marea, que han sido objeto de observaciones cada vez más detalladas gracias a misiones como NICER \cite{romaniPSRJ09520607Fastest2022, fonsecaRefinedMassGeometric2021, antoniadisMassivePulsarCompact2013}, LIGO y VIRGO \cite{collaborationGWTC21DeepExtended2022,theligoscientificcollaborationGW170817ObservationGravitational2017, theligoscientificcollaborationGW190814GravitationalWaves2020}. Con estos avances, los modelos teóricos se ven enriquecidos y enfrentan desafíos, como la identificación de límites para la masa máxima de una EN posible en la naturaleza, o la existencia de una ``brecha de masa'' (Mass-Gap) entre las EN de mayor masa y los agujeros negros más ligeros \cite{shaoNeutronStarBlack2022a}.

En el desarrollo de EdE, la Teoría Relativista de Campo Medio (TRCM) surge como una teoría efectiva para modelar interacciones nucleares bajo condiciones extremas de densidad \cite{waleckaTheoryHighlyCondensed1974,waleckaRelativisticNuclearManyBody1986}. La TRCM permite construir EdE que consideran distintas composiciones y acoplamientos entre partículas, posibilitando ajustes a los modelos basados en observaciones y datos experimentales cercanos a la densidad de saturación nuclear \cite{dutraRelativisticMeanFieldHadronic2014}. Sin embargo, la estructura interna de las EN sugiere la posible presencia de fases exóticas de materia, tales como hiperones o materia de quarks, desafiando aún más las formulaciones de EdE, especialmente en aquellas estrellas que alcanzan masas cercanas a $2.5$ masas solares \cite{lopesNatureMassgapObject2022}. La búsqueda de una EdE que logre reproducir observaciones actuales es, por lo tanto, crucial no solo para describir el comportamiento de la materia nuclear densa, sino también para realizar estudios más detallados sobre el destino y evolución final de estos objetos.

En esta propuesta de investigación, se plantea construir modelos de EdE mediante la TRCM y ajustar sus parámetros a partir de las observaciones astrofísicas actuales. La exploración del espacio de parámetros contribuirá a identificar las implicaciones físicas de la materia en su forma más extrema según las observaciones y brindará predicciones de la teoría empleada sobre la naturaleza de los objetos ubicados en el Mass-Gap.

\section{Notación y Convenciones}

A lo largo de este trabajo empleamos convenciones de notación y diferentes sistemas de unidades dependiendo del contexto físico específico. Para el desarrollo de la Relatividad General y el estudio de la estructura macroscópica de objetos compactos, empleamos unidades geometrizadas donde la velocidad de la luz $c$ y la constante gravitacional de Newton $G$ se establecen igual a la unidad $c = G = 1$. Para el tratamiento de la física nuclear y la construcción de ecuaciones de estado microscópicas, utilizamos unidades naturales donde la velocidad de la luz $c$ y la constante de Planck reducida $\hbar$ se fijan como unitarias $c = \hbar = 1$. En cuanto a las convenciones de notación, las letras griegas $\{\mu, \nu, \rho, \sigma, \ldots\}$ denotan índices tensoriales del conjunto $\{0,1,2,3\}$, mientras que las letras latinas $\{i, j, k, \ldots\}$ denotan componentes del conjunto $\{1,2,3\}$. El tensor métrico $g_{\mu\nu}$ tiene signatura $(+,-,-,-)$ y utilizamos la convención de suma de Einstein para índices repetidos.

%Se plantea el problema de investigación a estudiar y se brinda un contexto más amplio en la sección \ref{Planteamiento}. A continuación, en la sección \ref{MarcoTeorico}, se expondrán los principales conceptos requeridos para abordar el problema de investigación. Los objetivos del trabajo se presentan en la sección \ref{Objetivos}, mientras que, en la sección \ref{Metodologia}, se presenta la metodología planteada para alcanzar los objetivos de investigación, así como el cronograma de actividades para el desarrollo del trabajo. 









 

%Capítulo 2
%\chapter{Macrofísica: Objetos Compactos}
%\thispagestyle{fancy}

% Marco teórico - Macrofísica


%\section{Relatividad General}


%\section{Solución de Schwarzschild}


%\section{Ecuaciones de Estructura}

%\section{Observaciones de Estrellas de Neutrones}

%masas, radios, etc.

\chapter{Macrofísica: Objetos Compactos}

Los objetos compactos representan uno de los laboratorios naturales más extremos del universo, en los que la materia alcanza densidades, presiones y energías extremas que permiten extender nuestra comprensión de la física fundamental. Las estrellas de neutrones, en particular, constituyen el resultado final del colapso gravitacional de estrellas masivas, comprimiendo la materia a densidades que alcanzan varios órdenes de magnitud por encima de la densidad nuclear, en entornos de presión y temperatura que son extremadamente difíciles de reproducir en laboratorios terrestres \cite{baadeSuperNovae1934}. El estudio de estos objetos requiere necesariamente del escenario de la Relatividad General de Einstein, donde la curvatura del espacio-tiempo responde al contenido energético del sistema y viceversa.

En este capítulo desarrollamos los fundamentos teóricos necesarios para describir la estructura y propiedades de las estrellas de neutrones desde una perspectiva macroscópica. Comenzamos con una revisión de los fundamentos de la Relatividad General, estableciendo las ecuaciones de campo de Einstein que gobiernan la dinámica del espacio-tiempo. Posteriormente, exploramos la forma general de geometrías esféricamente simétricas, que proporciona un modelo básico para comprender la estructura de los objetos compactos. A partir de estos fundamentos, derivamos las ecuaciones de estructura que relacionan las propiedades microscópicas de la materia con las características macroscópicas observables de las estrellas de neutrones. Todas las secciones anteriores están fundamentadas en base al texto de Misner et al. \cite{misnerGravitation2017}. Finalmente, presentamos una revisión de las observaciones astronómicas más relevantes que han refinado nuestro entendimiento de estos objetos en la última década.

\section{Relatividad General}

La teoría de la Relatividad General, formulada por Einstein en 1915 \cite{einsteinFeldgleichungenGravitation1915}, es la descripción más exitosa de la gravitación como manifestación de la curvatura del espacio-tiempo. Esta teoría establece que la presencia de materia y energía modifica la geometría del espacio-tiempo, y que esta curvatura, a su vez, determina el movimiento de la materia y la propagación de la energía. La Relatividad General se fundamenta en dos principios básicos: el principio de equivalencia, según el cual los efectos de un campo gravitacional uniforme son localmente indistinguibles de los efectos de una aceleración uniforme, permitiendo que la gravitación pueda ser ``eliminada'' localmente mediante la elección apropiada de un sistema de coordenadas en caída libre; y el principio de covariancia general, el cual exige que las leyes de la física tengan la misma forma en todos los sistemas de coordenadas, requiriendo que las ecuaciones de la teoría sean expresadas en términos de tensores para asegurar su invariancia bajo transformaciones generales de coordenadas. Una representación esquemática del principio de equivalencia puede verse en la figura \ref{fig:equivalence_principle}, donde la esfera roja aparentemente cae en ambos sistemas, uno es un cohete acelerado, el otro la superficie de la tierra, siendo los dos sistemas de referencia indistinguibles.

\begin{figure}[h]
	\centering
	\includegraphics[width=0.7\linewidth]{Figuras/512px-Elevator_gravity.svg}
	\caption{Representación esquemática del principio de equivalencia. Markus Poessel.}
	\label{fig:equivalence_principle}
\end{figure}

\subsection{Ecuaciones de Campo de Einstein}

Las ecuaciones de campo de Einstein son un conjunto de ecuaciones diferenciales no lineales cuya forma compacta es:

\begin{equation}
	G_{\mu\nu} = 8\pi T_{\mu\nu},
	\label{eq:einstein_field}
\end{equation}

donde $G_{\mu\nu} = R_{\mu\nu} - \frac{1}{2}Rg_{\mu\nu}$ es el tensor de Einstein construido a partir del tensor de Ricci $R_{\mu\nu}$ y el escalar de curvatura $R$, y $T_{\mu\nu}$ es el tensor de energía-momento que describe las propiedades de la materia y energía presentes en el espacio-tiempo. Este conjunto de ecuaciones describe cómo la distribución de materia y energía determina la curvatura del espacio-tiempo, y cómo esta curvatura afecta el movimiento de la materia y la energía de vuelta. Las ecuaciones de Einstein satisfacen automáticamente las leyes de conservación del momentum y la energía a través de la identidad de Bianchi $G_{\mu\nu}{}^{;\rho} = 0$, lo que implica:

\begin{equation}
	\nabla_\rho T^{\mu\nu} = T^{\mu\nu}{}_{;\rho} = 0.
\end{equation}

Para describir la materia en el interior de las estrellas de neutrones, consideramos un modelo de fluido perfecto. El tensor de energía-momento es:

\begin{equation}
	T_{\mu\nu} = (\rho + P)u_\mu u_\nu - Pg_{\mu\nu},
	\label{eq:tensor_fluido_perfecto}
\end{equation}

donde $\rho$ es la densidad de energía del fluido, $P$ es la presión isótropa, y $u^\mu$ es la cuadrivelocidad del fluido, normalizada según $u^\mu u_\mu = 1$. Esta descripción asume que no hay flujos de calor, viscosidad, ni esfuerzos de corte en el fluido, aproximación válida para escalas de tiempo mucho mayores que los tiempos de relajación hidrodinámicos.

\section{Simetría Esférica}

Para describir la estructura de las estrellas de neutrones, vamos a resolver las ecuaciones de Einstein asumiendo simetría esférica y condiciones estáticas. La forma general de estas soluciones establece el punto de partida para entender tanto el interior como el exterior de objetos compactos con simetría esférica, en el marco de la teoría de la relatividad general. El elemento de línea de un sistema estático y esféricamente simétrico puede escribirse como:

\begin{equation}
	ds^2 = e^{2\phi(r)}dt^2 - e^{2\lambda(r)}dr^2 - r^2(d\theta^2 + \sin^2\theta d\varphi^2),
	\label{eq:metrica_esferica_general}
\end{equation}

donde $\phi(r)$ y $\lambda(r)$ son funciones arbitrarias de la coordenada radial $r$, y hemos utilizado coordenadas esféricas $(t,r,\theta,\varphi)$. La simetría esférica impone que la métrica sea invariante bajo rotaciones en el grupo $SO(3)$, mientras que la condición estática requiere la ausencia de términos cruzados $dt dr$, $dt d\theta$, y $dt d\varphi$, así como independencia explícita de la coordenada temporal. La métrica (\ref{eq:metrica_esferica_general}) representa la forma más general compatible con estas simetrías \cite{misnerGravitation2017}. Una representación gráfica de la geometría del espacio-tiempo que contiene una estrella con simetría esférica, estática, puede apreciarse en la figura \ref{fig:geometry-misner}, donde la región blanca representa la geometría exterior (fuera de la estrella), y la zona gris la región interior (dentro de la estrella).


%Para la métrica esféricamente simétrica (\ref{eq:metrica_esferica_general}), las componentes no triviales del tensor de Ricci son:
%
%\begin{align}
%	R_{tt} &= e^{2(\phi-\lambda)}\left[\phi'' + \phi'^2 - \phi'\lambda' + \frac{2\phi'}{r}\right], \\
%	R_{rr} &= -\phi'' - \phi'^2 + \phi'\lambda' + \frac{2\lambda'}{r}, \\
%	R_{\theta\theta} &= e^{-2\lambda}\left[r(\phi' - \lambda') - 1\right] + 1, \\
%	R_{\varphi\varphi} &= \sin^2\theta \cdot R_{\theta\theta},
%\end{align}
%
%donde $\phi' = \frac{d\phi(r)}{dr}$. Un resultado relevante para geometrías esféricamente simétricas es la posibilidad de expresar la función métrica $\lambda(r)$ en términos de una función de masa $m(r)$. Definiendo:
%
%\begin{equation}
%	e^{-2\lambda} = 1 - \frac{2m(r)}{r},
%	\label{eq:definicion_masa}
%\end{equation}
%
%la componente $R_{\theta\theta}$ del tensor de Ricci puede escribirse como:
%
%\begin{equation}
%	R_{\theta\theta} = r^2 e^{-2\lambda} \frac{1}{4\pi r^2}\frac{dm}{dr} - 1 + e^{-2\lambda}.
%\end{equation}

%Esta parametrización resulta especialmente útil porque la función $m(r)$ admite una interpretación física directa como la masa gravitacional contenida dentro de una esfera de radio $r$.

\begin{figure}[h]
	\centering
	\includegraphics[width=0.7\linewidth]{Figuras/geometry-misner}
	\caption{Esquema de la geometría del espacio-tiempo de una estrella con simetría esférica. Tomado de \cite{misnerGravitation2017}.}
	\label{fig:geometry-misner}
\end{figure}

La solución general esférica (\ref{eq:metrica_esferica_general}) incluye varios casos de interés físico. En la región de vacío, cuando $T_{\mu\nu} = 0$, las ecuaciones de Einstein $G_{\mu\nu} = 0$ determinan completamente las funciones $\phi(r)$ y $\lambda(r)$, conduciendo a la solución exterior de Schwarzschild \cite{schwarzschildGravitationalFieldMass1999}. En el interior estelar, en presencia de materia con tensor de energía-momento (\ref{eq:tensor_fluido_perfecto}), las ecuaciones de Einstein proporcionan relaciones diferenciales entre $\phi(r)$, $\lambda(r)$, y las variables del fluido $\rho(r)$ y $P(r)$. En la superficie estelar, en $r = R$, las soluciones interior y exterior deben coincidir de manera continua, imponiendo condiciones de frontera específicas.

Para que la solución sea físicamente aceptable, las funciones métricas deben satisfacer condiciones de regularidad apropiadas. En el centro ($r = 0$) se requiere que $m(0) = 0$ y $\phi'(0) = 0$ para evitar posibles singularidades físicas o de coordenadas. En todo el espacio se debe cumplir que $e^{2\lambda} > 0$ y $e^{2\phi} > 0$ para mantener la signatura métrica consistentemente en el espacio-tiempo. En el infinito se exige que $\phi(\infty) = 0$ y $\lambda(\infty) = 0$ para recuperar la métrica de Minkowski, es decir, que en el infinito no haya influencia gravitacional de la estrella y el espacio-tiempo sea plano.

%Con la solución general esférica (\ref{eq:metrica_esferica_general}) y la parametrización de masa (\ref{eq:definicion_masa}) es posible derivar las ecuaciones de estructura que describen las propiedades al interior de las estrellas de neutrones y otros objetos compactos, como veremos en la siguiente sección.

\section{Ecuaciones de Estructura}

Las ecuaciones de estructura determinan la relación entre la distribución de materia en el interior de una estrella y la geometría del espacio-tiempo que genera. Para estrellas de neutrones, estas ecuaciones conectan las propiedades microscópicas de la materia densa con las características macroscópicas observables. En este sistema físico, el tensor de energía-momento del fluido perfecto (\ref{eq:tensor_fluido_perfecto}) actúa como fuente de las ecuaciones de Einstein (\ref{eq:einstein_field}). Las componentes del tensor de energía-momento en coordenadas esféricas son:

\begin{align}
	T_t^{\phantom{t}t} &= -\rho(r), \\
	T_r^{\phantom{r}r} &= T_\theta^{\phantom{\theta}\theta} = T_\varphi^{\phantom{\varphi}\varphi} = P(r).
\end{align}

Las ecuaciones de campo $G_{\mu\nu} = 8\pi T_{\mu\nu}$ establecen un sistema de ecuaciones diferenciales para las funciones métricas. Las componentes independientes son:

\begin{itemize}
	\item Componente $(t,t)$:
	\begin{equation}
		e^{-2\lambda}\left(\frac{2\lambda'}{r} - \frac{1}{r^2}\right) + \frac{1}{r^2} = 8\pi\rho.
		\label{eq:einstein_tt}
	\end{equation}
	\item Componente $(r,r)$:
	\begin{equation}
		e^{-2\lambda}\left(\frac{2\phi'}{r} + \frac{1}{r^2}\right) - \frac{1}{r^2} = 8\pi P.
		\label{eq:einstein_rr}
	\end{equation}
	\item Componente $(\theta,\theta)$:
	\begin{equation}
		e^{-2\lambda}\left[\phi'' + \phi'^2 - \phi'\lambda' + \frac{\phi' - \lambda'}{r}\right] = 8\pi P.
		\label{eq:einstein_theta}
	\end{equation}
\end{itemize}

Un resultado relevante para geometrías esféricamente simétricas es la posibilidad de expresar la función métrica $\lambda(r)$ en términos de una función de masa $m(r)$. Definiendo:

\begin{equation}
	e^{-2\lambda} = 1 - \frac{2m(r)}{r},
	\label{eq:definicion_masa}
\end{equation}

y utilizando esta parametrización en la ecuación (\ref{eq:einstein_tt}), obtenemos:

\begin{equation}
	\frac{dm}{dr} = 4\pi r^2 \rho(r),
	\label{eq:masa}
\end{equation}

donde la función $m(r)$ representa la masa inercial total contenida dentro de una esfera de radio $r$, y su interpretación física se hace clara al integrar la ecuación (\ref{eq:masa}):

\begin{equation*}
	m(r) = \int_0^r 4\pi r'^2 \rho(r') dr'.
\end{equation*}

Para sistemas asintóticamente planos como en este caso, la masa inercial coincide con la masa gravitacional activa del límite Newtoniano \cite{vollickMeaningVariousMass2021}, la cual contiene la masa en reposo total, la energía interna total y la energía potencial gravitacional dentro del radio $r$ \cite{misnerGravitation2017}. De la ecuación (\ref{eq:einstein_rr}) y usando la parametrización (\ref{eq:definicion_masa}), la función métrica $\phi(r)$ satisface:

\begin{equation}
	\frac{d\phi}{dr} = \frac{m + 4\pi r^3 P}{r(r - 2m)}.
	\label{eq:phi}
\end{equation}

La ecuación (\ref{eq:phi}) determina la componente temporal de la métrica una vez conocidas las funciones $m(r)$ y $P(r)$. Finalmente, utilizando (\ref{eq:definicion_masa}), (\ref{eq:masa}) y (\ref{eq:phi}) para sustituir $\phi''$, $\phi'$ y $\lambda'$ de (\ref{eq:einstein_theta}), obtenemos la ecuación de equilibrio hidrostático relativista, también llamada ecuación de Tolman-Oppenheimer-Volkoff (TOV) \cite{oppenheimerMassiveNeutronCores1939}:

\begin{equation}
	\frac{dP}{dr} = -\frac{(\rho + P)(m + 4\pi r^3 P)}{r(r - 2m)}.
	\label{eq:tov}
\end{equation}

Esta importante ecuación describe el equilibrio entre la presión del fluido y la atracción gravitacional, incluyendo correcciones relativistas necesarias para sostener objetos compactos y permitir su estabilidad. Para resolver el sistema de ecuaciones de estructura (\ref{eq:masa} - \ref{eq:tov}), se requiere de una ecuación de estado que relacione la presión y la densidad:

\begin{equation}
	P = P(\rho) \quad \text{o,} \quad \rho = \rho(P).
	\label{eq:ecuacion_estado}
\end{equation}

% Además, como condiciones de la estrella, se requiere tener una densidad central diferente de cero $\rho(0) = \rho_c > 0$ y se determina el limite exterior de la estrella $r = R$ como el punto donde la presión se anula $P(R) = 0$. En este punto, la masa total de la estrella es $M = m(R)$.

La solución se obtiene integrando numéricamente desde el centro hacia afuera, comenzando con una densidad central $\rho(0) = \rho_c > 0$ dada. El radio estelar $R$ se determina por la condición $P(R) = 0$, y la masa total $M = m(R)$ resulta de la integración de la ecuación de masa. En el límite no relativista ($M \ll R$ y $P \ll \rho$), la ecuación TOV se reduce a la ecuación clásica de equilibrio hidrostático:

\begin{equation}
	\frac{dP}{dr} = -\rho \frac{m}{r^2}.
\end{equation}

Para estrellas de neutrones, las correcciones relativistas en el sistema de ecuaciones son necesarias debido a su alta compacidad $M/R \sim 0.2-0.4$. Este sistema proporciona la base teórica para predecir las propiedades macroscópicas de las estrellas de neutrones a partir de modelos microscópicos de la materia nuclear densa.

\section{Observaciones de Estrellas de Neutrones}
\label{sec:obsNS}

Las observaciones astronómicas de estrellas de neutrones han experimentado una revolución en las últimas décadas \cite{pianMergersBinaryNeutron2021}, proporcionando restricciones cada vez más precisas sobre las propiedades de la materia densa. Estas mediciones permiten contrastar los modelos teóricos de ecuaciones de estado con la realidad física de estos objetos extremos. Las estrellas de neutrones se observan a través de múltiples canales electromagnéticos \cite{glendenningCompactStarsNuclear2000} y, recientemente, mediante ondas gravitacionales \cite{fonsecaNANOGravNineyearData2016}. Los principales métodos observacionales incluyen:

\begin{itemize}
	\item \textbf{Cronometraje de Pulsares}: Los pulsares son estrellas de neutrones altamente magnetizadas que emiten haces de radiación electromagnética. La precisión extrema en la medición de sus períodos de rotación permite determinar masas través del análisis orbital en sistemas binarios.
	
	\item \textbf{Astronomía de Rayos X}: Las estrellas de neutrones en sistemas binarios acretantes o como objetos aislados calientes pueden ser observadas en rayos X. Las misiones como NICER han revolucionado las mediciones de radio a través del modelado de puntos calientes superficiales.
	
	\item \textbf{Ondas Gravitacionales}: Las fusiones de estrellas de neutrones binarias, detectadas por LIGO-Virgo, proporcionan información sobre las propiedades de marea y la ecuación de estado a través de las modificaciones que introducen en la forma de onda gravitacional.
\end{itemize}

\subsection{Mediciones de Masa}

Las mediciones de masa de estrellas de neutrones se basan principalmente en el efecto Shapiro en sistemas binarios de pulsares. Este efecto relativista permite determinar las masas con alta precisión a partir del retraso temporal que experimenta la señal del pulsar al atravesar el campo gravitacional de su compañera.

Las mediciones más precisas incluyen:

\begin{itemize}
	\item \textbf{PSR J0740+6620}: Con una masa de $2.08 \pm 0.07 \, \masasol$ \cite{fonsecaRefinedMassGeometric2021}, representa la medición más precisa de una estrella de neutrones masiva.
	
	\item \textbf{PSR J0348+0432}: Una estrella de neutrones con masa $2.01 \pm 0.04 \, \masasol$ \cite{antoniadisMassivePulsarCompact2013}, crucial para establecer el límite inferior de la masa máxima.
	
	\item \textbf{PSR J0952-0607}: Con una masa reportada de $2.35 \pm 0.17 \, \masasol$ \cite{romaniPSRJ09520607Fastest2022}, aunque con mayor incertidumbre debido al método de determinación basado en espectrofotometría.
\end{itemize}

Estas observaciones establecen que las estrellas de neutrones pueden alcanzar masas superiores a $2 \masasol$, imponiendo restricciones sobre las ecuaciones de estado que deben ser suficientemente rígidas para soportar estas configuraciones. Otras estimaciones se han realizado empleando datos de ondas gravitacionales. Sin embargo, en mediciones mayores aún se discute la naturaleza del objeto observado (estrella de neutrones o agujero negro), como en el caso de GW190814 \cite{theligoscientificcollaborationGW190814GravitationalWaves2020, lopesNatureMassgapObject2022}, con una masa del objeto secundario estimado en $2.59^{+0.08}_{-0.09} \masasol$.

\subsection{Mediciones de Radio}

La misión NICER (Neutron star Interior Composition ExploreR) ha proporcionado algunas de las primeras mediciones simultáneas de masa y radio para estrellas de neutrones específicas. Estas mediciones se basan en el modelado de puntos calientes en la superficie de pulsares de milisegundo, analizando las modulaciones en el flujo de rayos X.

Los resultados más significativos incluyen, a 1$\sigma$ de confiabilidad:

\begin{itemize}
	\item \textbf{PSR J0740+6620}: Análisis independientes con NICER y X-ray Multi-Mirror han proporcionado valores $R = 13.7^{+2.6}_{-1.5}$ km \cite{millerRadiusPSRJ0740+66202021} y $R = 12.39^{+1.30}_{-0.98}$ km \cite{rileyNICERViewMassive2021};
	
	\item \textbf{PSR J0030+0451}: $M = 1.44^{+0.15}_{-0.14} \masasol$ y $R = 13.02^{+1.24}_{-1.06}$ km \cite{millerPSRJ0030+0451Mass2019};
	
	\item \textbf{PSR J0437-4715}: $M = 1.418 \pm 0.037 \masasol$ y $R = 11.36^{+0.95}_{-0.63}$ km \cite{choudhuryNICERViewNearest2024}.
\end{itemize}

Adicionalmente, se han realizado estimaciones de radio mediante la medición del parámetro de deformación de marea en ondas gravitacionales \cite{kumarTheoreticalExperimentalConstraints2024}. Sin embargo, para obtener estas estimaciones es necesario realizar modelamiento adicional, lo que hace esta estimación modelo-dependiente. Para el caso de \textit{GW190814} y asumiendo el objeto secundario como una estrella de neutrones de rotación rápida, se estimó un radio ecuatorial de $R_e = 14.1^{+1.5}_{-2.0}$ km, mientras que asumiendo una estrella no rotante, se estimó un radio canónico de $R_{1.4} = 13.3^{+0.5}_{-0.6}$ km, ambos en el intervalo de confianza a 90 \% \cite{biswasGW190814PropertiesSecondary2021}.\\


Las observaciones combinadas de masa y radio imponen restricciones consistentes sobre las posibles ecuaciones de estado de la materia nuclear densa. Estas restricciones pueden resumirse como:

\begin{itemize}
	\item \textbf{Rigidez Suficiente}: La ecuación de estado debe ser lo suficientemente rígida para soportar masas $\geq 2 \masasol$.
	
	\item \textbf{Radios Consistentes}: Los radios estelares deben estar en el rango $\sim 11-14$ km para masas típicas/canónicas de $\sim 1.4 \masasol$.
\end{itemize}

Estas restricciones observacionales permiten restringir las posibles ecuaciones de estado de la materia nuclear densa. Al resolver el sistema de ecuaciones de estructura (\ref{eq:masa} - \ref{eq:tov}) junto con una ecuación de estado específica (\ref{eq:ecuacion_estado}), es posible predecir la masa y radio de las estrellas de neutrones a partir de modelos microscópicos. De esta manera, las observaciones astronómicas establecen el puente entre la física microscópica de la materia nuclear densa y las características macroscópicas observables de estos objetos compactos, guiando la construcción y validación de modelos teóricos.

\begin{figure}[h]
	\centering
	\includegraphics[width=0.7\linewidth]{Figuras/ligo-virgo-graveyard}
	\caption{Masas en el cementerio estelar. Contiene las masas estimadas mediante diferentes fuentes y su (posible) naturaleza, a enero de 2024. Tomado de \href{https://www.ligo.caltech.edu/image/ligo20250826d}{ligo.caltech.edu}.}
	\label{fig:ligo-virgo-graveyard}
\end{figure}

Las próximas generaciones de detectores de ondas gravitacionales, junto con misiones espaciales mejoradas y telescopios de nueva generación, prometen expandir significativamente nuestro conocimiento observacional de las estrellas de neutrones. Se espera que estas observaciones proporcionen restricciones aún más precisas sobre las ecuaciones de estado y posiblemente revelen nueva física en el régimen de densidades ultra-altas. Es claro que, a mayor número de observaciones, mayores son los requerimientos de nuestras teorías, lo que permite filtrar modelos físicos que pretendan describir la masa a densidades tan altas.
En particular, la detección de estrellas de neutrones con masas en el rango $2.5-3 \masasol$ o la confirmación de transiciones de fase en el interior estelar a través de observaciones de estrellas, prometen descubrimientos de impacto para nuestra comprensión de la materia nuclear densa. En la figura \ref{fig:ligo-virgo-graveyard} se pueden apreciar las estimaciones de masa hasta enero de 2024, así como la posible naturaleza del objeto. Cabe destacar que en el rango de 2 a 5$\masasol$ hay incertidumbre en la naturaleza de varios objetos detectados.




%Capítulo 3
% Marco teórico - Microfísica
%\chapter{Microfísica: Ecuaciones de Estado}
%\thispagestyle{fancy}

% Introducción similar al capítulo sobre macrofísica. Aquí queremos mostrar brevemente la motivación de el estudio de ecuaciones de estado para entender la física de la materia en entornos tan densos y de tan alta energía.

%\section{Ecuaciones de Estado y Ejemplos}

%Definir las ecuaciones de estado (en este caso barótropas) y mostrar brevemente modelos y gráficas de modelos más simples: gas degenerado de neutrones, protones y electrones libres

%\section{Teoría Relativista de Campo Medio}

% Introducir la RMFT y mostrar las ventajas de una teoría relativista efectiva de campos mesónicos que se toman como uniformes mediante su valor medio en el estado base, frente a otros métodos como QCD y los Scrhodinger-based. Hablar sobre las simetrías y sus cantidades conservadas para explicar como se obtiene la expresión para la densidad de energía y de presión en una teoría de este estilo, construida sobre un lagrangeano.

%\section{Modelo del Estudio}

% Introducir el lagrangiano para un modelo que contenga neutrones, protones, electrones (campos esponoriales); un mesón escalar neutro sigma con autointeracciones de hasta 4to orden acoplado a la densidad escalar de nucleones, un mesón vectorial neutro omega acoplado a la corriente vectorial de nucleones y un mesón vectorial isovectorial neutro rho acoplado a la corriente de isospín de nucleones. Luego hallar las ecuaciones de movimiento para los campos, para finalmente llegar a la expresión de densidad de energía y presión. Hablar sobre los parámetros libres de este modelo.

%\subsection{Materia en Saturación Nuclear}

% Definir, explicar (importancia) y mostrar las mediciones de densidad de saturación, energía de enlace por nucleón, modulo de compresión, coeficiente de energía de simetría y pendiente del coeficiente de energía de simetría. Tomar las mismas mediciones de la propuesta.

\thispagestyle{fancy}
\chapter{Microfísica: Ecuaciones de Estado}
\label{chap:microfisica}

La descripción microscópica de la materia en entornos extremos de densidad es un problema de gran complejidad en la física moderna. En estas condiciones, con densidades que exceden significativamente la densidad nuclear de saturación, la materia exhibe comportamientos que requieren marcos teóricos que incorporen efectos relativistas y de muchos cuerpos, asegurando causalidad en el fluido y modelando las interacciones nucleares relevantes. Los modelos que describen esta materia establecen la conexión directa entre la física microscópica de las interacciones nucleares y las propiedades macroscópicas observables de las estrellas de neutrones \cite{oppenheimerMassiveNeutronCores1939}.

Diferentes teorías han sido propuestas para describir la materia en este régimen, desde enfoques no-relativistas basados en potenciales nucleares ajustados a datos experimentales \cite{myersNuclearPropertiesAccording1996, sakuragiSaturationNuclearMatter2016}, hasta teorías fundamentales como la cromodinámica cuántica (QCD) y sus extensiones efectivas \cite{drischlerGroundingNuclearPhysics2021}. Sin embargo, para obtener una extrapolación adecuada a densidades extremas, es necesario contar con un marco teórico que respete la causalidad y cuya complejidad computacional sea manejable. Es así como la teoría relativista de campo medio surge como una herramienta particularmente adecuada para abordar este régimen, brindando un tratamiento consistente que respeta la causalidad mientras incorpora las interacciones nucleares fuertes \cite{glendenningCompactStarsNuclear2000}. Este formalismo permite extrapolar desde las propiedades conocidas de materia nuclear simétrica hacia las condiciones asimétricas y de alta densidad relevantes para objetos compactos
\clearpage

\section{Ecuaciones de Estado y Ejemplos}

Una ecuación de estado define la relación termodinámica entre las variables que caracterizan el estado de equilibrio de un sistema físico. Para materia estelar a temperatura cero, consideramos ecuaciones de estado barotrópicas que relacionan la presión $P$ con la densidad de energía $\rho$. Esta relación contiene toda la información termodinámica necesaria para determinar la estructura de equilibrio hidrostático de estrellas de neutrones a través de las ecuaciones de Tolman-Oppenheimer-Volkoff (\sistemaTOV). La aproximación barotrópica es válida cuando los tiempos y escalas característicos de los procesos térmicos son despreciables comparados con las escalas hidrodinámicas y gravitacionales que determinan la estructura estelar. Se desprecia la temperatura debido a que la energía térmica y sus efectos son varios órdenes de magnitud inferiores a las energías internas de la materia en estrellas de neutrones \cite{shapiroBlackHolesWhite2008}.

\subsection{Ecuación Politrópica}

La ecuación de estado politrópica es uno de los modelos más sencillos para describir materia estelar, estableciendo una relación de ley de potencias entre la presión $P$ y la densidad de masa $\rho_m$:

\begin{equation}
	P = K (\rho_m)^{\gamma},
	\label{eq:eos_politropica}
\end{equation}

donde $K$ es una constante politrópica y $\gamma$ es el índice adiabático. Esta forma funcional, aunque fenomenológica, captura comportamientos asintóticos importantes de sistemas físicos más complejos y brinda soluciones analíticas o semi-analíticas para las ecuaciones de estructura estelar. El índice politrópico $n = 1/(\gamma - 1)$ determina las características de compresibilidad del material: valores bajos de $n$ corresponden a materia incompresible, mientras que valores altos describen sistemas altamente compresibles. Para materia ultra-relativista, el índice adiabátoco es $\gamma = 4/3$ ($n = 3$), mientras que para materia no-relativista degenerada, es $\gamma = 5/3$ ($n = 3/2$) \cite{chandrasekharIntroductionStudyStellar1970}. Este modelo de ecuación de estado es ampliamente utilizada en su versión politrópica a trozos (Piecewise Polytropic) para aproximar ecuaciones de estado más complejas mediante segmentos con diferentes índices politrópicos, lo que facilita su implementación en simulaciones numéricas \cite{becerraRealisticAnisotropicNeutron2024, raaijmakersConstraintsDenseMatter2021, chatziioannouNeutronStarsDense2024, choudhuryNICERViewNearest2024b}. En particular, es empleada para realizar estimaciones sobre el impacto de las mediciones astrofísicas de estrellas de neutrones en la ecuación de estado de la materia nuclear densa \cite{raaijmakersConstraintsDenseMatter2021}.

\subsection{Gas Ideal Degenerado}
\label{sec:gasnpe}

Para densidades suficientemente bajas tal que se pueda despreciar la interacción nuclear, un modelo más realista consiste en gases degenerados de fermiones, el modelo físico más simple frecuentemente empleado como referencia. Para materia nuclear compuesta por neutrones, protones y electrones, la ecuación de estado completa debe incluir las contribuciones de todas las especies presentes \cite{shapiroBlackHolesWhite2008}:

\begin{align}
	\rho &= \rho_n + \rho_p + \rho_e, \label{eq:densidad_total} \\
	P &= P_n + P_p + P_e, \label{eq:presion_total}
\end{align}

donde cada componente fermiónica contribuye según:

\begin{align}
	\rho_i &= \frac{g_i}{8\pi^2} \int_0^{p_{Fi}} p^2\sqrt{p^2 + m_i^2} \, dp, \label{eq:densidad_fermi_general} \\
	P_i &= \frac{g_i}{24\pi^2} \int_0^{p_{Fi}} \frac{p^4}{\sqrt{p^2 + m_i^2}} \, dp, \label{eq:presion_fermi_general}
\end{align}

con masas $m_n = 939.6$ MeV, $m_p = 938.3$ MeV, $m_e = 0.511$ MeV para los neutrones, protones y electrones respectivamente, y degeneraciones estadísticas (de espín) $g_i = 2$ para todas las especies. Los momentos de Fermi $p_{Fi}$ están determinados por las densidades de número mediante $n_i = \frac{g_i p_{Fi}^3}{6\pi^2}$. Para resolver el sistema, se imponen restricciones adicionales sobre la composición de la materia garantizando el equilibrio termodinámico: la neutralidad de carga eléctrica 

\begin{equation}
	n_p = n_e,
	\label{eq:neutralidad_carga}
\end{equation}

y el equilibrio beta débil 

\begin{equation}
	n \rightleftharpoons p + e^- + \bar{\nu}_e \implies \mu_n = \mu_p + \mu_e,
	\label{eq:equilibrio_beta}
\end{equation}

entre los potenciales químicos $\mu_i$, suponiendo que los neutrinos escapan del sistema sin alterar su energía. Estas restricciones permiten expresar todas las densidades en función de un parámetro libre como el momento de Fermi del electrón $p_{Fe}$, lo que reduce el sistema a una parametrización unidimensional.

La figura \ref{fig:eosnpe} muestra la ecuación de estado adimensionalizada para un gas ideal de neutrones, protones y electrones libres, obtenida tras resolver el sistema de integrales (\ref{eq:densidad_fermi_general}) y (\ref{eq:presion_fermi_general}) e interpolar, verificando que es causal ($c_s^2 < 1$). Adicionalmente, la figura \ref{fig:gas_npe_fraccion} muestra la fracción de neutrones respecto al número total de nucleones (bariones) $n_n/n_B = n_n/(n_n+n_p)$ como función de la densidad bariónica $n_B$. Se muestra la fracción de neutrones en el rango de densidades de 7.0$\times 10^{12}$ g/cm$^3$ a 1.6$\times 10^{16}$ g/cm$^3$. La fracción de neutrones en este rango de densidades inicia muy cercana al 100\%, disminuyendo lentamente a medida que la densidad aumenta, alcanzando aproximadamente un 94.5\% al llegar a $10^{16}$ g/cm$^3$, consistente con lo esperado para materia nuclear en equilibrio beta \cite{shapiroBlackHolesWhite2008}.

\begin{figure}[h]
	\centering
	\begin{subfigure}{0.42\linewidth}
		\centering
		\includegraphics[width=\linewidth]{Figuras/gas_npe}
		\caption{}
		\label{fig:eosnpe}
	\end{subfigure}
	\hfill
	\begin{subfigure}{0.57\linewidth}
		\centering
		\includegraphics[width=\linewidth]{Figuras/gas_npe_fraccion}
		\caption{}
		\label{fig:gas_npe_fraccion}
	\end{subfigure}
	\caption[Propiedades del gas ideal degenerado]{Propiedades del gas ideal degenerado de neutrones, protones y electrones. (a) Ecuación de estado normalizada. La ecuación de estado es causal ($c_s^2 = dP/d\rho < 1$). $\rho_0$ es el parámetro de adimensionalización y corresponde a la densidad de energía a $\rho_m=10^{18}$ g/cm$^3$. (b) Fracción de neutrones respecto al número de nucleones $n_n/n_B$ como función de la densidad bariónica $n_B$.}
	\label{fig:gas_npe_propiedades}
\end{figure}

Luego de obtener la ecuación de estado e integrando las ecuaciones TOV (\sistemaTOV) obtenemos las masas y radios de estrellas construidas con este material, como se muestra en la figura \ref{fig:mrnpe}. Este modelo sencillo predice una masa máxima de apenas 0.7$\, \masasol$, muy inferior a las masas que se han observado de estrellas de neutrones (listadas en la sección \ref{sec:obsNS}), motivando la búsqueda de modelos más realistas que incluyan las interacciones nucleares para lograr replicar las observaciones.

\begin{figure}[h]
	\centering
	\includegraphics[width=0.9\linewidth]{Figuras/gas_npe_MR}
	\caption[Relaciones masa-radio de gas ideal degenerado]{Relaciones de masa - radio (izquierda) y masa - densidad central de masa (derecha) de estrellas de neutrones constituidas por un gas ideal de neutrones, protones y electrones libres.}
	\label{fig:mrnpe}
\end{figure}

\section{Teoría Relativista de Campo Medio}
\label{sec:rmft}

La teoría relativista de campo medio, formulada por Walecka en 1974 \cite{waleckaTheoryHighlyCondensed1974}, es un marco teórico que describe las interacciones nucleares mediante el intercambio de mesones efectivos, extendiendo el modelo de gas degenerado (discutido en \ref{sec:gasnpe}) e incorporando naturalmente los efectos relativistas cuando los nucleones alcanzan velocidades comparables a la velocidad de la luz en condiciones de alta densidad. Esta aproximación ofrece ventajas significativas respecto a enfoques no-relativistas basados en potenciales fenomenológicos, ya que es una teoría covariante y reproduce simultáneamente las propiedades de saturación nuclear, el comportamiento asintótico a alta densidad, y la consistencia causal relativista. Adicionalmente, al ser una teoría efectiva, requiere de un menor esfuerzo computacional respecto a teorías más fundamentales como QCD.

El formalismo se construye a partir de un lagrangiano que describe nucleones interactuando a través de campos mesónicos: los campos fermiónicos representan los grados de libertad nucleónicos y leptónicos, mientras que los campos bosónicos representan mesones que median las interacciones fuertes. La aproximación de campo medio consiste en reemplazar los operadores de campo mesónicos por sus valores esperados en el estado base degenerado:

\begin{equation}
	\langle \phi_i(x^\mu) \rangle = \phi_i^0 \equiv \text{constante},
	\label{eq:campo_medio}
\end{equation}

donde $\phi_i$ denota los diferentes campos mesónicos del modelo. Esta aproximación es válida cuando las fluctuaciones cuánticas son pequeñas comparadas con los valores esperados de los campos, condición que se satisface para materia nuclear densa cuanto mayor es la densidad del sistema \cite{waleckaRelativisticNuclearManyBody1986}.

\subsection{Simetrías y Conservaciones}

El formalismo de la teoría relativista de campo medio se construye sobre la teoría cuántica de campos. La densidad lagrangiana $\mathcal{L}(\psi, \partial_\mu \psi, \phi_a, \partial_\mu \phi_a)$ describe las interacciones entre campos fermiónicos $\psi$ y campos bosónicos $\phi_a$ junto con sus términos libres. Esta densidad lagrangiana debe satisfacer los requerimientos de localidad, covariancia de Lorentz, y las simetrías internas relevantes para las interacciones nucleares fuertes \cite{glendenningCompactStarsNuclear2000}. La acción del sistema se define como la integral de la densidad lagrangiana sobre el volumen espaciotemporal:

\begin{equation}
	S[\psi, \phi_a] = \int d^4x \, \mathcal{L}(\psi, \partial_\mu \psi, \phi_a, \partial_\mu \phi_a),
	\label{eq:accion}
\end{equation}

donde $d^4x$ es el elemento de volumen en coordenadas de Minkowski. El principio de acción estacionaria establece que las configuraciones físicas de los campos corresponden a los extremos de este funcional, lo que conduce a las ecuaciones de Euler-Lagrange para todos los campos:

\begin{equation}
	\frac{\partial \mathcal{L}}{\partial \varphi_b} - \partial_\mu \left( \frac{\partial \mathcal{L}}{\partial (\partial_\mu \varphi_b)} \right) = 0,
	\label{eq:euler_lagrange}
\end{equation}

donde $\varphi_b$ representa todos los campos presentes. La teoría presenta simetrías externas e internas que determinan sus leyes de conservación y sus consecuencias físicas. Las simetrías externas son las transformaciones del grupo de Poincaré: traslaciones espaciotemporales $x^\mu \mapsto x'^\mu = x^\mu + a^\mu$ y boosts de Lorentz $x^\mu \mapsto x'^\mu = \Lambda^\mu{}_\nu x^\nu$. Las simetrías internas relevantes incluyen la simetría de gauge global $U(1)$ $\psi \mapsto e^{-i\lambda}\psi$ asociada con la conservación del número bariónico, y la simetría de isospín $SU(2)$ $\psi \mapsto e^{-i\boldsymbol{\tau}\cdot\boldsymbol{\lambda}}\psi$ asociada con la relación entre neutrones y protones, en la aproximación de iguales masas.

El teorema de Noether establece una correspondencia entre simetrías continuas del lagrangiano y cantidades conservadas. Para cada simetría continua existe una corriente conservada correspondiente que satisface una ecuación de continuidad. El tensor de energía-momento, asociado a la simetría externa, se define como:

\begin{equation}
	T^{\mu\nu} = \frac{\partial \mathcal{L}}{\partial (\partial_\mu \varphi_j)} \partial^\nu \varphi_j - \eta^{\mu\nu} \mathcal{L} \implies \partial_\mu T^{\mu\nu} = 0,
	\label{eq:tensor_energia_momento}
\end{equation}

garantizando que la energía total $\int d^3x T^{00}$ del sistema se conserve en el tiempo en ausencia de fuerzas externas (flujos de momento en la frontera).

La corriente bariónica, asociada a la simetría interna $U(1)$, se define como:

\begin{equation}
	J_B^\mu = \sum_N \bar{\psi}_N \gamma^\mu \psi_N \implies \partial_\mu J_B^\mu = 0,
	\label{eq:corriente_barionica}
\end{equation}

donde $\gamma^\mu$ son las matrices de Dirac y la suma se extiende sobre todas las especies de bariones presentes. Esto implica que el número bariónico $\int d^3x \, J_B^0$ integrado sobre el volumen total del sistema permanece constante en el tiempo, reflejando el hecho experimental de que los bariones no se crean ni se destruyen en interacciones fuertes como se ha evidenciado por ALICE en el LHC \cite{acharyaGlobalBaryonNumber2020} y se ha probado desde su postulación en 1940 \cite{gurrExperimentalTestBaryon1967}.

La corriente de isospín, asociada a la simetría interna $SU(2)$, se define como:

\begin{equation}
	\boldsymbol{J}_I^{\mu} = \sum_N \bar{\psi}_N \gamma^\mu \frac{\boldsymbol{\tau}}{2} \psi_N \implies \partial_\mu J_I^{\mu a} = 0,
	\label{eq:corriente_isospin}
\end{equation}

donde $\boldsymbol{\tau} = \tau^a$ ($a = 1, 2, 3$) son las matrices de Pauli en el espacio de isospín. La simetría de isospín ha sido ampliamente utilizada para describir las interacciones nucleares. En la realidad física esta simetría está ligeramente rota por las diferencias de masa entre neutrones y protones, y por las interacciones electromagnéticas. Sin embargo, dado que estas violaciones son pequeñas en la escala de interacciones fuertes, pueden tratarse como pequeñas correcciones e ignorarse en una primera aproximación \cite{baczykIsospinsymmetryBreakingMasses2018}. La conservación aproximada de isospín justifica el tratamiento unificado de neutrones y protones en modelos de materia nuclear.

% Estas leyes de conservación derivadas del teorema de Noether añaden restricciones importantes sobre la dinámica del sistema y establecen conexiones directas entre las simetrías fundamentales de la teoría y las cantidades físicamente observables.

%\vspace{-15pt}

\section{Modelo del Estudio}
\label{sec:modelo}

El modelo empleado en este trabajo incorpora nucleones (protones y neutrones) interactuando mediante campos mesónicos, y electrones libres. El lagrangiano total incluye términos para nucleones acoplados a los mesones, términos libres para electrones, un mesón escalar neutro $\sigma$ con autointeracciones no-lineales hasta cuarto orden, un mesón vectorial neutro $\omega^\mu$, y un mesón vectorial isovectorial $\boldsymbol{\rho^\mu}$ \cite{glendenningCompactStarsNuclear2000}:

\begin{equation}
	\begin{aligned}
		\mathcal{L} = &\bar{\psi} \left[ \gamma^\mu \left( i\partial_\mu - g_{\omega} \omega_\mu - \half g_{\rho} \boldsymbol{\tau} \cdot \rhomeson_\mu \right) - (m - g_{\sigma} \sigma) \right] \psi  \\
		&+ \frac{1}{2} \left( \partial_\mu \sigma \partial^\mu \sigma - m_\sigma^2 \sigma^2 \right) - \frac{1}{3}bm (g_\sigma\sigma)^3 - \frac{1}{4}c(g_\sigma\sigma)^4  \\
		&- \frac{1}{4} \omega_{\mu\nu} \omega^{\mu\nu} + \frac{1}{2} m_\omega^2 \omega_\mu \omega^\mu  \\
		&- \frac{1}{4} \rhomeson_{\mu\nu} \cdot \rhomeson^{\mu\nu} + \frac{1}{2} m_\rho^2 \rhomeson_\mu \cdot \rhomeson^\mu \\
		&- g_\rho \rhomeson_\mu \cdot [\rhomeson_\nu \cross \rhomeson^{\nu\mu} + 2g_\rho(\rhomeson^\mu \cross \rhomeson^\nu) \cross \rhomeson_\nu] \\
		&+ \bar{\psi}_e \left( i\gamma^\mu \partial_\mu - m_e \right) \psi_e,
		\label{eq:lagrangiano}
	\end{aligned}
\end{equation}

donde $\psi$ es una representación conveniente de los campos nucleónicos como un espinor de ocho componentes, $\psi_e$ es el campo del electrón, $\omega_{\mu\nu} = \partial_\mu \omega_\nu - \partial_\nu \omega_\mu$ y $\rhomeson_{\mu\nu} = \partial_\mu \rhomeson_\nu - \partial_\nu \rhomeson_\mu$ son los tensores antisimétricos de campo asociados a los mesones $\omega$ y $\rhomeson$ respectivamente, $m \approx 938.92 \text{MeV}$ es la masa de un nucleón considerada igual para protones y neutrones, y $m_i$ es la masa de la especie $i$. Los parámetros adimensionales de acoplamiento $g_{\sigma}$, $g_{\omega}$, $g_{\rho}$ cuantifican la intensidad de las interacciones nucleón-mesón, y los parámetros adimensionales $b$, $c$ regulan la autointeracción del mesón sigma. En la construcción de este lagrangiano se acopla el campo escalar con la densidad escalar, el campo vectorial con la corriente bariónica, y el campo isovectorial con la 3-corriente de isospín. Esta última corriente contiene no solo una contribución por los nucleones, sino también una contribución por la corriente propia del campo $\rhomeson$ y otra por la interacción del mesón con su propia 3-corriente.


Las ecuaciones de movimiento para los campos se obtienen mediante las ecuaciones de Euler-Lagrange (\ref{eq:euler_lagrange}). Para los campos mesónicos, se satisfacen:

\begin{gather}
	\left(\square + m_\sigma^2\right) \sigma = g_\sigma\left[\bar{\psi} \psi - bm(g_\sigma\sigma)^2 - c(g_\sigma\sigma)^3\right], \label{eq:eom_sigma_full} \\
	(\square + m_\omega^2) \omega^\mu - \partial^\mu \partial_\nu \omega^\nu = g_{\omega} \bar{\psi} \gamma^\mu \psi, \label{eq:eom_omega_full} \\
	\begin{aligned}
		(\square + m_\rho^2) \rhomeson^\mu - \partial^\mu \partial_\nu \rhomeson^\nu = \half g_{\rho} \bar{\psi} \gamma^\mu \boldsymbol{\tau} \psi - 2g_\rho&[\rhomeson_\nu \cross \rhomeson^{\nu\mu} + \partial_\nu(\rhomeson^\mu \cross \rhomeson^\nu) \\ 
		&+ 4g_\rho[(\rhomeson^\mu \cdot \rhomeson_\nu)\rhomeson^\nu - (\rhomeson_\nu \cdot \rhomeson^\nu)\rhomeson^\mu]]. \label{eq:eom_rho_full}
	\end{aligned}
\end{gather}

Para los campos fermiónicos se obtienen las ecuaciones de Dirac, acoplada a los campos mesónicos para los nucleones y libre para los electrones:

\begin{gather*}
	\left[ \gamma^\mu \left( i\partial_\mu - g_{\omega} \omega_\mu - \half g_{\rho} \boldsymbol{\tau} \cdot \rhomeson_\mu \right) - (m - g_{\sigma} \sigma) \right] \psi = 0, \\
	(i\gamma^\mu \partial_\mu - m_e) \psi_e = 0.
\end{gather*}

Hasta el momento tenemos un conjunto de ecuaciones de movimiento diferenciales, no lineales y acopladas que describen la dinámica completa del sistema. Aplicamos la aproximación de campo medio descrita en la sección \ref{sec:rmft}, considerando que tenemos materia estática y uniforme en su estado base, de modo que los valores esperados de los campos mesónicos son constantes en el espacio y el tiempo. En este caso, las derivadas espaciales y temporales de los campos mesónicos se anulan, simplificando las ecuaciones de movimiento a un sistema algebraico acoplado. Mantendremos las etiquetas para los campos, recordado que ahora son el valor esperado en el estado base. Para los mesones (\ref{eq:eom_sigma_full} - \ref{eq:eom_rho_full}), sustituyendo consistentemente las fuentes de corriente por su valor esperado, se obtiene:
% \begin{equation*}
% 	\begin{aligned}
% 		m_\sigma^2 \sigma &= g_\sigma \left[ \langle \bar{\psi} \psi \rangle - bm(g_\sigma\sigma)^2 - c(g_\sigma\sigma)^3\right], \\
% 		m_\omega^2 \omega^\mu &= g_{\omega} \langle \bar{\psi} \gamma^\mu \psi \rangle, \\
% 		m_\rho^2 \rhomeson^\mu &= \half g_{\rho} \langle \bar{\psi} \gamma^\mu \boldsymbol{\tau} \psi \rangle - 8g_\rho^2( (\rhomeson^\mu \cdot \rhomeson_\nu)\rhomeson^\nu - (\rhomeson_\nu \cdot \rhomeson^\nu)\rhomeson^\mu).
% 	\end{aligned}
% \end{equation*}

\begin{equation}
	\begin{aligned}
		m_\sigma^2 \sigma &= g_\sigma \left[\langle \bar{\psi}_p \psi_p \rangle + \langle \bar{\psi}_n \psi_n \rangle - bm(g_\sigma\sigma)^2 - c(g_\sigma\sigma)^3\right], \\
		m_\omega^2 \omega^\mu &= g_\omega \left[\langle \bar{\psi}_p \gamma^\mu \psi_p \rangle + \langle \bar{\psi}_n \gamma^\mu \psi_n \rangle \right], \\
		m_\rho^2 \rho_3^\mu &= g_\rho \left[\half \langle \bar{\psi}_p \gamma^\mu \psi_p \rangle - \half \langle \bar{\psi}_n \gamma^\mu \psi_n \rangle \right],
		\label{eq:eom_meson_mf}
	\end{aligned}
\end{equation}

donde $\psi_i$ representa el campo fermiónico para la especie $i = \{p, n\}$, y las primeras dos componentes del mesón $\rhomeson^\mu$ son escritas en términos de los operadores de creación y aniquilación para mesones rho cargados $\rho_\pm^\mu = \frac{1}{\sqrt{2}}(\rho_1^\mu\pm i\rho_2^\mu)$, luego su valor esperado se anula en el estado base del sistema \cite{glendenningCompactStarsNuclear2000}, anulando el término fuente de la corriente propia del campo.

Para los campos fermiónicos se tienen ecuaciones sin dependencia de las coordenadas de espacio-tiempo, siendo estados propios de momento:

\begin{equation}
	\begin{gathered}
		\left[ \gamma^\mu \left( p_\mu - g_{\omega} \omega_\mu - g_{\rho} I_3 \rho_{3\mu} \right) - (m - g_{\sigma} \sigma) \right] \psi(p^\nu) = 0, \\
		(\gamma^\mu p_\mu - m_e) \psi_e(p^\nu) = 0,
		\label{eq:eom_fermion_mf}
	\end{gathered}
\end{equation}

donde $I_3 = \{+\half \; \text{para protones}, -\half \; \text{para neutrones}\}$ es el isospín de la partícula. En analogía con la ecuación de Dirac libre, definimos las cantidades de 4-momento y masa efectivas para nucleones:
%\vspace{-8pt}
\begin{equation}
	\begin{gathered}
		P^\mu = p^\mu - g_{\omega} \omega^\mu - g_{\rho} I_3 \rho_{3}^\mu,\\
		m^* = m - g_\sigma \sigma,
	\end{gathered}	
\end{equation}

obteniendo entonces la relación de dispersión relativista:

\begin{equation}
	\left(P_\mu P^\mu - m^{*2}\right)\psi(P^\nu) = 0,
\end{equation}

luego los valores propios de energía para los nucleones son:

\begin{equation}
	\epsilon(\Vec{p})_{I_3} = \sqrt{(\Vec{p}-g_\omega \Vec{\omega}-g_\rho I_3\Vec{\rho}_{3})^2+(m-g_\sigma\sigma)^2} + g_\omega\omega_0 + g_\rho I_3\rho_{30}.
	\label{eq:particleenergy} 
\end{equation}

% Aunque todas las integrales anteriores son analíticas, la ecuación de movimiento para el campo escalar (\ref{eq:eom_meson_mf}) es una ecuación no lineal en $\sigma$ que debe resolverse autoconsistentemente con métodos numéricos. Además, resolviendo el sistema con $g_\sigma \sigma$, $g_\omega \omega_0$ y $g_\rho \rho_{30}$ como variables, el modelo tiene como parámetros libres unicamente los cocientes $g_\sigma/m_\sigma$, $g_\omega/m_\omega$ y $g_\rho/m_\rho$, junto con los parámetros de autointeracción escalar.

Ahora, los valores esperados de los operadores en (\ref{eq:eom_meson_mf}) se calculan con el método descrito en el Apéndice \ref{apendice:valores_esperados}. Luego, las ecuaciones de movimiento quedan de la siguiente forma:

\begin{equation}
	\begin{aligned}
		m_\sigma^2 \sigma &= g_\sigma \sum_N \frac{1}{\pi^2}\int_0^{p_{FN}} \frac{p^2 (m-g_\sigma\sigma) dp}{\sqrt{p^2 + (m-g_\sigma\sigma)^{2}}} - g_\sigma bm(g_\sigma\sigma)^2 - g_\sigma c(g_\sigma\sigma)^3, \\
		m_\omega^2 \omega_0 &= \frac{1}{3\pi^2} g_\omega(p_{Fp}^3 + p_{Fn}^3), \\
		m_\rho^2 \rho_{30} &= \frac{1}{6\pi^2} g_\rho(p_{Fp}^3 - p_{Fn}^3),
	\end{aligned}
	\label{eq:eom_meson_mf_sustituidas}
\end{equation}

donde $N$ indica la suma sobre protones y neutrones. Observamos que las componentes espaciales de los campos vectoriales se anulan debido a la isotropía del sistema en su estado base. 

Para hallar la ecuación de estado, calculamos el tensor de energía-momento canónico (\ref{eq:tensor_energia_momento}) en el marco de referencia del fluido en reposo e isotrópico, utilizando el lagrangiano (\ref{eq:lagrangiano}). Luego, hallamos el valor esperado de sus componentes empleando el mismo método del Apéndice \ref{apendice:valores_esperados} y las identificamos con las cantidades termodinámicas de densidad de energía y presión para un fluido perfecto (\ref{eq:tensor_fluido_perfecto}). Las expresiones para densidad de energía y presión son:

\begin{gather}
	\begin{aligned}
		\rho = \langle T^{00} \rangle &= \frac{1}{2}m_\sigma^2\sigma^2 + \frac{1}{3}bm(g_\sigma\sigma)^3 + \frac{1}{4}c(g_\sigma\sigma)^4 + \frac{1}{2}m_\omega^2\omega_0^2 + \frac{1}{2}m_\rho^2\rho_{30}^2 \\
		&+ \sum_N \frac{1}{\pi^2}\int_0^{p_{FN}} p^2 dp \sqrt{p^2 + (m-g_\sigma\sigma)^2} + \frac{1}{\pi^2}\int_0^{p_{Fe}} p^2 dp \sqrt{p^2 + m_e^2},
		\label{eq:densidad_energia}
	\end{aligned} \\
	\begin{aligned}
		P = \frac{1}{3}\sum_i \langle T^{ii} \rangle &= - \frac{1}{2}m_\sigma^2\sigma^2 - \frac{1}{3}bm(g_\sigma\sigma)^3 - \frac{1}{4}c(g_\sigma\sigma)^4 + \frac{1}{2}m_\omega^2\omega_0^2 + \frac{1}{2}m_\rho^2\rho_{30}^2 \\
		&+ \sum_N \frac{1}{3\pi^2}\int_0^{p_{FN}} \frac{p^4 dp}{\sqrt{p^2 + (m-g_\sigma\sigma)^2}} + \frac{1}{3\pi^2}\int_0^{p_{Fe}} \frac{p^4 dp}{\sqrt{p^2 + m_e^2}}.
		\label{eq:presion}
	\end{aligned}
\end{gather}

Ahora, con el fin de resolver numéricamente el sistema, nos interesa reescribir las ecuaciones necesarias en términos de variables adimensionales. Definimos entonces las siguientes cantidades:

\begin{equation}
	\begin{aligned}
		&x = \frac{p}{m}, & &\xsigma = 1 - \frac{g_\sigma \sigma}{m} = \frac{m^*}{m}, \\
		&A_i = \left(\frac{g_i}{m_i}\right)^2 m^2, & &\tilde{n} = \frac{n_B}{m^3} = \frac{1}{3\pi^2}(x_{Fp}^3+x_{Fn}^3), \\
		& \tilde{\rho} = \frac{2\rho}{m^4}, & & \tilde{P} = \frac{2P}{m^4},
	\end{aligned}
	\label{eq:variables_adimensionales}
\end{equation}

donde $i = \{\sigma, \omega, \rho\}$. Usando estas variables, la ecuación de movimiento para el campo escalar (\ref{eq:eom_meson_mf_sustituidas}) se escribe como:

\begin{equation}
	(1 - \xsigma) - A_\sigma \left[ \frac{1}{\pi} \sum_N \int_0^{x_{FN}} \frac{\xsigma x^2 dx}{\sqrt{x^2 + \xsigma^2}} - b(1-\xsigma)^2 - c(1-\xsigma)^3 \right] = 0,
	\label{eq:eom_sigma_adimensional}
\end{equation}

donde $x_{Fi} = p_{Fi}/m$ son los momentos de Fermi adimensionales. Las expresiones para densidad de energía (\ref{eq:densidad_energia}) y presión (\ref{eq:presion}), utilizando las expresiones para la densidad bariónica (\ref{eq:densidad_barionica}) y la densidad de 3-isospín (\ref{eq:densidad_isospin}), quedan:

\begin{gather}
	\begin{aligned}
		\tilde{\rho} = &\frac{1}{A_\sigma}(1-\xsigma)^2 + \frac{2}{3}b(1-\xsigma)^3 + \frac{1}{2}c(1-\xsigma)^4 + A_\omega \tilde{n}^2 + \frac{1}{36 \pi^4}A_\rho(x_{Fp}^3-x_{Fn}^3)^2 \\
		&+ \sum_N \frac{2}{\pi^2}\int_0^{x_{FN}} \sqrt{x^2 + \xsigma^2} x^2 dx + \frac{2}{\pi^2}\int_0^{x_{Fe}} \sqrt{x^2 + \left(\frac{m_e}{m}\right)^2} x^2 dx,
		\label{eq:densidad_energia_adimensional}
	\end{aligned} \\
	\begin{aligned}
		\tilde{P} = &- \frac{1}{A_\sigma}(1-\xsigma)^2 - \frac{2}{3}b(1-\xsigma)^3 - \frac{1}{2}c(1-\xsigma)^4 + A_\omega \tilde{n}^2 + \frac{1}{36 \pi^4}A_\rho(x_{Fp}^3-x_{Fn}^3)^2 \\
		&+ \sum_N \frac{2}{3\pi^2}\int_0^{x_{FN}} \frac{x^4 dx}{\sqrt{x^2 + \xsigma^2}} + \frac{2}{3\pi^2}\int_0^{x_{Fe}} \frac{x^4 dx}{\sqrt{x^2 + \left(\frac{m_e}{m}\right)^2}}.
		\label{eq:presion_adimensional}
	\end{aligned}
\end{gather}

Estas expresiones son funciones de la densidad de número de protones, neutrones y electrones. Si queremos describir la materia en el interior de estrellas de neutrones, debemos considerar ligaduras que nos permitirán cerrar el sistema de ecuaciones. Es necesario imponer las mismas condiciones de neutralidad local de carga (\ref{eq:neutralidad_carga}) y equilibrio beta (\ref{eq:equilibrio_beta}) discutidas para el gas ideal degenerado en la sección \ref{sec:gasnpe}, así como la conservación del número de bariones $n_B = n_p+n_n$, obteniendo:

\begin{equation}
	\begin{gathered}
		x_{Fp} = x_{Fe}, \\
		\sqrt{x_{Fn}^2+\xsigma^2} - \sqrt{x_{Fp}^2+\xsigma^2} - \frac{1}{6\pi^2}A_\rho - \sqrt{x_{Fe}^2+\left(\frac{m_e}{m}\right)^2} = 0,\\
		x_{Fp} = (3\pi^2\tilde{n} - x_{Fn}^3)^{1/3}.
	\end{gathered}
	\label{eq:ligaduras_adimensionales}
\end{equation}

De las expresiones adimensionalizadas (\ref{eq:eom_sigma_adimensional} - \ref{eq:ligaduras_adimensionales}) podemos entender el sistema físico descrito por el modelo. Primero, es importante notar que, si bien definimos ocho constantes inicialmente ($g_i$, $m_i$ con $i=\{\sigma, \omega, \rho\}$, $b$, $c$), las ecuaciones solo dependen de cinco combinaciones adimensionales de estas constantes: los cocientes $A_i$, y los parámetros de autointeracción escalar $b$ y $c$. Por lo tanto, el modelo tiene cinco parámetros libres que deben ser ajustados para reproducir datos experimentales o teóricos adicionales. En segundo lugar, podemos deducir de las integrales en las ecuaciones para el campo escalar, la densidad de energía y la presión que el campo escalar $\sigma$ actúa para reducir la masa efectiva de los nucleones disminuyendo la energía por partícula a mayores densidades, mientras que el campo vectorial $\omega_0$ aumenta la energía por partícula debido a su acoplamiento con la densidad bariónica total. Esto indica que el mesón $\sigma$ genera una interacción atractiva entre nucleones, mientras que el mesón $\omega$ genera una interacción repulsiva. Finalmente, el mesón $\rho$ actúa para ajustar la diferencia entre las densidades de protones y neutrones con un efecto ``repulsivo'' ante la asimetría del sistema.

\begin{figure}[h]
	\centering
	\includegraphics[width=\linewidth]{Figuras/materia_estelar_base}
	\caption[Ecuación de estado para el núcleo de estrellas de neutrones.]{Ecuación de estado (izquierda) y variables termodinámicas en función de la densidad de masa (derecha) para el núcleo de estrellas de neutrones empleando teoría relativista de campo medio. $\rho_0=m^4/2$ es el factor de adimensionalización. Se usaron los parámetros $\left(\frac{g_\sigma}{m_\sigma}\right)^2=12.684\,\text{fm}^2$, $\left(\frac{g_\omega}{m_\omega}\right)^2=7.148\,\text{fm}^2$, $\left(\frac{g_\rho}{m_\rho}\right)^2=4.410\,\text{fm}^2$, $b=5.610\times10^{-3}$ y $c=-6.986\times10^{-3}$ como referencia.}
	\label{fig:materiaestelarbase}
\end{figure}


Dados los parámetros libres del modelo, tenemos un sistema cerrado que podemos resolver numéricamente para cada valor de $n_B$, obteniendo la ecuación de estado $\tilde{\rho}(\tilde{P})$, como se muestra en la figura \ref{fig:materiaestelarbase}. Sin embargo, aún es necesario restringir los parámetros que hacen al modelo físicamente consistente. Para ello, acudimos a las propiedades empíricas de la materia nuclear en saturación, descritas a continuación.


\subsection{Materia en Saturación Nuclear}
\label{sec:saturacion}

Si suprimimos los electrones libres del modelo (\ref{eq:lagrangiano}) y consideramos materia nuclear simétrica ($n_p = n_n$), el sistema se reduce a un fluido de nucleones interactuando mediante los campos mesónicos. Este sistema debe reproducir las propiedades de la materia nuclear en saturación, caracterizada por parámetros empíricos que añaden restricciones obligatorias para cualquier teoría microscópica válida \cite{kumarTheoreticalExperimentalConstraints2024}. De estos parámetros, consideraremos cinco que pueden determinarse experimentalmente mediante mediciones en laboratorios terrestres, siendo una herramienta de calibración para los modelos teóricos nucleares.

La densidad de saturación nuclear, $n_0$, define la densidad a la cual la materia nuclear simétrica alcanza su estado de mínima energía de enlace por nucleón, $\frac{B}{A}$. Tras aplicar un análisis bayesiano a una colección de ligaduras de la Teoría Funcional de la Densidad, que incluyen modelos relativistas de campo medio y de Skyrme, se obtienen valores \cite{drischlerBayesianMixtureModel2024}:

\begin{align}
	&n_0 = 0.157 \pm 0.010 \, \text{fm}^{-3} \label{eq:densidad_saturacion}, \\
	&\frac{B}{A} = \frac{\rho(n_0)}{n_0} - m = -15.97 \pm 0.40 \, \text{MeV}. \label{eq:energia_enlace_saturacion}
\end{align}

Este resultado empírico es consistente con estimaciones realizadas a partir de mediciones de dispersión de electrones con violación de paridad en $^{208}$Pb en el experimento PREX \cite{horowitzInsightsNuclearSaturation2020} y ajustes de 1654 núcleos atómicos a un modelo de gota líquida \cite{kumarTheoreticalExperimentalConstraints2024, myersNuclearPropertiesAccording1996}. En nuestro modelo (\ref{eq:lagrangiano}), estas propiedades de saturación se determinan hallando el mínimo de la función $\tfrac{B}{A}(n_B)$ en ausencia de electrones, como se muestra de ejemplo en la figura \ref{fig:saturacionbase}.

\begin{figure}[h]
	\centering
	\includegraphics[width=0.7\linewidth]{Figuras/saturacion_base}
	\caption[Energía de enlace por nucleón en saturación nuclear]{Energía de enlace por nucleón en función de la densidad bariónica para materia nuclear simétrica sin electrones. La densidad de saturación $n_0$ y la energía de enlace por nucleón en saturación $\frac{B}{A}$ se señalan en el mínimo. Se usaron los parámetros $\left(\frac{g_\sigma}{m_\sigma}\right)^2=12.684\,\text{fm}^2$, $\left(\frac{g_\omega}{m_\omega}\right)^2=7.148\,\text{fm}^2$, $\left(\frac{g_\rho}{m_\rho}\right)^2=4.410\,\text{fm}^2$, $b=5.610\times10^{-3}$, $c=-6.986\times10^{-3}$.}
	\label{fig:saturacionbase}
\end{figure}

El módulo de compresibilidad nuclear, $K_0$, caracteriza la rigidez de la materia nuclear ante compresiones alrededor del punto de saturación. Este parámetro es determinante para la extrapolación de la ecuación de estado a densidades supranucleares, y es estimado mediante las resonancias monopolares en núcleos pesados. Considerando un modelo de cadena de núcleos para esta resonancia en $^{208}$Pb, se obtiene \cite{kumarTheoreticalExperimentalConstraints2024, khanConstrainingNuclearEquation2012}:

\begin{equation}
	K_0 = 9n_0^2 \frac{\partial^2}{\partial n^2}\left(\frac{\rho}{n}\right)\bigg|_{n=n_0} = 230 \pm 40 \, \text{MeV}.
	\label{eq:modulo_compresibilidad_empirico}
\end{equation}

Esta cantidad está relacionada algebraicamente con cantidades de nuestro modelo de materia simétrica en saturación mediante:

\begin{equation}
	\begin{gathered}
		\frac{K_0}{3m} = \frac{2}{\pi^2}A_\omega x_F^3 + \frac{x_F^2}{\sqrt{x_F^2 + \xsigma^2}} - \frac{2A_\sigma}{\pi^2}\frac{x_F^3\xsigma^2}{x_F^2+\xsigma^2}F^{-1}, \\
		F = 1 + A_\sigma (1-\xsigma)[2b+3c(1-\xsigma)]+\frac{2A_\sigma}{\pi^2}\int_0^{x_F} \frac{x^4 dx}{(x^2+\xsigma^2)^{3/2}},
	\end{gathered}
	\label{eq:modulo_compresibilidad}
\end{equation}

donde $x_Fp=x_Fn=x_F$.

La energía de simetría, $a_\text{sym}$, cuantifica el costo energético de desviarse de la composición simétrica. Esta cantidad determina la composición de protones y neutrones en materia nuclear densa, afectando directamente las propiedades de las estrellas de neutrones. La pendiente de la energía de simetría, $L_0$, describe la dependencia de la energía de simetría con la densidad de bariones alrededor de la densidad de saturación.

Para estas dos cantidades, tras una revisión de 28 estimaciones tanto de experimentos terrestres como de observaciones astronómicas de estrellas de neutrones, se estiman valores representativos \cite{kumarTheoreticalExperimentalConstraints2024, liUnderstandingAstrophysicalEffects2019}:

\begin{align}
	&a_\text{sym} = \frac{1}{2} \frac{\partial^2}{\partial t^2}\left(\frac{E}{A}\right)\bigg|_{t=0} = 31.6 \pm 2.7 \, \text{MeV}, \label{eq:energia_simetria_empirico} \\
	&L_0 = 3n_0 \frac{\partial a_\text{sym}}{\partial n}\bigg|_{n=n_0} = 58.9 \pm 16 \, \text{MeV}. \label{eq:pendiente_simetria_empirico},
\end{align}

donde $t = (n_n - n_p)/n_B$ es el parámetro de asimetría de isospín. Es de notar que, si bien el valor fiduciario tiene una baja incertidumbre, las estimaciones individuales de $L_0$ varían ampliamente con intervalos de confianza que se estiman entre 20 y 120 MeV debido a la dificultad para acceder a materia altamente asimétrica en el laboratorio \cite{esteeProbingSymmetryEnergy2021}. En nuestro modelo, estas cantidades pueden calcularse como:

\begin{align}
	\frac{a_\text{sym}}{m} &= \frac{x_F^2}{6\sqrt{x_F^2 + \xsigma^2}} + \frac{1}{12\pi^2}A_\rho x_F^3, \label{eq:energia_simetria} \\
	\frac{L_0}{m} &= \frac{x_F^2}{6\sqrt{x_F^2 + \xsigma^2}}\left(1 + \frac{\xsigma^2}{x_F^2 + \xsigma^2}\right) + \frac{1}{4\pi^2}A_\rho x_F^3. \label{eq:pendiente_simetria}
\end{align}

Estas cinco propiedades empíricas ($n_0$, $\frac{B}{A}$, $K_0$, $a_\text{sym}$, $L_0$) definen restricciones que cualquier ecuación de estado microscópica debe satisfacer para ser físicamente viable. En el contexto de la teoría relativista de campo medio, estas propiedades y sus incertidumbres se utilizan para determinar las regiones en el espacio de parámetros que producen ecuaciones de estado físicamente aceptables frente a mediciones nucleares experimentales.

\section{Corteza de la Estrella}
\label{sec:corteza}

El modelo de teoría relativista de campo medio desarrollado es una descripción de la materia nuclear en el régimen de alta densidad, específicamente para densidades bariónicas superiores a aproximadamente $0.1$ fm$^{-3}$ o $\rho_m \gtrsim 1.7 \times 10^{14}$ g/cm$^3$, donde las interacciones nucleón-nucleón dominan la dinámica del sistema y la aproximación de campo medio tiene sentido. Sin embargo, las estrellas de neutrones presentan una estructura estratificada que incluye regiones de menor densidad donde esta aproximación deja de ser aplicable. La corteza de la estrella, que se extiende desde la superficie hasta el núcleo denso, abarca un rango de densidades de varios órdenes de magnitud y requiere tratamientos teóricos distintos según el régimen de densidad considerado \cite{shapiroBlackHolesWhite2008}.

El alcance del presente estudio se concentra en la descripción microscópica de la materia nuclear en el régimen de altas energías en las estrellas de neutrones, donde las densidades superan varias veces la densidad de saturación $n_0$. Para la descripción de la corteza utilizamos las ecuaciones de estado estándar BPS y BBP. Para densidades inferiores al punto de goteo de neutrones ($n_{\text{drip}} \approx 2.4\times10^{-4}$ fm$^{-3}$ o $\rho_m \approx 4 \times 10^{11}$ g/cm$^3$), la materia se compone de una red cristalina de núcleos pesados embebidos en un gas degenerado de electrones, configuración conocida como corteza externa. En este régimen, la ecuación de estado BPS (Baym, Pethick y Sutherland, 1971) \cite{baymGroundStateMatter1971} es una descripción consistentemente basada en la minimización de la energía del estado base de núcleos atómicos para cada densidad, junto con la contribución del gas de electrones relativista. El modelo BPS determina la composición nuclear óptima mediante la competencia entre la energía de masa nuclear, la energía de Coulomb de la red, y la presión del gas electrónico. Por encima de la densidad de goteo de neutrones y hasta densidades del orden de $0.2$ fm$^{-3}$ o $\rho_m \approx 3.2 \times 10^{14}$ g/cm$^3$ se encuentra la corteza interna, donde los núcleos coexisten con un gas de neutrones libres además del gas de electrones. En este régimen, la ecuación de estado BBP (Baym, Bethe y Pethick, 1971) \cite{baymNeutronStarMatter1971} describe la materia mediante un modelo de gota líquida que considera las contribuciones energéticas de los núcleos, el gas de neutrones libres, y las interacciones nucleares efectivas de la red. Este modelo determina autoconsistentemente la composición nuclear y la fracción de neutrones libres mediante la minimización de la energía total del sistema, que incluye términos de energía de superficie, energía de Coulomb, y energía de simetría nuclear.

Para construir una ecuación de estado unificada que abarque todo el rango de densidades presente en la estrella, desde la superficie hasta el núcleo, empleamos el método de interpolación \href{https://es.wikipedia.org/wiki/Interpolador_c\%C3\%BAbico_de_Hermite}{PCHIP} (Piecewise Cubic Hermite Interpolating Polynomial), que garantiza una transición suave y monótona entre las diferentes ecuaciones de estado. Este método de interpolación preserva la forma de los datos y evita oscilaciones no físicas que podrían introducir otros métodos de interpolación polinómica. Definimos la densidad de empalme entre la ecuación de estado BBP y nuestro modelo de teoría relativista de campo medio entre $n_B = 0.043$ fm$^{-3}$ y $n_B = 0.063$ fm$^{-3}$, correspondiente a una densidad de masa de entre $7.20 \times 10^{13}$ g/cm$^3$ y $1.05 \times 10^{14}$ g/cm$^3$. Esta elección se fundamenta en la necesidad de garantizar la causalidad del fluido en toda la estrella, es decir, que la velocidad del sonido $c^2_s = dP/d\rho$ no exceda la velocidad de la luz $c$ en ningún punto. El resultado de esta interpolación para un conjunto de parámetros de ejemplo y su comprobación de causalidad se muestra en la figura \ref{fig:causalidadmateriabase}.

\begin{figure}[h]
	\centering
	\includegraphics[width=0.7\linewidth]{Figuras/causalidad_materia_estelar_completa}
	\caption[Ecuación de estado unificada y causalidad.]{Ecuación de estado unificada para la estrella de neutrones, construida mediante interpolación PCHIP entre las ecuaciones de estado BPS, BBP y el modelo TRCM. Las X marcan los límites de la región de empalme. El panel inferior muestra el cuadrado de la velocidad del sonido $c_s^2$ para verificar la condición de causalidad. Se usaron los parámetros $\left(\frac{g_\sigma}{m_\sigma}\right)^2=12.684\,\text{fm}^2$, $\left(\frac{g_\omega}{m_\omega}\right)^2=7.148\,\text{fm}^2$, $\left(\frac{g_\rho}{m_\rho}\right)^2=4.410\,\text{fm}^2$, $b=5.610\times10^{-3}$, $c=-6.986\times10^{-3}$.}
	\label{fig:causalidadmateriabase}
\end{figure}

% Figura: Interpolación PCHIP
% [Espacio para figura mostrando P vs rho para BPS, BBP, interpolación PCHIP, y modelo RMFT, 
% con énfasis en la región de empalme en n_B = 0.062 fm^{-3}. Incluir panel secundario 
% mostrando la velocidad del sonido v_s/c para verificar causalidad (v_s < c) en toda la región.]


%Capítulo 4
\chapter{Resultados y discusión}
% Resultados

Este capítulo se estructura en dos partes complementarias. Primero, se realiza un análisis del comportamiento general del modelo, examinando las soluciones autoconsistentes para el campo escalar (\ref{eq:eom_sigma_adimensional}) y los momentos de Fermi del neutrón y protón (\ref{eq:ligaduras_adimensionales}), la descomposición de la energía de enlace en sus contribuciones mesónicas y las condiciones de aceptabilidad física, basadas en los criterios recopilados por Hernández et al. \cite{hernandezAcceptabilityConditionsRelativistic2021}.

La segunda parte aborda la caracterización sistemática del espacio de parámetros. El objetivo central es identificar los conjuntos de parámetros que satisfacen simultáneamente las restricciones nucleares y las observaciones astrofísicas actuales. Este estudio establece relaciones directas entre los parámetros microscópicos de la materia (\ref{eq:lagrangiano}) y las propiedades macroscópicas observables, permitiendo comprender cómo las interacciones nucleares determinan la estructura de las estrellas de neutrones. En particular, se busca determinar el conjunto de parámetros que maximiza la masa estelar predicha, estableciendo así límites superiores teóricos que puedan ser confrontados con las observaciones de los objetos más masivos detectados hasta la fecha.

\clearpage
\section{Sobre el Modelo Nuclear}

Antes de explorar el espacio de parámetros, es importante analizar el comportamiento de las variables dinámicas del modelo y cómo estas determinan las propiedades de la materia nuclear. A continuación, se discuten algunas características relevantes.

\subsection{Soluciones de Autoconsistencia}

La ecuación del movimiento para el campo escalar en su forma adimensional (\ref{eq:eom_sigma_adimensional}) junto con la condición de equilibrio beta y neutralidad de carga (\ref{eq:ligaduras_adimensionales}) forman un sistema de ecuaciones acopladas no lineales que deben resolverse de manera autoconsistente para cada valor de la densidad bariónica $n_B$. Para un conjunto de parámetros específico, se muestra en la figura \ref{fig:soluciones_autoconsistentes} la evolución de la masa efectiva adimensional $\xsigma = (1 - g_\sigma \sigma / m) = m^*/m$ y los momentos de Fermi adimensionales del neutrón $x_{Fn}$ y el protón $x_{Fp}$ en función de la densidad bariónica $n_B$. Las soluciones se muestran para densidades bariónicas desde $n_B = 0.001\,\text{fm}^{-3}$ ($\approx 1.7\times 10^{12} \, \text{g cm}^{-3}$) hasta $n_B = 100\,\text{fm}^{-3}$ ($\approx 1.7\times 10^{17} \, \text{g cm}^{-3}$), mostrando el comportamiento de estas variables en el rango relevante para la materia nuclear y las estrellas de neutrones.

\begin{figure}[h]
    \centering
    \includegraphics[width=0.95\linewidth]{Figuras/resultado_autoconsistencia}
    \caption[Soluciones autoconsistentes del modelo relativista de campo medio]{Soluciones autoconsistentes para la masa efectiva adimensional $\xsigma$ (izquierda) y los momentos de Fermi adimensionales del neutrón $x_{Fn}$ y el protón $x_{Fp}$ (derecha) en función de la densidad bariónica $n_B$. Se usaron los parámetros $\left(\frac{g_\sigma}{m_\sigma}\right)^2=12.684\,\text{fm}^2$, $\left(\frac{g_\omega}{m_\omega}\right)^2=7.148\,\text{fm}^2$, $\left(\frac{g_\rho}{m_\rho}\right)^2=4.410\,\text{fm}^2$, $b=5.610\times10^{-3}$, $c=-6.986\times10^{-3}$.}
    \label{fig:soluciones_autoconsistentes}
\end{figure}

Notamos de la figura \ref{fig:soluciones_autoconsistentes} que $\xsigma$ disminuye con la densidad, disminuyendo la masa efectiva de los nucleones en el medio. Además, el valor de $\xsigma$ parece estar acotado tanto superior como inferiormente, donde a densidades cada vez menores ($n_B << n_0 = 0.157\,\text{fm}^{-3}$) se aproxima a 1, es decir, la masa efectiva tiende a la masa del nucleón en el vacío $m^* \to m$, y por otro lado, a densidades cada vez mayores ($n_B >> n_0$) la masa efectiva tiende a cero $m^* \to 0$. Esta disminución de la masa efectiva con la densidad implica que el mesón escalar $\sigma$ disminuye la energía $e(p) = \sqrt{p^2 + m^{*2}}$ de los nucleones en el medio cuando los nucleones se acercan entre sí, reflejando la naturaleza atractiva de esta interacción. En cuanto a los momentos de fermi, se observa que ambos aumentan con la densidad bariónica, siendo $x_{Fn}$ siempre mayor a $x_{Fp}$ debido a la condición de equilibrio beta y neutralidad de carga. A medida que la densidad aumenta, la diferencia entre ambos momentos de Fermi disminuye, indicando que la materia se hace más simétrica en isospin a altas densidades, aunque siempre con exceso de neutrones. Comparando los comportamientos de $\xsigma$, $x_{Fn}$ y $x_{Fp}$, se observa que la disminución de la masa efectiva está correlacionada con el aumento de los momentos de Fermi, ya que una masa efectiva menor implica una mayor energía cinética para los nucleones a un mismo nivel de Fermi, favoreciendo la ocupación de estados de mayor momento.

\subsection{Descomposición de la Energía de Enlace}

La energía de enlace por nucleón $E/A$ surge de la competencia entre la energía cinética de los nucleones y las contribuciones de los campos mesónicos. Es posible descomponer esta energía en términos de los potenciales asociados a los mesones $\sigma$ (atractivo), $\omega$ (repulsivo) y $\rho$ (isovectorial). A bajas densidades, la contribución atractiva del campo $\sigma$ domina, permitiendo la formación de estados ligados y la saturación de la materia nuclear. A medida que aumenta la densidad, el término repulsivo del campo $\omega$ crece rápidamente, dominando la energética y proporcionando la rigidez necesaria para soportar estrellas de neutrones masivas contra el colapso gravitacional. El campo $\rho$, por otro lado, contribuye principalmente a la energía de simetría, diferenciando la energía de la materia de neutrones pura respecto a la materia simétrica isospin.

\subsection{Condiciones de Aceptabilidad Física}

Para que una ecuación de estado sea físicamente admisible, debe satisfacer ciertos criterios teóricos fundamentales en todo el rango de densidades relevante para las estrellas de neutrones:

\begin{itemize}
    \item \textbf{Estabilidad Microscópica:} La presión debe ser una función creciente de la densidad de energía, es decir, $dP/d\mathcal{E} > 0$. Esto asegura la estabilidad mecánica y termodinámica de la materia frente a fluctuaciones locales.
    \item \textbf{Causalidad:} La velocidad del sonido en el medio, definida como $v_s = \sqrt{dP/d\mathcal{E}}$, no debe exceder la velocidad de la luz en el vacío ($v_s \le 1$ en unidades naturales). La violación de esta condición implicaría la propagación de señales superlumínicas, lo cual es incompatible con la relatividad especial.
\end{itemize}

Todas las ecuaciones de estado consideradas en este trabajo son verificadas para cumplir estrictamente con las condiciones de estabilidad y causalidad hasta las densidades centrales máximas alcanzadas en las estrellas de neutrones, asegurando la consistencia física de los resultados presentados.

\section{Estudio del Espacio de Parámetros}

Como se estableció en el capítulo anterior (sección \ref{sec:modelo}), el modelo considerado para la ecuación de estado en el marco de la teoría relativista de campo medio contiene cinco parámetros libres: los tres acoplamientos mesón-nucleón $A_\sigma, A_\omega$ y $A_\rho$, junto con los parámetros de autointeracción del mesón escalar $b$ y $c$. Estos parámetros determinan completamente la ecuación de estado de la materia nuclear y, consecuentemente, las propiedades macroscópicas de las estrellas de neutrones a través de las ecuaciones de estructura TOV (\ref{eq:tov}). El ajuste de estos parámetros se realiza imponiendo que el modelo reproduzca las propiedades empíricas de la materia nuclear en saturación, descritas en la sección \ref{sec:saturacion}, y las observaciones astrofísicas, descritas en la sección \ref{sec:obsNS}.

Múltiples métodos son empleados para el ajuste de parámetros: algunos estudios utilizan análisis bayesianos \cite{chenBuildingRelativisticMean2014, huangConstrainingRelativisticMean2024}, mientras que otros aplican herramientas de aprendizaje de máquina y redes neuronales \cite{guoInsightsNeutronStar2024}. Sin embargo, cada conjunto de parámetros que satisface estas restricciones nucleares produce una ecuación de estado diferente a altas densidades, generando predicciones distintas para las propiedades estelares observables como la masa máxima y el radio. Por esta misma razón, dentro de este mismo formalismo han sido propuestos una gran variedad de modelos de ecuaciones de estado, los cuales se filtran añadiendo cada vez más restricciones físicas \cite{dutraRelativisticMeanFieldHadronic2014}. Esta falta de unicidad en el espacio de parámetros refleja la incertidumbre en la extrapolación desde la densidad de saturación nuclear hacia los regímenes de densidad extrema presentes en el interior de las estrellas de neutrones.

En este estudio, se realiza una exploración sistemática del espacio de parámetros del modelo relativista de campo medio con el fin de identificar los conjuntos de parámetros que satisfacen simultáneamente las restricciones nucleares y las observaciones astrofísicas actuales.

\subsection{Correlaciones entre Parámetros y Observables}

Debido a que los parámetros del modelo representan las intensidades físicas de las interacciones nucleares, es posible establecer correlaciones directas entre variaciones en estos parámetros y cambios en las propiedades nucleares y estelares. Por ejemplo, un aumento en el acoplamiento vectorial $A_\omega$ incrementa la repulsión entre nucleones, resultando en una disminución de la densidad de saturación $n_0$. Del mismo modo, un incremento en el acoplamiento isovectorial $A_\rho$ eleva la energía de simetría $a_\text{sym}$, afectando la composición protón-neutrón de la materia nuclear.

Consideremos el conjunto de parámetros \cite{glendenningCompactStarsNuclear2000}:

\begin{equation}
\begin{aligned}
&A_\sigma = 12.684 \, m^2, &
&A_\omega = 7.148 \, m^2, \\
&A_\rho = 4.410 \, m^2, &
&b = 5.610\times10^{-3}, \\
&c = -6.986\times10^{-3}, & &
\label{eq:params_glendenning}
\end{aligned}
\end{equation}
	
cuyas propiedades nucleares y estelares son:

\begin{equation}
\begin{aligned}
&n_0 = 0.153 \, \text{fm}^{-3}, &
&\frac{B}{A} = -16.30 \, \text{MeV}, \\
&K_0 = 201.23 \, \text{MeV}, &
&a_\text{sym} = 32.54 \, \text{MeV}, \\
&L_0 = 79.63 \, \text{MeV}, & 
&M_\text{max} = 2.33 \, \masasol, \\
&R_{1.4} = 13.76 \, \text{km}, &
&C_\text{max} = 0.298, &
\label{eq:props_glendenning}
\end{aligned}
\end{equation}

donde $C_\text{max} = G M_\text{max} / c^2 R_{(M_\text{max})}$ es la compacidad de la estrella de masa máxima, cuyo límite superior se establece al rededor de $C_\text{max}= \lesssim 0.33$ tras evaluar implicaciones de observaciones recientes en ecuaciones de estado \cite{annalaMultimessengerConstraintsUltradense2022}. Realizando variaciones de estos parámetros alrededor de los valores en (\ref{eq:params_glendenning}), se observan las tendencias recopiladas en la figura \ref{fig:correlacionesparams}. La figura muestra el coeficiente de correlación de Pearson entre cada parámetro y cada propiedad nuclear y estelar. Los valores representan la fuerza y dirección de la relación lineal entre dos variables: valores cercanos a 1.0 indican una correlación positiva fuerte (si aumento el parámetro, aumenta la propiedad), mientras que valores cercanos a -1.0 indican una correlación negativa fuerte (si aumento el parámetro, disminuye la propiedad). Los valores intermedios reflejan correlaciones más débiles o nulas, en las que el parámetro tiene poca o ninguna influencia en la propiedad considerada. Es necesario notar que estas correlaciones son válidas localmente alrededor del conjunto de parámetros elegido, pero pueden variar en otras regiones del espacio de parámetros. Además, este coeficiente cuantifica la fuerza de la correlación, más no la pendiente de la relación entre las variables. 


\begin{figure}[h]
	\centering
	\includegraphics[width=0.7\linewidth]{Figuras/correlaciones_params}
	\caption[Correlaciones entre parámetros y observables]{Correlaciones cualitativas entre variaciones en los parámetros del modelo relativista de campo medio alrededor de los valores en (\ref{eq:params_glendenning}), y cambios en las propiedades nucleares y estelares. Valores cercanos a 1.0 indican una correlación positiva fuerte, mientras que valores cercanos a -1.0 indican una correlación negativa fuerte. Los valores intermedios reflejan correlaciones más débiles o nulas.}
	\label{fig:correlacionesparams}
\end{figure}

Es necesario resaltar la independencia de $n_0$, $B/A$ y $K_0$ respecto a variaciones en $A_\rho$, lo cual es consistente con la interpretación física de este acoplamiento como responsable únicamente de las interacciones isovectoriales. Adicionalmente, parece oportuno basar el estudio en el plano de parámetros $A_\sigma - A_\omega$ pues ambos acoplamientos están fuertemente correlacionados con todas las propiedades nucleares y estelares, sugiriendo una mayor influencia en la determinación de la ecuación de estado. La naturaleza atractiva del acoplamiento escalar $A_\sigma$ y la naturaleza repulsiva del acoplamiento vectorial $A_\omega$ explican las correlaciones opuestas observadas en la figura \ref{fig:correlacionesparams} para estas dos interacciones, salvo para la masa máxima $M_\text{max}$, donde ambos acoplamientos muestran una correlación positiva fuerte. 

Por otro lado, la relación lineal entre los parámetros $b$ y $c$ con la compresibilidad $K_0$ no es tan clara como la de $A_\sigma$ y $A_\omega$, pero sigue siendo útil para ajustar esta propiedad nuclear. De igual forma, sus correlaciones con las demás propiedades nucleares y estelares son más débiles, aunque siguen siendo significativas para el ajuste del modelo. Finalmente, dado el interés en encontrar ecuaciones de estado que permitan estrellas de neutrones con masas máximas elevadas, según lo observado en la figura \ref{fig:correlacionesparams} parece conveniente aumentar simultáneamente los parámetros $A_\sigma$, $A_\omega$ y $c$, y disminuir $A_\rho$, siempre manteniendo las propiedades nucleares dentro de los rangos aceptables.

Para estudiar el impacto de los parámetros del modelo en la ecuación de estado y las propiedades macroscópicas de las estrellas de neutrones, se presentan en las figuras \ref{fig:variacion_acoplamientos} y \ref{fig:variacion_autointeraccion} las ecuaciones de estado y las curvas masa-radio obtenidas al variar independientemente cada uno de los parámetros alrededor de los valores en (\ref{eq:params_glendenning}), manteniendo los demás parámetros fijos. Las ecuaciones de estado se presentan a partir de la densidad de saturación $n_0 = 0.157 \, \text{fm}^{-3}$ ($\approx 2.6\times 10^{14} \, \text{g cm}^{-3}$) y hasta $n_0 = 1.5 \, \text{fm}^{-3}$ ($\approx 2.5\times 10^{15} \, \text{g cm}^{-3}$), mostrando su efecto en la materia del núcleo dado que la ecuación de estado para la corteza, discutida en la sección \ref{sec:corteza}, es independiente de estos parámetros y es igual para todos los casos. Con estas visualizaciones, podemos corroborar cómo afecta cada parámetro a la rigidez de la ecuación de estado, a la masa y al radio de las estrellas de neutrones. 

En cuanto a la rigidez de la ecuación de estado, se observa que los parámetros $A_\sigma$ y $A_\omega$ aumentan apreciable y considerablemente la rigidez al incrementar sus valores respectivamente, mientras que el parámetro $A_\rho$ muestra un impacto casi despreciable. Por otro lado, los parámetros de autointeracción escalar $b$ y $c$ reducen la rigidez de la ecuación de estado, siendo el efecto de $c$ considerablemente más pronunciado que el de $b$. Respecto a la masa máxima, el parámetro $A_\omega$ es el que tiene mayor impacto al aumentarla, seguido por $c$ que la aumenta notoriamente y $A_\sigma$ que la aumenta ligeramente. En contraste, $A_\rho$ tiene un impacto despreciable en la masa máxima, mientras que $b$ la reduce ligeramente. Por último, con relación al radio de las estrellas de neutrones, $A_\omega$ produce el mayor aumento, seguido por aumentos considerables y notorios de $c$ y $b$ respectivamente, mientras que $A_\rho$ solo produce un ligero aumento. Por el contrario, $A_\sigma$ es el único parámetro que disminuye considerablemente el radio al aumentar su valor.

\begin{figure}[h]
    \centering
    \begin{subfigure}{0.75\linewidth}
        \centering
        \includegraphics[width=\linewidth]{Figuras/variacion_params_sigma}
        \caption{}
        \label{fig:variacion_sigma}
    \end{subfigure}
    \begin{subfigure}{0.75\linewidth}
        \centering
        \includegraphics[width=\linewidth]{Figuras/variacion_params_omega}
        \caption{}
        \label{fig:variacion_omega}
    \end{subfigure}
    
    \begin{subfigure}{0.75\linewidth}
        \centering
        \includegraphics[width=\linewidth]{Figuras/variacion_params_rho}
        \caption{}
        \label{fig:variacion_rho}
    \end{subfigure}
    \caption[Ecuaciones de estado y curvas masa-radio al variar los parámetros de acoplamiento]{Ecuaciones de estado (izquierda) y curvas masa-radio (derecha) obtenidas al variar los parámetros de acoplamiento mesón-nucleón alrededor de los valores en (\ref{eq:params_glendenning}), manteniendo los demás parámetros fijos. (a) Variación de $A_\sigma$. (b) Variación de $A_\omega$. (c) Variación de $A_\rho$.}
    \label{fig:variacion_acoplamientos}
\end{figure}

\begin{figure}[h]
    \centering
    \begin{subfigure}{0.75\linewidth}
        \centering
        \includegraphics[width=\linewidth]{Figuras/variacion_params_b}
        \caption{}
        \label{fig:variacion_b}
    \end{subfigure}
    \begin{subfigure}{0.75\linewidth}
        \centering
        \includegraphics[width=\linewidth]{Figuras/variacion_params_c}
        \caption{}
        \label{fig:variacion_c}
    \end{subfigure}
    \caption[Ecuaciones de estado y curvas masa-radio al variar los parámetros de autointeracción escalar]{Ecuaciones de estado (izquierda) y curvas masa-radio (derecha) obtenidas al variar los parámetros de autointeracción del mesón escalar alrededor de los valores en (\ref{eq:params_glendenning}), manteniendo los demás parámetros fijos. (a) Variación de $b$. (b) Variación de $c$.}
    \label{fig:variacion_autointeraccion}
\end{figure}
\clearpage

\subsection{Regiones Válidas en el Espacio de Parámetros}

La exploración del espacio de parámetros del modelo relativista de campo medio es un problema de optimización multidimensional bajo múltiples restricciones. El espacio completo está definido por los cinco parámetros libres: $A_\sigma$, $A_\omega$, $A_\rho$, $b$ y $c$. La metodología empleada para identificar conjuntos de parámetros físicamente aceptables consta de varias etapas sistemáticas.

Queremos identificar los conjuntos de parámetros que satisfacen las restricciones dentro de los márgenes experimentales. Estos conjuntos comprenden regiones del espacio de parámetros que son consistentes con las propiedades nucleares conocidas. Como puede evidenciarse en la figura \ref{fig:correlacionesparams}, los parámetros del modelo que influyen claramente en todas las propiedades nucleares y estelares son $A_\sigma$ y $A_\omega$. Por lo tanto, el método sistemático empleado consiste en los siguientes pasos, ilustrados en el diagrama de flujo de la figura \ref{fig:flowchartmetodo}:

\begin{enumerate}
    \item Se elige un par de valores específicos para los parámetros $b$ y $c$, que junto con $A_\sigma$ y $A_\omega$ influyen en el módulo de compresibilidad $K_0$.
    
    \item Se realiza un barrido sistemático en el plano $A_\sigma - A_\omega$, buscando la región que satisface simultáneamente las restricciones de:
    \begin{itemize}
        \item Densidad de saturación: $0.147 \, \text{fm}^{-3} \leq n_0 \leq 0.167 \, \text{fm}^{-3}$ (\ref{eq:densidad_saturacion})
        \item Energía de enlace por nucleón: $-16.37 \, \text{MeV} \leq B/A \leq -15.57 \, \text{MeV}$ (\ref{eq:energia_enlace_saturacion})
    \end{itemize}
    
    \item Se verifica que los valores del módulo de compresibilidad $K_0$ dentro de esta región cumplan con la restricción: $190 \, \text{MeV} \leq K_0 \leq 270 \, \text{MeV}$ (\ref{eq:modulo_compresibilidad_empirico}).
    
    \item Si la región identificada satisface las tres restricciones anteriores, se ajusta el parámetro $A_\rho$ para satisfacer las restricciones de:
    \begin{itemize}
        \item Energía de simetría: $28.9 \, \text{MeV} \leq a_\text{sym} \leq 34.3 \, \text{MeV}$ (\ref{eq:energia_simetria_empirico})
        \item Pendiente de la energía de simetría: $40 \, \text{MeV} \leq L_0 \leq 100 \, \text{MeV}$. Aunque el rango experimental fiduciario es más reducido (\ref{eq:pendiente_simetria_empirico}), esta propiedad tiene mayor libertad debido a la falta de consenso en su valor, permitiéndonos utilizar este intervalo más amplio en nuestro análisis \cite{kumarTheoreticalExperimentalConstraints2024}.
    \end{itemize}
    Esto se realiza sin afectar las otras propiedades nucleares, pues estas no dependen de $A_\rho$.
    
    \item Se repite el procedimiento para diferentes elecciones de los parámetros $b$ y $c$, explorando así el espacio de parámetros del modelo.
    
    \item Finalmente, se calculan las propiedades estelares (masa máxima $M_\text{max}$, radio canónico $R_{1.4}$, y compacidad máxima $C_\text{max}$) para las regiones de parámetros que satisfacen todas las restricciones nucleares anteriores.
\end{enumerate}

%\clearpage

%\begin{wrapfigure}{l}{0.38\textwidth}
%    \centering
%    \includegraphics[width=\linewidth]{Figuras/flowchart_metodo}
%    \caption[Diagrama de flujo del método de exploración del espacio de parámetros]{Diagrama de flujo del procedimiento sistemático para explorar el espacio de parámetros del modelo relativista de campo medio.}
%    \label{fig:flowchartmetodo}
%\end{wrapfigure}

\begin{figure}
	\centering
	\includegraphics[width=.38\linewidth]{Figuras/flowchart_metodo}
	\caption[Diagrama de flujo del método de exploración del espacio de parámetros]{Diagrama de flujo del procedimiento sistemático para explorar el espacio de parámetros del modelo relativista de campo medio.}
	\label{fig:flowchartmetodo}
\end{figure}

Aplicando este método, se identifican múltiples conjuntos de parámetros que satisfacen las restricciones nucleares. Cada conjunto produce una ecuación de estado diferente a altas densidades, generando diferentes predicciones para las propiedades estelares. Por ejemplo, tras realizar este proceso para un par de valores específicos de $b$ y $c$, se obtiene la región del plano $A_\sigma - A_\omega$ mostrada en la figura \ref{fig:region_nuclear_a}. Cada punto en esta región corresponde a un conjunto de parámetros válido que puede ser empleado para determinar las propiedades estelares resultantes. Además, si variamos los valores de $b$ y $c$ de modo que $K_0$ se mantenga dentro del rango aceptable, y ajustamos $A_\rho$ para cumplir las propiedades de simetría, obtenemos otras regiones válidas, como la mostrada en la figura \ref{fig:region_nuclear_b}. Esta nueva región tiene valores mayores de $A_\sigma$, $A_\omega$ y $c$, y valores menores de $A_\rho$ y $b$ en comparación con la región anterior. Asimismo, aunque la nueva región tiene valores mayores de $K_0$ y valores ligeramente menores de $a_\text{sym}$ y $L_0$ respecto a la región anterior, ambos conjuntos cumplen con las restricciones nucleares establecidas experimentalmente, lo que demuestra la libertad en la elección de parámetros dentro del modelo.

\begin{figure}[h]
    \centering
    \begin{subfigure}{\linewidth}
        \centering
        \includegraphics[width=\linewidth]{Figuras/new_ejemplo_espacio_2}
        \caption{}
        \label{fig:region_nuclear_a}
    \end{subfigure}
    
    \begin{subfigure}{\linewidth}
        \centering
        \includegraphics[width=\linewidth]{Figuras/new_ejemplo_espacio_3}
        \caption{}
        \label{fig:region_nuclear_b}
    \end{subfigure}
    \caption[Regiones del espacio de parámetros que satisfacen las restricciones nucleares]{Región del espacio de parámetros que satisface las restricciones nucleares en el plano $A_\sigma - A_\omega$. (a) Conjunto de parámetros obtenido mediante la metodología descrita. (b) Región válida obtenida al variar $b$ y $c$ de (a) para mantener $K_0$ en el rango aceptable, y ajustar $A_\rho$ para las propiedades de simetría. Las figuras muestran diferentes regiones del plano para facilitar la visualización.}
    \label{fig:region_nuclear}
\end{figure}

Siguiendo esta metodología, se consiguen diversos conjuntos de parámetros que cumplen las restricciones nucleares. Una vez obtenidos estos conjuntos, es posible calcular las propiedades estelares correspondientes, como la masa máxima $M_\text{max}$, la compacidad máxima $C_\text{max}$ y el radio de estrellas de neutrones de masa $1.4 \, \masasol$, $R_{1.4}$. Para algunos conjuntos hallados, se muestran sus propiedades estelares en la figura \ref{fig:props_estelares}. La masa máxima, la propiedad de mayor interés astrofísico, varía significativamente entre los diferentes conjuntos de parámetros, con valores entre aproximadamente $2.02 \, \masasol$ y $2.63 \, \masasol$, y aumenta con valores mayores de $A_\sigma$ y $A_\omega$, al igual que el radio $R_{1.4}$, que toma valores entre $12.5 \, \text{km}$ y $13.7 \, \text{km}$. En cuanto a la compacidad máxima $C_\text{max}$, se observa una tendencia similar respecto a los parámetros $A_\sigma$ y $A_\omega$, con valores entre $0.279$ y $0.310$. Sin embargo, esta propiedad parece tener máximos locales en el plano que indican una dependencia más compleja con los parámetros del modelo.

Estos valores predichos por el modelo son consistentes con las observaciones astrofísicas actuales. En el rango de masas máximas obtenidas, salvo por la región en la figura \ref{fig:props_estelar_2}, predicen valores superiores al límite inferior de la estimación en radiación electromagnética para PSR J0952-0607 de $2.18 \, \masasol$ \cite{fonsecaRefinedMassGeometric2021}. Sin embargo, únicamente la región en la figura \ref{fig:props_estelar_4} contiene conjuntos de parámetros que superan el límite inferior del secundario en el evento de ondas gravitacionales \textit{GW190814} de $2.50 \, \masasol$ \cite{theligoscientificcollaborationGW190814GravitationalWaves2020}, candidato para estrella de neutrones. En cuanto a los radios canónicos, el rango de radios calculados está en concordancia con las estimaciones recientes basadas en observaciones, sugiriendo radios de entre $11.52$ km y $13.80$ km para estrellas de masa $1.4 \, \masasol$ como PSR J0030+0451 \cite{millerPSRJ0030+0451Mass2019, rileyNICERViewPSR2019}, PSR J0437-4715 \cite{choudhuryNICERViewNearest2024} y las estimaciones de los modelos basados en PSR J0740+6620 \cite{millerRadiusPSRJ0740+66202021} y \textit{GW190814} \cite{biswasGW190814PropertiesSecondary2021}. Finalmente, las compacidades máximas obtenidas están por debajo del límite teórico de $C_\text{max} \lesssim 0.33$ \cite{annalaMultimessengerConstraintsUltradense2022}. Este análisis permite, por ejemplo, descartar el conjunto de parámetros en la figura \ref{fig:props_estelar_2}, que no puede reproducir estrellas de neutrones con masas superiores a $2.18 \, \masasol$, inconsistente con las observaciones actuales.

En vista de que diferentes conjuntos de parámetros presentados satisfacen las restricciones nucleares impuestas y producen propiedades estelares consistentes con las observaciones, es evidente que el modelo relativista de campo medio con los parámetros adecuados tiene la capacidad de describir materia nuclear en estrellas de neutrones. Restricciones adicionales pueden imponerse para evaluar la validez del modelo en futuras investigaciones.

\begin{figure}[h]
    \centering
    \begin{subfigure}{\linewidth}
        \centering
        \includegraphics[width=0.99\linewidth]{Figuras/new_props_estelar_2}
        \caption{}
        \label{fig:props_estelar_2}
    \end{subfigure}
    
    \begin{subfigure}{\linewidth}
        \centering
        \includegraphics[width=0.99\linewidth]{Figuras/new_props_estelar_1}
        \caption{}
        \label{fig:props_estelar_1}
    \end{subfigure}
    
    \begin{subfigure}{\linewidth}
        \centering
        \includegraphics[width=0.99\linewidth]{Figuras/new_props_estelar_3}
        \caption{}
        \label{fig:props_estelar_3}
    \end{subfigure}
    
    \begin{subfigure}{\linewidth}
        \centering
        \includegraphics[width=0.99\linewidth]{Figuras/new_props_estelar_4}
        \caption{}
        \label{fig:props_estelar_4}
    \end{subfigure}
    \caption[Propiedades estelares para diferentes conjuntos de parámetros]{Propiedades estelares obtenidas para diferentes conjuntos de parámetros que satisfacen las restricciones nucleares. Las figuras muestran diferentes regiones del plano $A_\sigma - A_\omega$ para facilitar la visualización.}
    \label{fig:props_estelares}
\end{figure}
\clearpage


\subsection{Masa Máxima}

La búsqueda del conjunto de parámetros que maximiza la masa estelar predicha es un objetivo central del análisis. Este conjunto establece el límite superior teórico para la masa de estrellas de neutrones dentro del marco del modelo considerado, y su comparación con las observaciones de estrellas masivas como las mencionadas en la sección \ref{sec:obsNS} permite evaluar si el modelo es capaz de explicar las configuraciones estelares más extremas observadas.

\begin{figure}[h]
	\centering
	\begin{subfigure}{\linewidth}
		\centering
		\includegraphics[width=\linewidth]{Figuras/new_ejemplo_espacio_6}
		\caption{}
		\label{fig:ejemplo_espacio_maxmass}
	\end{subfigure}
	
	\begin{subfigure}{\linewidth}
		\centering
		\includegraphics[width=\linewidth]{Figuras/new_props_estelar_6}
		\caption{}
		\label{fig:props_estelar_maxmass}
	\end{subfigure}
	\caption[Región de mayor masa máxima en el espacio de parámetros]{Región del espacio de parámetros con la mayor masa estelar predicha. (a) Propiedades nucleares para la región de parámetros que produce la mayor masa máxima. (b) Propiedades estelares correspondientes: masa máxima $M_\text{max}$, compacidad máxima $C_\text{max}$ y radio canónico $R_{1.4}$ para configuraciones en esta región.}
	\label{fig:max_mass}
\end{figure}

La región de mayores masas obtenidas para este modelo mediante la metodología descrita, así como sus propiedades nucleares y estelares, se muestran en la figura \ref{fig:max_mass}. En este estudio, se logró obtener una masa máxima de $M_\text{max} = 2.79 \, \masasol$, superando ampliamente el límite observado en radiación electromagnética de $2.35 \, \masasol$ para PSR J0952-0607 \cite{romaniPSRJ09520607Fastest2022}, así como el observado en ondas gravitacionales de $2.59 \, \masasol$ para el secundario en \textit{GW190814} \cite{theligoscientificcollaborationGW190814GravitationalWaves2020}. Esto indica que el modelo relativista de campo medio puede reproducir estrellas de neutrones extremadamente masivas, siendo un marco teórico consistente con las observaciones astrofísicas más exigentes. No queda claro que la región de parámetros de mayor masa encontrada en este estudio sea el límite absoluto dentro del modelo, por lo que futuros estudios y técnicas de optimización más avanzadas podrían revelar conjuntos de parámetros que produzcan masas aún mayores, sin contradecir las restricciones nucleares impuestas. De la misma forma, se halla una compacidad máxima de $C_\text{max} = 0.316$, ligeramente inferior al límite estimado de $C_\text{max} \lesssim 0.33$ \cite{annalaMultimessengerConstraintsUltradense2022}, y un radio canónico máximo $R_{1.4}$ de entre $12.79$ y $13.72 \, \text{km}$ para estrellas de masa $1.4 \, \masasol$ en esta región de parámetros. Estos valores de radio son consistentes con las estimaciones basadas en observaciones astrofísicas actuales, que sugieren un radio máximo de $R_{1.4} \lesssim 13.8$ km \cite{millerPSRJ0030+0451Mass2019, rileyNICERViewPSR2019, biswasGW190814PropertiesSecondary2021}.

Otras regiones con masas aún mayores pueden encontrarse, como la que se muestra en la figura \ref{fig:mayor_masa_invalida}, que logra una masa máxima de hasta $M_\text{max} = 3.11 \, \masasol$ respetando las restricciones nucleares. No obstante, esta región predice radios canónicos de hasta $R_{1.4} = 14.28 \, \text{km}$, que exceden las estimaciones mencionadas anteriormente. Por esta razón, esta región es descartada en el análisis, ya que no es consistente con las observaciones astrofísicas actuales. Este ejemplo ilustra la importancia de considerar múltiples restricciones al ajustar los parámetros del modelo, ya que cumplir únicamente con las propiedades nucleares no garantiza la validez astrofísica del conjunto de parámetros.

\begin{figure}[h]
    \centering
    \includegraphics[width=\linewidth]{Figuras/new_props_estelar_5}
    \caption[Región no válida astrofísicamente]{Región del espacio de parámetros que produce una masa máxima elevada pero radios canónicos inconsistentes con las observaciones astrofísicas actuales.}
    \label{fig:mayor_masa_invalida}
\end{figure}

\subsection{Comparación de Regiones}

Las regiones de parámetros mostradas en las figuras \ref{fig:region_nuclear}, \ref{fig:props_estelares} y \ref{fig:max_mass} representan diferentes áreas del espacio de parámetros que satisfacen las restricciones nucleares y producen diversas propiedades estelares. Es relevante comparar estas regiones para entender cómo las variaciones en los parámetros afectan las predicciones del modelo. En la figura \ref{fig:comparacion_regiones} se presenta una comparación directa en el plano $A_\sigma$ - $A_\omega$ entre las regiones que satisfacen las restricciones nucleares y son consistentes con las estimaciones observacionales, y las masas máximas que producen. Como notamos anteriormente, el aumento en los acoplamientos $A_\sigma$ y $A_\omega$ conduce a un incremento en la masa máxima predicha. Más aún, para mantener las propiedades de simetría dentro de los rangos aceptables, es necesario reducir el acoplamiento isovectorial $A_\rho$ al aumentar $A_\sigma$ y $A_\omega$, y puede explicarse por la necesidad de compensar el efecto de los acoplamientos escalares y vectoriales que tienden a aumentar la energía, requiriendo una disminución en la contribución isovectorial. Esta comparación resalta la interdependencia entre los parámetros del modelo y las propiedades estelares, por lo que es importante considerar estas relaciones al ajustar el modelo a observaciones astrofísicas.

\begin{figure}[h]
    \centering
    \includegraphics[width=0.8\linewidth]{Figuras/new_comparacion_regiones_validas}
    \caption[Comparación de regiones en el plano $A_\sigma$ - $A_\omega$]{Comparación de las regiones del plano de parámetros $A_\sigma$ - $A_\omega$ que satisfacen las restricciones nucleares y son consistentes con las estimaciones observacionales, y las masas máximas que producen.}
    \label{fig:comparacion_regiones}
\end{figure}

% Introducimos y discutimos la comparación de las EoS y M-R obtenidas para la configuración de mayor masa máxima en cada región válida del espacio de parámetros.

Es interesante comparar las ecuaciones de estado y las curvas masa-radio obtenidas para la configuración de mayor masa máxima en cada región válida del espacio de parámetros. Esta comparación permite evaluar cómo las diferentes elecciones de parámetros afectan la rigidez de la ecuación de estado y las propiedades macroscópicas de las estrellas de neutrones. En la figura \ref{fig:comparacion_regiones_eos}, se presentan las ecuaciones de estado y las curvas masa-radio correspondientes a las configuraciones de mayor masa máxima en cada una de las regiones válidas discutidas anteriormente. Vale destacar que se comprobó que todas las ecuaciones de estado cumplen con el límite de velocidad de la luz, asegurando la causalidad del modelo en todas las configuraciones consideradas. Además, las ecuaciones de estado se presentan a partir de la densidad de saturación $n_0 = 0.157 \, \text{fm}^{-3}$ ($\approx 2.6\times 10^{14} \, \text{g cm}^{-3}$) y hasta $n_0 = 4.0 \, \text{fm}^{-3}$ ($\approx 6.7\times 10^{15} \, \text{g cm}^{-3}$), mostrando únicamente la materia del núcleo debido a que la ecuación de estado para la corteza, discutida en la sección \ref{sec:corteza}, es independiente de los parámetros del modelo y es igual para todas las regiones.

Observamos que las ecuaciones de estado varían entre las diferentes regiones, reflejando las distintas elecciones de parámetros. En particular, la rigidez de la ecuación de estado, que influye directamente en la masa máxima y el radio de las estrellas de neutrones, muestra diferencias apreciables aumentando con regiones que producen mayores masas máximas y radios canónicos. Las curvas masa-radio exhiben también variaciones considerables, con diferencias en la masa máxima alcanzada y en los radios correspondientes para estrellas de masa $1.4 \, \masasol$. Resaltamos el hecho de que todas las configuraciones presentadas cumplen con las restricciones nucleares y son consistentes con las observaciones astrofísicas actuales, lo que muestra la capacidad del modelo relativista de campo medio para describir la materia nuclear en estrellas de neutrones bajo diferentes elecciones de parámetros.

\begin{figure}[h]
    \centering
    \includegraphics[width=0.95\linewidth]{Figuras/comparacion_regiones_validas_eos}
    \caption[Comparación de ecuaciones de estado y curvas masa-radio de diferentes regiones]{Comparación de las ecuaciones de estado (izquierda) y las curvas masa-radio (derecha) obtenidas para la configuración de mayor masa máxima en cada región válida del espacio de parámetros, mostradas en la figura \ref{fig:comparacion_regiones}.}
    \label{fig:comparacion_regiones_eos}
\end{figure}

Adicionalmente, es pertinente analizar el comportamiento microscópico de la materia nuclear predicha por las configuraciones de mayor masa máxima en cada región. En la figura \ref{fig:comparacion_regiones_fraccion} se presentan la energía de enlace por nucleón ($B/A$) y la fracción de neutrones respecto a la densidad bariónica ($n_n/n_B$) para estas configuraciones. La energía de enlace se grafica en un rango de densidades desde $0.001 \, \text{fm}^{-3}$ ($\approx 1.7\times 10^{12} \, \text{g cm}^{-3}$) hasta $0.7 \, \text{fm}^{-3}$ ($\approx 1.2\times 10^{15} \, \text{g cm}^{-3}$), lo que permite observar el comportamiento de la materia nuclear simétrica en función de los parámetros de cada región en saturación y a densidades varias veces superior a la saturación. Por otro lado, la fracción de neutrones se presenta desde la densidad de saturación hasta $100 \, \text{fm}^{-3}$ ($\approx 1.7\times 10^{17} \, \text{g cm}^{-3}$), cubriendo las densidades supranucleares características del núcleo estelar. A partir de la gráfica de $B/A$, se confirma lo observado en la figura \ref{fig:comparacion_regiones_eos}: las ecuaciones de estado que producen mayores masas máximas son efectivamente más rígidas, ya que la energía de enlace aumenta más rápidamente con la densidad. Respecto a la composición, se observa que para una región de densidades intermedia, la fracción de neutrones es mayor para las configuraciones de menor masa máxima. Sin embargo, este comportamiento se invierte a mayores densidades, donde las configuraciones de mayor masa máxima pasan a tener la mayor fracción de neutrones. Esta inflexión sucede aproximadamente entre $0.55$ y $0.7 \, \text{fm}^{-3}$ ($\approx 9.2\times 10^{14}$ y $1.2\times 10^{15} \, \text{g cm}^{-3}$).

\begin{figure}[h]
    \centering
    \includegraphics[width=0.95\linewidth]{Figuras/comparacion_regiones_validas_fraccion}
    \caption[Energía de enlace y fracción de neutrones para diferentes regiones]{Energía de enlace por nucleón (izquierda) y fracción de neutrones (derecha) para la configuración de mayor masa máxima en las regiones de la figura \ref{fig:comparacion_regiones}. La facción de protones es entonces $n_p/n_B = 1 - n_n/n_B$.}
    \label{fig:comparacion_regiones_fraccion}
\end{figure}

%Capítulo 5
\chapter{Conclusiones}
\thispagestyle{fancy}

% Conclusiones

\section{Principales hallazgos}

\section{Limitaciones del estudio}

\section{Recomendaciones para futuros trabajos}

%Capítulo 6
\chapter{Apéndices}

\section{Valores esperados de los operadores}
\label{apendice:valores_esperados}

Para calcular los valores esperados de las corrientes en (\ref{eq:eom_meson_mf}), usaremos un método económico para ahorrarnos la construcción de los campos fermiónicos \cite{glendenningCompactStarsNuclear2000}. La forma general del valor esperado de un operador $\Gamma$ en el estado base de un sistema de muchos fermiones es:

\begin{equation}
	\langle \bar{\psi} \Gamma \psi \rangle = \sum_{\kappa}\int \frac{d^3p}{(2\pi)^3} (\bar{\psi} \Gamma \psi)_{\Vec{p},\kappa} \Theta(\epsilon_F - \epsilon(\Vec{p})_\kappa),
	\label{eq:valor_esperado_general}
\end{equation}

donde $\kappa$ representa los grados de libertad internos (espín, isospín), $\epsilon_F$ es la energía de Fermi del sistema, y $\Theta(x)$ la función escalón de Heaviside. $(\bar{\psi} \Gamma \psi)_{\Vec{p},\kappa}$ es el valor esperado, en el espacio de fase, del operador $\Gamma$ en el estado de un solo fermión con momento $\Vec{p}$ y grado de libertad interno $\kappa$. Para hallar el integrando acudimos al hamiltoniano de Dirac, y a su valor esperado:

\begin{equation}
	\begin{gathered}
		H_D^{I_3} = \gamma_0\left[\Vec{\gamma}\cdot\Vec{p} + \gamma^\mu(g_\omega \omega_\mu + g_\rho I_3\rho_{3\mu})+(m-g_\sigma\sigma)\right],\\
		(\psi^\dagger H_D^{I_3} \psi)_{\Vec{p},\kappa} = \epsilon(\Vec{p})_{\kappa} = P_0(\Vec{p}) + g_\omega\omega_0+g_\rho I_3\rho_{30}.
		\label{eq:hamiltonian}
	\end{gathered}
\end{equation}

Notamos que el hamiltoniano depende del isospín $I_3$ pero no del espín, de modo que los estados de espín son degenerados con ocupación dos. Tomando la derivada del hamiltoniano respecto a cualquier variable $\xi$ del hamiltoniano (\ref{eq:hamiltonian}), y usando la regla de la cadena, obtenemos \cite{glendenningCompactStarsNuclear2000}:

\begin{equation}
	\begin{aligned}
		\frac{\partial}{\partial \xi} (\psi^\dagger H_D^{I_3} \psi)_{\Vec{p},\kappa} = (\psi^\dagger \frac{\partial H_D^{I_3}}{\partial \xi} \psi)_{\Vec{p},\kappa} + \epsilon(\Vec{p})_\kappa \cancel{\frac{\partial}{\partial \xi} (\psi^\dagger \psi)_{\Vec{p},\kappa}}, \\
	\end{aligned}
	\label{eq:derivada_hamiltoniano}
\end{equation}

donde el último término se anula porque $\psi$ es una función propia. Tomando la derivada con respecto a $p^i$, se obtiene:

\begin{equation*}
	(\bar{\psi} \gamma^i \psi)_{\Vec{p},\kappa} = \frac{\partial \epsilon(\Vec{p})_\kappa}{\partial p^i},
\end{equation*}

y por (\ref{eq:valor_esperado_general}), la corriente de nucleones es:

\begin{equation}
	\begin{aligned}
		\langle \bar{\psi} \gamma^i \psi \rangle &= 2\sum_{I_3}\int \frac{d^3p}{(2\pi)^3} \frac{\partial \epsilon(\Vec{p})_{I_3}}{\partial p^i} \Theta(\epsilon_F - \epsilon(\Vec{p})_{I_3}) \\
												 &= 2\sum_{I_3}\int \frac{d^2p}{(2\pi)^3} \int dp^i \frac{\partial \epsilon(\Vec{p})_{I_3}}{\partial p^i} \Theta(\epsilon_F - \epsilon(\Vec{p})_{I_3}) \\
												 &= 0,
		\label{eq:corriente_barionica_espacial}
	\end{aligned}		
\end{equation}

donde la integral se anula porque los valores de energía en las fronteras del volumen (de momento) son los mismos. La componente espacial de la corriente bariónica es nula, como era de esperarse en un sistema estático y uniforme, de modo que las componentes espaciales tanto del mesón $\omega$ como del mesón $\rho_3$ se anulan (ver ecuaciones (\ref{eq:eom_meson_mf})). Como tenemos solo componentes temporales de estos mesones, la energía de las partículas (\ref{eq:particleenergy}) depende únicamente de la magnitud del momento $\epsilon(\Vec{p}) = \epsilon(p)$, lo que implica que la superficie de Fermi es esférica en el espacio de momentos.

Ahora, tomando la derivada (\ref{eq:derivada_hamiltoniano}) con respecto a $\omega_0$:

\begin{equation*}
	(\bar{\psi} \gamma^0 \psi)_{\Vec{p},\kappa} = 1,
\end{equation*}

luego por (\ref{eq:valor_esperado_general}), la densidad bariónica es:

\begin{equation}
	\begin{aligned}
		\langle \bar{\psi} \gamma^0 \psi \rangle &= 2\sum_{I_3}\int \frac{4 \pi p^2 dp}{(2\pi)^3} \Theta(\epsilon_F - \epsilon(p)_{I_3}) \\
												 &= \int_0^{p_{Fp}} \frac{p^2 dp}{\pi^2} + \int_0^{p_{Fn}} \frac{p^2 dp}{\pi^2} \\
												 &= \frac{1}{3\pi^2}(p_{Fp}^3 + p_{Fn}^3) \equiv n_p + n_n = n_B,
		\label{eq:densidad_barionica}
	\end{aligned}
\end{equation}

donde $p_{Fp}$ y $p_{Fn}$ son los momentos de Fermi para protones y neutrones respectivamente, y $n_B$ es la densidad bariónica total. De modo análogo, dado que el valor esperado es lineal y en vista de la ecuación para el campo rho (\ref{eq:eom_meson_mf}), la componente temporal de la tercera componente de la corriente de isospín es:

\begin{equation}
		\langle \bar{\psi} \gamma^0 \tau_3 \psi \rangle = \half (n_p - n_n).
		\label{eq:densidad_isospin}
\end{equation}

Finalmente, tomando la derivada (\ref{eq:derivada_hamiltoniano}) con respecto a $m^*$:

\begin{equation*}
	(\bar{\psi} \psi)_{\Vec{p},\kappa} = \frac{m^*}{\sqrt{p^2 + m^{*2}}},
\end{equation*}

y por (\ref{eq:valor_esperado_general}), la densidad escalar es:

\begin{equation}
	\begin{aligned}
		\langle \bar{\psi} \psi \rangle &= 2\sum_{I_3}\int \frac{4 \pi p^2 dp}{(2\pi)^3} \frac{m^*}{\sqrt{p^2 + m^{*2}}} \Theta(\epsilon_F - \epsilon(p)_{I_3}) \\
										&= \sum_N \frac{1}{\pi^2}\int_0^{p_{FN}} \frac{p^2 (m-g_\sigma\sigma) dp}{\sqrt{p^2 + (m-g_\sigma\sigma)^{2}}} \equiv n_s. \\
		\label{eq:densidad_escalar}
	\end{aligned}
\end{equation}


%\bibliographystyle{ieeetr}
%\bibliography{PropuestadeTesis}
\printbibliography


%\renewcommand{\bibname}{Referencias}
%\addcontentsline{toc}{chapter}{Referencias}
%\bibliography{Bibliografia} % Reemplaza 'biblio' con el nombre de tu archivo .bib
\end{document}
