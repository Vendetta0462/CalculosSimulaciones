\chapter{Introducción}
\thispagestyle{fancy}

% Introducción

La materia en las estrellas de neutrones (EN) alcanza uno de los estados más extremos y densos conocidos en el universo observable. Estos objetos compactos, resultado del colapso gravitacional de estrellas masivas (sugerido por primera vez en 1934 por Baade y Zwicky \cite{baadeRemarksSuperNovaeCosmic1934}), comprimen la materia a densidades que superan las condiciones terrestres \cite{glendenningCompactStarsNuclear2000}. Las propiedades y características de la materia nuclear a densidades tan altas son objeto de estudio en la actualidad. Estudiar estos objetos estelares abre las puertas a la comprensión de la materia densa, ofreciendo un laboratorio natural donde se pueden contrastar modelos teóricos mediante observaciones astrofísicas. En adición, es posible corroborar indirectamente la validez de teorías nucleares y subatómicas bajo condiciones actualmente inalcanzables en el laboratorio \cite{raaijmakersConstraintsDenseMatter2021}.

Para este trabajo, el estudio de las EN se centra en las Ecuaciones de Estado (EdE), que describen la relación entre presión, densidad de energía y temperatura, siendo clave para resolver las ecuaciones de Tolman-Oppenheimer-Volkoff (TOV). Estas ecuaciones determinan características macroscópicas de las EN, como la relación masa-radio y las propiedades de marea, que han sido objeto de observaciones cada vez más detalladas gracias a misiones como NICER \cite{romaniPSRJ09520607Fastest2022, fonsecaRefinedMassGeometric2021, antoniadisMassivePulsarCompact2013}, LIGO y VIRGO \cite{collaborationGWTC21DeepExtended2022,theligoscientificcollaborationGW170817ObservationGravitational2017, theligoscientificcollaborationGW190814GravitationalWaves2020}. Con estos avances, los modelos teóricos se ven enriquecidos y enfrentan desafíos, como la identificación de límites para la masa máxima de una EN posible en la naturaleza, o la existencia de una ``brecha de masa'' (Mass-Gap) entre las EN de mayor masa y los agujeros negros más ligeros \cite{shaoNeutronStarBlack2022a}.

En el desarrollo de EdE, la Teoría Relativista de Campo Medio (TRCM) surge como una teoría efectiva para modelar interacciones nucleares bajo condiciones extremas de densidad \cite{waleckaTheoryHighlyCondensed1974,waleckaRelativisticNuclearManyBody1986}. La TRCM permite construir EdE que consideran distintas composiciones y acoplamientos entre partículas, posibilitando ajustes a los modelos basados en observaciones y datos experimentales cercanos a la densidad de saturación nuclear \cite{dutraRelativisticMeanFieldHadronic2014}. Sin embargo, la estructura interna de las EN sugiere la posible presencia de fases exóticas de materia, tales como hiperones o materia de quarks, desafiando aún más las formulaciones de EdE, especialmente en aquellas estrellas que alcanzan masas cercanas a $2.5$ masas solares \cite{lopesNatureMassgapObject2022}. La búsqueda de una EdE que logre reproducir observaciones actuales es, por lo tanto, crucial no solo para describir el comportamiento de la materia nuclear densa, sino también para realizar estudios más detallados sobre el destino y evolución final de estos objetos.

En esta propuesta de investigación, se plantea construir modelos de EdE mediante la TRCM y ajustar sus parámetros a partir de las observaciones astrofísicas actuales. La exploración del espacio de parámetros contribuirá a identificar las implicaciones físicas de la materia en su forma más extrema según las observaciones y brindará predicciones de la teoría empleada sobre la naturaleza de los objetos ubicados en el Mass-Gap.

\section{Notación y Convenciones}

A lo largo de este trabajo empleamos convenciones de notación y diferentes sistemas de unidades dependiendo del contexto físico específico. Para el desarrollo de la Relatividad General y el estudio de la estructura macroscópica de objetos compactos, empleamos unidades geometrizadas donde la velocidad de la luz $c$ y la constante gravitacional de Newton $G$ se establecen igual a la unidad $c = G = 1$. Para el tratamiento de la física nuclear y la construcción de ecuaciones de estado microscópicas, utilizamos unidades naturales donde la velocidad de la luz $c$ y la constante de Planck reducida $\hbar$ se fijan como unitarias $c = \hbar = 1$. En cuanto a las convenciones de notación, las letras griegas $\{\mu, \nu, \rho, \sigma, \ldots\}$ denotan índices tensoriales del conjunto $\{0,1,2,3\}$, mientras que las letras latinas $\{i, j, k, \ldots\}$ denotan componentes del conjunto $\{1,2,3\}$. El tensor métrico $g_{\mu\nu}$ tiene signatura $(+,-,-,-)$ y utilizamos la convención de suma de Einstein para índices repetidos.

%Se plantea el problema de investigación a estudiar y se brinda un contexto más amplio en la sección \ref{Planteamiento}. A continuación, en la sección \ref{MarcoTeorico}, se expondrán los principales conceptos requeridos para abordar el problema de investigación. Los objetivos del trabajo se presentan en la sección \ref{Objetivos}, mientras que, en la sección \ref{Metodologia}, se presenta la metodología planteada para alcanzar los objetivos de investigación, así como el cronograma de actividades para el desarrollo del trabajo. 









 