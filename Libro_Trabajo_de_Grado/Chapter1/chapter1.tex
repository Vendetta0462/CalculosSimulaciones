\chapter{Introducción}

Las estrellas de neutrones presentan uno de los estados más extremos de la materia en el universo observable, empleándose como laboratorios naturales para el estudio de la física fundamental. Estos objetos compactos, remanentes del colapso gravitacional de estrellas masivas (un mecanismo sugerido por primera vez en 1934 por Baade y Zwicky \cite{baadeRemarksSuperNovaeCosmic1934}), comprimen masas comparables o superiores a la del Sol en radios de apenas unos pocos kilómetros. En su interior, la materia alcanza densidades que superan en varios órdenes de magnitud la densidad de saturación nuclear ($n_0 \approx 0.16 \, \text{fm}^{-3}$), creando condiciones de presión y densidad aún imposibles de reproducir en experimentos terrestres \cite{glendenningCompactStarsNuclear2000}. En consecuencia, el estudio de estos objetos permite poner a prueba las teorías de la interacción fuerte y la gravedad en regímenes de energía aún no explorados.

El vínculo entre la física nuclear microscópica y la estructura astrofísica macroscópica reside en la ecuación de estado, la cual describe la relación termodinámica entre las variables de estado, como la presión y la densidad de energía, del material estelar. Esta relación es el ingrediente faltante para resolver las ecuaciones de estructura estelar en Relatividad General, conocidas como ecuaciones de Tolman-Oppenheimer-Volkoff, determinando las propiedades observables de la estrella como la masa y el radio. Sin embargo, la determinación de la ecuación de estado a densidades supranucleares permanece como una gran incógnita de la física moderna. Mientras que las propiedades de la materia nuclear están restringidas cerca de la densidad de saturación gracias a experimentos de laboratorio, la extrapolación hacia densidades superiores depende fuertemente de los modelos teóricos empleados y de los grados de libertad considerados \cite{chatziioannouNeutronStarsDense2024}.

En la última década, la astrofísica de objetos compactos ha experimentado una revolución gracias a la astronomía de ondas gravitacionales y a las observaciones de rayos X de alta precisión. La detección de fusiones de estrellas de neutrones por las colaboraciones LIGO y Virgo \cite{collaborationGWTC21DeepExtended2022}, como el evento \textit{GW170817} \cite{theligoscientificcollaborationGW170817ObservationGravitational2017}, y el descubrimiento de objetos en el rango de masas entre las estrellas de neutrones más pesadas y los agujeros negros más ligeros, como el secundario en \textit{GW190814} \cite{theligoscientificcollaborationGW190814GravitationalWaves2020}, han motivado el estudio de sus implicaciones en las propiedades de la materia densa en su interior. Simultáneamente, la medición precisa de masas de púlsares, como PSR J0740+6620 \cite{fonsecaRefinedMassGeometric2021} y PSR J0952-0607 \cite{romaniPSRJ09520607Fastest2022}, junto con las estimaciones de radio obtenidas por la misión NICER, exigen que los modelos teóricos sean capaces de soportar masas elevadas sin violar las restricciones de tamaño \cite{antoniadisMassivePulsarCompact2013, shaoNeutronStarBlack2022}.

Para abordar este problema, la Teoría Relativista de Campo Medio se presenta como un marco teórico robusto y eficaz \cite{waleckaTheoryHighlyCondensed1974, waleckaRelativisticNuclearManyBody1986}. A diferencia de los enfoques no relativistas, este formalismo garantiza la covariancia y causalidad necesarias para una descripción física consistente de la materia ultradensa. No obstante, la teoría contiene parámetros libres asociados a los acoplamientos entre nucleones y mesones que deben ser calibrados cuidadosamente. La falta de unicidad en estos parámetros ha dado lugar a una gran variedad de modelos cuya validez debe ser confrontada continuamente con la nueva evidencia observacional \cite{dutraRelativisticMeanFieldHadronic2014}.

El presente trabajo tiene como objetivo realizar una exploración sistemática del espacio de parámetros del modelo relativista de campo medio con interacciones no lineales, con el fin de construir ecuaciones de estado que satisfagan simultáneamente las restricciones de la física nuclear experimental y las observaciones astrofísicas más recientes. Se busca identificar las regiones del espacio de parámetros que permiten la existencia de estrellas de neutrones masivas, evaluando la capacidad del modelo para explicar objetos extremos como el componente secundario de \textit{GW190814} \cite{lopesNatureMassgapObject2022}, analizando las implicaciones físicas de estas configuraciones en términos de la materia nuclear densa.

Este documento está estructurado de la siguiente manera: en el Capítulo 2 se presentan los fundamentos macrofísicos, derivando las ecuaciones de estructura estelar desde la Relatividad General. El Capítulo 3 detalla el marco microfísico, introduciendo la Teoría Relativista de Campo Medio y el modelo específico empleado. El Capítulo 4 expone los resultados de la exploración del espacio de parámetros, el análisis de las ecuaciones de estado resultantes y las características del modelo nuclear. Finalmente, en el Capítulo 5 se presentan las conclusiones y perspectivas futuras derivadas de esta investigación.

\section{Notación y Convenciones}

A lo largo de este trabajo empleamos convenciones de notación y diferentes sistemas de unidades dependiendo del contexto físico específico. Para el desarrollo de la Relatividad General y el estudio de la estructura macroscópica de objetos compactos, empleamos unidades geometrizadas donde la velocidad de la luz $c$ y la constante gravitacional de Newton $G$ se establecen igual a la unidad $c = G = 1$. Para el tratamiento de la física nuclear y la construcción de ecuaciones de estado microscópicas, utilizamos unidades naturales donde la velocidad de la luz $c$ y la constante de Planck reducida $\hbar$ se fijan como unitarias $c = \hbar = 1$. En cuanto a las convenciones de notación, las letras griegas $\{\mu, \nu, \rho, \sigma, \ldots\}$ denotan índices tensoriales del conjunto $\{0,1,2,3\}$, mientras que las letras latinas $\{i, j, k, \ldots\}$ denotan componentes del conjunto $\{1,2,3\}$. El tensor métrico $g_{\mu\nu}$ tiene signatura $(+,-,-,-)$ y utilizamos la convención de suma de Einstein para índices repetidos.

%Se plantea el problema de investigación a estudiar y se brinda un contexto más amplio en la sección \ref{Planteamiento}. A continuación, en la sección \ref{MarcoTeorico}, se expondrán los principales conceptos requeridos para abordar el problema de investigación. Los objetivos del trabajo se presentan en la sección \ref{Objetivos}, mientras que, en la sección \ref{Metodologia}, se presenta la metodología planteada para alcanzar los objetivos de investigación, así como el cronograma de actividades para el desarrollo del trabajo. 









 