\chapter{Resultados y discusión}
% Resultados

Este capítulo se estructura en dos partes complementarias. Primero, se realiza un análisis del comportamiento general del modelo, examinando las soluciones autoconsistentes para el campo escalar (\ref{eq:eom_sigma_adimensional}) y los momentos de Fermi del neutrón y protón (\ref{eq:ligaduras_adimensionales}), la descomposición de la energía de enlace en sus contribuciones mesónicas y las condiciones de aceptabilidad física, basadas en los criterios recopilados por Hernández et al. \cite{hernandezAcceptabilityConditionsRelativistic2021}.

La segunda parte aborda la caracterización sistemática del espacio de parámetros. El objetivo central es identificar los conjuntos de parámetros que satisfacen simultáneamente las restricciones nucleares y las observaciones astrofísicas actuales. Este estudio establece relaciones directas entre los parámetros microscópicos de la materia (\ref{eq:lagrangiano}) y las propiedades macroscópicas observables, permitiendo comprender cómo las interacciones nucleares determinan la estructura de las estrellas de neutrones. En particular, se busca determinar el conjunto de parámetros que maximiza la masa estelar predicha, estableciendo así límites superiores teóricos que puedan ser confrontados con las observaciones de los objetos más masivos detectados hasta la fecha.

\clearpage
\section{Sobre el Modelo Nuclear}

Antes de explorar el espacio de parámetros, es importante analizar el comportamiento de las variables dinámicas del modelo y cómo estas determinan las propiedades de la materia nuclear. A continuación, se discuten algunas características relevantes.

\subsection{Soluciones de Autoconsistencia}

La ecuación del movimiento para el campo escalar en su forma adimensional (\ref{eq:eom_sigma_adimensional}) junto con la condición de equilibrio beta y neutralidad de carga (\ref{eq:ligaduras_adimensionales}) forman un sistema de ecuaciones acopladas no lineales que deben resolverse de manera autoconsistente para cada valor de la densidad bariónica $n_B$. Para un conjunto de parámetros específico, se muestra en la figura \ref{fig:soluciones_autoconsistentes} la evolución de la masa efectiva adimensional $\xsigma = (1 - g_\sigma \sigma / m) = m^*/m$ y los momentos de Fermi adimensionales del neutrón $x_{Fn}$ y el protón $x_{Fp}$ en función de la densidad bariónica $n_B$. Las soluciones se muestran para densidades bariónicas desde $n_B = 0.001\,\text{fm}^{-3}$ ($\approx 1.7\times 10^{12} \, \text{g cm}^{-3}$) hasta $n_B = 100\,\text{fm}^{-3}$ ($\approx 1.7\times 10^{17} \, \text{g cm}^{-3}$), mostrando el comportamiento de estas variables en el rango relevante para la materia nuclear y las estrellas de neutrones.

\begin{figure}[h]
    \centering
    \includegraphics[width=0.95\linewidth]{Figuras/resultado_autoconsistencia}
    \caption[Soluciones autoconsistentes del modelo relativista de campo medio]{Soluciones autoconsistentes para la masa efectiva adimensional $\xsigma$ (izquierda) y los momentos de Fermi adimensionales del neutrón $x_{Fn}$ y el protón $x_{Fp}$ (derecha) en función de la densidad bariónica $n_B$. Se usaron los parámetros $\left(\frac{g_\sigma}{m_\sigma}\right)^2=12.684\,\text{fm}^2$, $\left(\frac{g_\omega}{m_\omega}\right)^2=7.148\,\text{fm}^2$, $\left(\frac{g_\rho}{m_\rho}\right)^2=4.410\,\text{fm}^2$, $b=5.610\times10^{-3}$, $c=-6.986\times10^{-3}$.}
    \label{fig:soluciones_autoconsistentes}
\end{figure}

Notamos de la figura \ref{fig:soluciones_autoconsistentes} que $\xsigma$ disminuye con la densidad, disminuyendo la masa efectiva de los nucleones en el medio. Además, el valor de $\xsigma$ parece estar acotado tanto superior como inferiormente, donde a densidades cada vez menores ($n_B << n_0 = 0.157\,\text{fm}^{-3}$) se aproxima a 1, es decir, la masa efectiva tiende a la masa del nucleón en el vacío $m^* \to m$, y por otro lado, a densidades cada vez mayores ($n_B >> n_0$) la masa efectiva tiende a cero $m^* \to 0$. Esta disminución de la masa efectiva con la densidad implica que el mesón escalar $\sigma$ disminuye la energía $e(p) = \sqrt{p^2 + m^{*2}}$ de los nucleones en el medio cuando los nucleones se acercan entre sí, reflejando la naturaleza atractiva de esta interacción. En cuanto a los momentos de fermi, se observa que ambos aumentan con la densidad bariónica, siendo $x_{Fn}$ siempre mayor a $x_{Fp}$ debido a la condición de equilibrio beta y neutralidad de carga. A medida que la densidad aumenta, la diferencia entre ambos momentos de Fermi disminuye, indicando que la materia se hace más simétrica en isospin a altas densidades, aunque siempre con exceso de neutrones. Comparando los comportamientos de $\xsigma$, $x_{Fn}$ y $x_{Fp}$, se observa que la disminución de la masa efectiva está correlacionada con el aumento de los momentos de Fermi, ya que una masa efectiva menor implica una mayor energía cinética para los nucleones a un mismo nivel de Fermi, favoreciendo la ocupación de estados de mayor momento.

\subsection{Descomposición de la Energía de Enlace}

La energía de enlace por nucleón $E/A$ surge de la competencia entre la energía cinética de los nucleones y las contribuciones de los campos mesónicos. Es posible descomponer esta energía en términos de los potenciales asociados a los mesones $\sigma$ (atractivo), $\omega$ (repulsivo) y $\rho$ (isovectorial). A bajas densidades, la contribución atractiva del campo $\sigma$ domina, permitiendo la formación de estados ligados y la saturación de la materia nuclear. A medida que aumenta la densidad, el término repulsivo del campo $\omega$ crece rápidamente, dominando la energética y proporcionando la rigidez necesaria para soportar estrellas de neutrones masivas contra el colapso gravitacional. El campo $\rho$, por otro lado, contribuye principalmente a la energía de simetría, diferenciando la energía de la materia de neutrones pura respecto a la materia simétrica isospin.

\subsection{Condiciones de Aceptabilidad Física}

Para que una ecuación de estado sea físicamente admisible, debe satisfacer ciertos criterios teóricos fundamentales en todo el rango de densidades relevante para las estrellas de neutrones:

\begin{itemize}
    \item \textbf{Estabilidad Microscópica:} La presión debe ser una función creciente de la densidad de energía, es decir, $dP/d\mathcal{E} > 0$. Esto asegura la estabilidad mecánica y termodinámica de la materia frente a fluctuaciones locales.
    \item \textbf{Causalidad:} La velocidad del sonido en el medio, definida como $v_s = \sqrt{dP/d\mathcal{E}}$, no debe exceder la velocidad de la luz en el vacío ($v_s \le 1$ en unidades naturales). La violación de esta condición implicaría la propagación de señales superlumínicas, lo cual es incompatible con la relatividad especial.
\end{itemize}

Todas las ecuaciones de estado consideradas en este trabajo son verificadas para cumplir estrictamente con las condiciones de estabilidad y causalidad hasta las densidades centrales máximas alcanzadas en las estrellas de neutrones, asegurando la consistencia física de los resultados presentados.

\section{Estudio del Espacio de Parámetros}

Como se estableció en el capítulo anterior (sección \ref{sec:modelo}), el modelo considerado para la ecuación de estado en el marco de la teoría relativista de campo medio contiene cinco parámetros libres: los tres acoplamientos mesón-nucleón $A_\sigma, A_\omega$ y $A_\rho$, junto con los parámetros de autointeracción del mesón escalar $b$ y $c$. Estos parámetros determinan completamente la ecuación de estado de la materia nuclear y, consecuentemente, las propiedades macroscópicas de las estrellas de neutrones a través de las ecuaciones de estructura TOV (\ref{eq:tov}). El ajuste de estos parámetros se realiza imponiendo que el modelo reproduzca las propiedades empíricas de la materia nuclear en saturación, descritas en la sección \ref{sec:saturacion}, y las observaciones astrofísicas, descritas en la sección \ref{sec:obsNS}.

Múltiples métodos son empleados para el ajuste de parámetros: algunos estudios utilizan análisis bayesianos \cite{chenBuildingRelativisticMean2014, huangConstrainingRelativisticMean2024}, mientras que otros aplican herramientas de aprendizaje de máquina y redes neuronales \cite{guoInsightsNeutronStar2024}. Sin embargo, cada conjunto de parámetros que satisface estas restricciones nucleares produce una ecuación de estado diferente a altas densidades, generando predicciones distintas para las propiedades estelares observables como la masa máxima y el radio. Por esta misma razón, dentro de este mismo formalismo han sido propuestos una gran variedad de modelos de ecuaciones de estado, los cuales se filtran añadiendo cada vez más restricciones físicas \cite{dutraRelativisticMeanFieldHadronic2014}. Esta falta de unicidad en el espacio de parámetros refleja la incertidumbre en la extrapolación desde la densidad de saturación nuclear hacia los regímenes de densidad extrema presentes en el interior de las estrellas de neutrones.

En este estudio, se realiza una exploración sistemática del espacio de parámetros del modelo relativista de campo medio con el fin de identificar los conjuntos de parámetros que satisfacen simultáneamente las restricciones nucleares y las observaciones astrofísicas actuales.

\subsection{Correlaciones entre Parámetros y Observables}

Debido a que los parámetros del modelo representan las intensidades físicas de las interacciones nucleares, es posible establecer correlaciones directas entre variaciones en estos parámetros y cambios en las propiedades nucleares y estelares. Por ejemplo, un aumento en el acoplamiento vectorial $A_\omega$ incrementa la repulsión entre nucleones, resultando en una disminución de la densidad de saturación $n_0$. Del mismo modo, un incremento en el acoplamiento isovectorial $A_\rho$ eleva la energía de simetría $a_\text{sym}$, afectando la composición protón-neutrón de la materia nuclear.

Consideremos el conjunto de parámetros \cite{glendenningCompactStarsNuclear2000}:

\begin{equation}
\begin{aligned}
&A_\sigma = 12.684 \, m^2, &
&A_\omega = 7.148 \, m^2, \\
&A_\rho = 4.410 \, m^2, &
&b = 5.610\times10^{-3}, \\
&c = -6.986\times10^{-3}, & &
\label{eq:params_glendenning}
\end{aligned}
\end{equation}
	
cuyas propiedades nucleares y estelares son:

\begin{equation}
\begin{aligned}
&n_0 = 0.153 \, \text{fm}^{-3}, &
&\frac{B}{A} = -16.30 \, \text{MeV}, \\
&K_0 = 201.23 \, \text{MeV}, &
&a_\text{sym} = 32.54 \, \text{MeV}, \\
&L_0 = 79.63 \, \text{MeV}, & 
&M_\text{max} = 2.33 \, \masasol, \\
&R_{1.4} = 13.76 \, \text{km}, &
&C_\text{max} = 0.298, &
\label{eq:props_glendenning}
\end{aligned}
\end{equation}

donde $C_\text{max} = G M_\text{max} / c^2 R_{(M_\text{max})}$ es la compacidad de la estrella de masa máxima, cuyo límite superior se establece al rededor de $C_\text{max}= \lesssim 0.33$ tras evaluar implicaciones de observaciones recientes en ecuaciones de estado \cite{annalaMultimessengerConstraintsUltradense2022}. Realizando variaciones de estos parámetros alrededor de los valores en (\ref{eq:params_glendenning}), se observan las tendencias recopiladas en la figura \ref{fig:correlacionesparams}. La figura muestra el coeficiente de correlación de Pearson entre cada parámetro y cada propiedad nuclear y estelar. Los valores representan la fuerza y dirección de la relación lineal entre dos variables: valores cercanos a 1.0 indican una correlación positiva fuerte (si aumento el parámetro, aumenta la propiedad), mientras que valores cercanos a -1.0 indican una correlación negativa fuerte (si aumento el parámetro, disminuye la propiedad). Los valores intermedios reflejan correlaciones más débiles o nulas, en las que el parámetro tiene poca o ninguna influencia en la propiedad considerada. Es necesario notar que estas correlaciones son válidas localmente alrededor del conjunto de parámetros elegido, pero pueden variar en otras regiones del espacio de parámetros. Además, este coeficiente cuantifica la fuerza de la correlación, más no la pendiente de la relación entre las variables. 


\begin{figure}[h]
	\centering
	\includegraphics[width=0.7\linewidth]{Figuras/correlaciones_params}
	\caption[Correlaciones entre parámetros y observables]{Correlaciones cualitativas entre variaciones en los parámetros del modelo relativista de campo medio alrededor de los valores en (\ref{eq:params_glendenning}), y cambios en las propiedades nucleares y estelares. Valores cercanos a 1.0 indican una correlación positiva fuerte, mientras que valores cercanos a -1.0 indican una correlación negativa fuerte. Los valores intermedios reflejan correlaciones más débiles o nulas.}
	\label{fig:correlacionesparams}
\end{figure}

Es necesario resaltar la independencia de $n_0$, $B/A$ y $K_0$ respecto a variaciones en $A_\rho$, lo cual es consistente con la interpretación física de este acoplamiento como responsable únicamente de las interacciones isovectoriales. Adicionalmente, parece oportuno basar el estudio en el plano de parámetros $A_\sigma - A_\omega$ pues ambos acoplamientos están fuertemente correlacionados con todas las propiedades nucleares y estelares, sugiriendo una mayor influencia en la determinación de la ecuación de estado. La naturaleza atractiva del acoplamiento escalar $A_\sigma$ y la naturaleza repulsiva del acoplamiento vectorial $A_\omega$ explican las correlaciones opuestas observadas en la figura \ref{fig:correlacionesparams} para estas dos interacciones, salvo para la masa máxima $M_\text{max}$, donde ambos acoplamientos muestran una correlación positiva fuerte. 

Por otro lado, la relación lineal entre los parámetros $b$ y $c$ con la compresibilidad $K_0$ no es tan clara como la de $A_\sigma$ y $A_\omega$, pero sigue siendo útil para ajustar esta propiedad nuclear. De igual forma, sus correlaciones con las demás propiedades nucleares y estelares son más débiles, aunque siguen siendo significativas para el ajuste del modelo. Finalmente, dado el interés en encontrar ecuaciones de estado que permitan estrellas de neutrones con masas máximas elevadas, según lo observado en la figura \ref{fig:correlacionesparams} parece conveniente aumentar simultáneamente los parámetros $A_\sigma$, $A_\omega$ y $c$, y disminuir $A_\rho$, siempre manteniendo las propiedades nucleares dentro de los rangos aceptables.

Para estudiar el impacto de los parámetros del modelo en la ecuación de estado y las propiedades macroscópicas de las estrellas de neutrones, se presentan en las figuras \ref{fig:variacion_acoplamientos} y \ref{fig:variacion_autointeraccion} las ecuaciones de estado y las curvas masa-radio obtenidas al variar independientemente cada uno de los parámetros alrededor de los valores en (\ref{eq:params_glendenning}), manteniendo los demás parámetros fijos. Las ecuaciones de estado se presentan a partir de la densidad de saturación $n_0 = 0.157 \, \text{fm}^{-3}$ ($\approx 2.6\times 10^{14} \, \text{g cm}^{-3}$) y hasta $n_0 = 1.5 \, \text{fm}^{-3}$ ($\approx 2.5\times 10^{15} \, \text{g cm}^{-3}$), mostrando su efecto en la materia del núcleo dado que la ecuación de estado para la corteza, discutida en la sección \ref{sec:corteza}, es independiente de estos parámetros y es igual para todos los casos. Con estas visualizaciones, podemos corroborar cómo afecta cada parámetro a la rigidez de la ecuación de estado, a la masa y al radio de las estrellas de neutrones. 

En cuanto a la rigidez de la ecuación de estado, se observa que los parámetros $A_\sigma$ y $A_\omega$ aumentan apreciable y considerablemente la rigidez al incrementar sus valores respectivamente, mientras que el parámetro $A_\rho$ muestra un impacto casi despreciable. Por otro lado, los parámetros de autointeracción escalar $b$ y $c$ reducen la rigidez de la ecuación de estado, siendo el efecto de $c$ considerablemente más pronunciado que el de $b$. Respecto a la masa máxima, el parámetro $A_\omega$ es el que tiene mayor impacto al aumentarla, seguido por $c$ que la aumenta notoriamente y $A_\sigma$ que la aumenta ligeramente. En contraste, $A_\rho$ tiene un impacto despreciable en la masa máxima, mientras que $b$ la reduce ligeramente. Por último, con relación al radio de las estrellas de neutrones, $A_\omega$ produce el mayor aumento, seguido por aumentos considerables y notorios de $c$ y $b$ respectivamente, mientras que $A_\rho$ solo produce un ligero aumento. Por el contrario, $A_\sigma$ es el único parámetro que disminuye considerablemente el radio al aumentar su valor.

\begin{figure}[h]
    \centering
    \begin{subfigure}{0.75\linewidth}
        \centering
        \includegraphics[width=\linewidth]{Figuras/variacion_params_sigma}
        \caption{}
        \label{fig:variacion_sigma}
    \end{subfigure}
    \begin{subfigure}{0.75\linewidth}
        \centering
        \includegraphics[width=\linewidth]{Figuras/variacion_params_omega}
        \caption{}
        \label{fig:variacion_omega}
    \end{subfigure}
    
    \begin{subfigure}{0.75\linewidth}
        \centering
        \includegraphics[width=\linewidth]{Figuras/variacion_params_rho}
        \caption{}
        \label{fig:variacion_rho}
    \end{subfigure}
    \caption[Ecuaciones de estado y curvas masa-radio al variar los parámetros de acoplamiento]{Ecuaciones de estado (izquierda) y curvas masa-radio (derecha) obtenidas al variar los parámetros de acoplamiento mesón-nucleón alrededor de los valores en (\ref{eq:params_glendenning}), manteniendo los demás parámetros fijos. (a) Variación de $A_\sigma$. (b) Variación de $A_\omega$. (c) Variación de $A_\rho$.}
    \label{fig:variacion_acoplamientos}
\end{figure}

\begin{figure}[h]
    \centering
    \begin{subfigure}{0.75\linewidth}
        \centering
        \includegraphics[width=\linewidth]{Figuras/variacion_params_b}
        \caption{}
        \label{fig:variacion_b}
    \end{subfigure}
    \begin{subfigure}{0.75\linewidth}
        \centering
        \includegraphics[width=\linewidth]{Figuras/variacion_params_c}
        \caption{}
        \label{fig:variacion_c}
    \end{subfigure}
    \caption[Ecuaciones de estado y curvas masa-radio al variar los parámetros de autointeracción escalar]{Ecuaciones de estado (izquierda) y curvas masa-radio (derecha) obtenidas al variar los parámetros de autointeracción del mesón escalar alrededor de los valores en (\ref{eq:params_glendenning}), manteniendo los demás parámetros fijos. (a) Variación de $b$. (b) Variación de $c$.}
    \label{fig:variacion_autointeraccion}
\end{figure}
\clearpage

\subsection{Regiones Válidas en el Espacio de Parámetros}

La exploración del espacio de parámetros del modelo relativista de campo medio es un problema de optimización multidimensional bajo múltiples restricciones. El espacio completo está definido por los cinco parámetros libres: $A_\sigma$, $A_\omega$, $A_\rho$, $b$ y $c$. La metodología empleada para identificar conjuntos de parámetros físicamente aceptables consta de varias etapas sistemáticas.

Queremos identificar los conjuntos de parámetros que satisfacen las restricciones dentro de los márgenes experimentales. Estos conjuntos comprenden regiones del espacio de parámetros que son consistentes con las propiedades nucleares conocidas. Como puede evidenciarse en la figura \ref{fig:correlacionesparams}, los parámetros del modelo que influyen claramente en todas las propiedades nucleares y estelares son $A_\sigma$ y $A_\omega$. Por lo tanto, el método sistemático empleado consiste en los siguientes pasos, ilustrados en el diagrama de flujo de la figura \ref{fig:flowchartmetodo}:

\begin{enumerate}
    \item Se elige un par de valores específicos para los parámetros $b$ y $c$, que junto con $A_\sigma$ y $A_\omega$ influyen en el módulo de compresibilidad $K_0$.
    
    \item Se realiza un barrido sistemático en el plano $A_\sigma - A_\omega$, buscando la región que satisface simultáneamente las restricciones de:
    \begin{itemize}
        \item Densidad de saturación: $0.147 \, \text{fm}^{-3} \leq n_0 \leq 0.167 \, \text{fm}^{-3}$ (\ref{eq:densidad_saturacion})
        \item Energía de enlace por nucleón: $-16.37 \, \text{MeV} \leq B/A \leq -15.57 \, \text{MeV}$ (\ref{eq:energia_enlace_saturacion})
    \end{itemize}
    
    \item Se verifica que los valores del módulo de compresibilidad $K_0$ dentro de esta región cumplan con la restricción: $190 \, \text{MeV} \leq K_0 \leq 270 \, \text{MeV}$ (\ref{eq:modulo_compresibilidad_empirico}).
    
    \item Si la región identificada satisface las tres restricciones anteriores, se ajusta el parámetro $A_\rho$ para satisfacer las restricciones de:
    \begin{itemize}
        \item Energía de simetría: $28.9 \, \text{MeV} \leq a_\text{sym} \leq 34.3 \, \text{MeV}$ (\ref{eq:energia_simetria_empirico})
        \item Pendiente de la energía de simetría: $40 \, \text{MeV} \leq L_0 \leq 100 \, \text{MeV}$. Aunque el rango experimental fiduciario es más reducido (\ref{eq:pendiente_simetria_empirico}), esta propiedad tiene mayor libertad debido a la falta de consenso en su valor, permitiéndonos utilizar este intervalo más amplio en nuestro análisis \cite{kumarTheoreticalExperimentalConstraints2024}.
    \end{itemize}
    Esto se realiza sin afectar las otras propiedades nucleares, pues estas no dependen de $A_\rho$.
    
    \item Se repite el procedimiento para diferentes elecciones de los parámetros $b$ y $c$, explorando así el espacio de parámetros del modelo.
    
    \item Finalmente, se calculan las propiedades estelares (masa máxima $M_\text{max}$, radio canónico $R_{1.4}$, y compacidad máxima $C_\text{max}$) para las regiones de parámetros que satisfacen todas las restricciones nucleares anteriores.
\end{enumerate}

%\clearpage

%\begin{wrapfigure}{l}{0.38\textwidth}
%    \centering
%    \includegraphics[width=\linewidth]{Figuras/flowchart_metodo}
%    \caption[Diagrama de flujo del método de exploración del espacio de parámetros]{Diagrama de flujo del procedimiento sistemático para explorar el espacio de parámetros del modelo relativista de campo medio.}
%    \label{fig:flowchartmetodo}
%\end{wrapfigure}

\begin{figure}
	\centering
	\includegraphics[width=.38\linewidth]{Figuras/flowchart_metodo}
	\caption[Diagrama de flujo del método de exploración del espacio de parámetros]{Diagrama de flujo del procedimiento sistemático para explorar el espacio de parámetros del modelo relativista de campo medio.}
	\label{fig:flowchartmetodo}
\end{figure}

Aplicando este método, se identifican múltiples conjuntos de parámetros que satisfacen las restricciones nucleares. Cada conjunto produce una ecuación de estado diferente a altas densidades, generando diferentes predicciones para las propiedades estelares. Por ejemplo, tras realizar este proceso para un par de valores específicos de $b$ y $c$, se obtiene la región del plano $A_\sigma - A_\omega$ mostrada en la figura \ref{fig:region_nuclear_a}. Cada punto en esta región corresponde a un conjunto de parámetros válido que puede ser empleado para determinar las propiedades estelares resultantes. Además, si variamos los valores de $b$ y $c$ de modo que $K_0$ se mantenga dentro del rango aceptable, y ajustamos $A_\rho$ para cumplir las propiedades de simetría, obtenemos otras regiones válidas, como la mostrada en la figura \ref{fig:region_nuclear_b}. Esta nueva región tiene valores mayores de $A_\sigma$, $A_\omega$ y $c$, y valores menores de $A_\rho$ y $b$ en comparación con la región anterior. Asimismo, aunque la nueva región tiene valores mayores de $K_0$ y valores ligeramente menores de $a_\text{sym}$ y $L_0$ respecto a la región anterior, ambos conjuntos cumplen con las restricciones nucleares establecidas experimentalmente, lo que demuestra la libertad en la elección de parámetros dentro del modelo.

\begin{figure}[h]
    \centering
    \begin{subfigure}{\linewidth}
        \centering
        \includegraphics[width=\linewidth]{Figuras/new_ejemplo_espacio_2}
        \caption{}
        \label{fig:region_nuclear_a}
    \end{subfigure}
    
    \begin{subfigure}{\linewidth}
        \centering
        \includegraphics[width=\linewidth]{Figuras/new_ejemplo_espacio_3}
        \caption{}
        \label{fig:region_nuclear_b}
    \end{subfigure}
    \caption[Regiones del espacio de parámetros que satisfacen las restricciones nucleares]{Región del espacio de parámetros que satisface las restricciones nucleares en el plano $A_\sigma - A_\omega$. (a) Conjunto de parámetros obtenido mediante la metodología descrita. (b) Región válida obtenida al variar $b$ y $c$ de (a) para mantener $K_0$ en el rango aceptable, y ajustar $A_\rho$ para las propiedades de simetría. Las figuras muestran diferentes regiones del plano para facilitar la visualización.}
    \label{fig:region_nuclear}
\end{figure}

Siguiendo esta metodología, se consiguen diversos conjuntos de parámetros que cumplen las restricciones nucleares. Una vez obtenidos estos conjuntos, es posible calcular las propiedades estelares correspondientes, como la masa máxima $M_\text{max}$, la compacidad máxima $C_\text{max}$ y el radio de estrellas de neutrones de masa $1.4 \, \masasol$, $R_{1.4}$. Para algunos conjuntos hallados, se muestran sus propiedades estelares en la figura \ref{fig:props_estelares}. La masa máxima, la propiedad de mayor interés astrofísico, varía significativamente entre los diferentes conjuntos de parámetros, con valores entre aproximadamente $2.02 \, \masasol$ y $2.63 \, \masasol$, y aumenta con valores mayores de $A_\sigma$ y $A_\omega$, al igual que el radio $R_{1.4}$, que toma valores entre $12.5 \, \text{km}$ y $13.7 \, \text{km}$. En cuanto a la compacidad máxima $C_\text{max}$, se observa una tendencia similar respecto a los parámetros $A_\sigma$ y $A_\omega$, con valores entre $0.279$ y $0.310$. Sin embargo, esta propiedad parece tener máximos locales en el plano que indican una dependencia más compleja con los parámetros del modelo.

Estos valores predichos por el modelo son consistentes con las observaciones astrofísicas actuales. En el rango de masas máximas obtenidas, salvo por la región en la figura \ref{fig:props_estelar_2}, predicen valores superiores al límite inferior de la estimación en radiación electromagnética para PSR J0952-0607 de $2.18 \, \masasol$ \cite{fonsecaRefinedMassGeometric2021}. Sin embargo, únicamente la región en la figura \ref{fig:props_estelar_4} contiene conjuntos de parámetros que superan el límite inferior del secundario en el evento de ondas gravitacionales \textit{GW190814} de $2.50 \, \masasol$ \cite{theligoscientificcollaborationGW190814GravitationalWaves2020}, candidato para estrella de neutrones. En cuanto a los radios canónicos, el rango de radios calculados está en concordancia con las estimaciones recientes basadas en observaciones, sugiriendo radios de entre $11.52$ km y $13.80$ km para estrellas de masa $1.4 \, \masasol$ como PSR J0030+0451 \cite{millerPSRJ0030+0451Mass2019, rileyNICERViewPSR2019}, PSR J0437-4715 \cite{choudhuryNICERViewNearest2024} y las estimaciones de los modelos basados en PSR J0740+6620 \cite{millerRadiusPSRJ0740+66202021} y \textit{GW190814} \cite{biswasGW190814PropertiesSecondary2021}. Finalmente, las compacidades máximas obtenidas están por debajo del límite teórico de $C_\text{max} \lesssim 0.33$ \cite{annalaMultimessengerConstraintsUltradense2022}. Este análisis permite, por ejemplo, descartar el conjunto de parámetros en la figura \ref{fig:props_estelar_2}, que no puede reproducir estrellas de neutrones con masas superiores a $2.18 \, \masasol$, inconsistente con las observaciones actuales.

En vista de que diferentes conjuntos de parámetros presentados satisfacen las restricciones nucleares impuestas y producen propiedades estelares consistentes con las observaciones, es evidente que el modelo relativista de campo medio con los parámetros adecuados tiene la capacidad de describir materia nuclear en estrellas de neutrones. Restricciones adicionales pueden imponerse para evaluar la validez del modelo en futuras investigaciones.

\begin{figure}[h]
    \centering
    \begin{subfigure}{\linewidth}
        \centering
        \includegraphics[width=0.99\linewidth]{Figuras/new_props_estelar_2}
        \caption{}
        \label{fig:props_estelar_2}
    \end{subfigure}
    
    \begin{subfigure}{\linewidth}
        \centering
        \includegraphics[width=0.99\linewidth]{Figuras/new_props_estelar_1}
        \caption{}
        \label{fig:props_estelar_1}
    \end{subfigure}
    
    \begin{subfigure}{\linewidth}
        \centering
        \includegraphics[width=0.99\linewidth]{Figuras/new_props_estelar_3}
        \caption{}
        \label{fig:props_estelar_3}
    \end{subfigure}
    
    \begin{subfigure}{\linewidth}
        \centering
        \includegraphics[width=0.99\linewidth]{Figuras/new_props_estelar_4}
        \caption{}
        \label{fig:props_estelar_4}
    \end{subfigure}
    \caption[Propiedades estelares para diferentes conjuntos de parámetros]{Propiedades estelares obtenidas para diferentes conjuntos de parámetros que satisfacen las restricciones nucleares. Las figuras muestran diferentes regiones del plano $A_\sigma - A_\omega$ para facilitar la visualización.}
    \label{fig:props_estelares}
\end{figure}
\clearpage


\subsection{Masa Máxima}

La búsqueda del conjunto de parámetros que maximiza la masa estelar predicha es un objetivo central del análisis. Este conjunto establece el límite superior teórico para la masa de estrellas de neutrones dentro del marco del modelo considerado, y su comparación con las observaciones de estrellas masivas como las mencionadas en la sección \ref{sec:obsNS} permite evaluar si el modelo es capaz de explicar las configuraciones estelares más extremas observadas.

\begin{figure}[h]
	\centering
	\begin{subfigure}{\linewidth}
		\centering
		\includegraphics[width=\linewidth]{Figuras/new_ejemplo_espacio_6}
		\caption{}
		\label{fig:ejemplo_espacio_maxmass}
	\end{subfigure}
	
	\begin{subfigure}{\linewidth}
		\centering
		\includegraphics[width=\linewidth]{Figuras/new_props_estelar_6}
		\caption{}
		\label{fig:props_estelar_maxmass}
	\end{subfigure}
	\caption[Región de mayor masa máxima en el espacio de parámetros]{Región del espacio de parámetros con la mayor masa estelar predicha. (a) Propiedades nucleares para la región de parámetros que produce la mayor masa máxima. (b) Propiedades estelares correspondientes: masa máxima $M_\text{max}$, compacidad máxima $C_\text{max}$ y radio canónico $R_{1.4}$ para configuraciones en esta región.}
	\label{fig:max_mass}
\end{figure}

La región de mayores masas obtenidas para este modelo mediante la metodología descrita, así como sus propiedades nucleares y estelares, se muestran en la figura \ref{fig:max_mass}. En este estudio, se logró obtener una masa máxima de $M_\text{max} = 2.79 \, \masasol$, superando ampliamente el límite observado en radiación electromagnética de $2.35 \, \masasol$ para PSR J0952-0607 \cite{romaniPSRJ09520607Fastest2022}, así como el observado en ondas gravitacionales de $2.59 \, \masasol$ para el secundario en \textit{GW190814} \cite{theligoscientificcollaborationGW190814GravitationalWaves2020}. Esto indica que el modelo relativista de campo medio puede reproducir estrellas de neutrones extremadamente masivas, siendo un marco teórico consistente con las observaciones astrofísicas más exigentes. No queda claro que la región de parámetros de mayor masa encontrada en este estudio sea el límite absoluto dentro del modelo, por lo que futuros estudios y técnicas de optimización más avanzadas podrían revelar conjuntos de parámetros que produzcan masas aún mayores, sin contradecir las restricciones nucleares impuestas. De la misma forma, se halla una compacidad máxima de $C_\text{max} = 0.316$, ligeramente inferior al límite estimado de $C_\text{max} \lesssim 0.33$ \cite{annalaMultimessengerConstraintsUltradense2022}, y un radio canónico máximo $R_{1.4}$ de entre $12.79$ y $13.72 \, \text{km}$ para estrellas de masa $1.4 \, \masasol$ en esta región de parámetros. Estos valores de radio son consistentes con las estimaciones basadas en observaciones astrofísicas actuales, que sugieren un radio máximo de $R_{1.4} \lesssim 13.8$ km \cite{millerPSRJ0030+0451Mass2019, rileyNICERViewPSR2019, biswasGW190814PropertiesSecondary2021}.

Otras regiones con masas aún mayores pueden encontrarse, como la que se muestra en la figura \ref{fig:mayor_masa_invalida}, que logra una masa máxima de hasta $M_\text{max} = 3.11 \, \masasol$ respetando las restricciones nucleares. No obstante, esta región predice radios canónicos de hasta $R_{1.4} = 14.28 \, \text{km}$, que exceden las estimaciones mencionadas anteriormente. Por esta razón, esta región es descartada en el análisis, ya que no es consistente con las observaciones astrofísicas actuales. Este ejemplo ilustra la importancia de considerar múltiples restricciones al ajustar los parámetros del modelo, ya que cumplir únicamente con las propiedades nucleares no garantiza la validez astrofísica del conjunto de parámetros.

\begin{figure}[h]
    \centering
    \includegraphics[width=\linewidth]{Figuras/new_props_estelar_5}
    \caption[Región no válida astrofísicamente]{Región del espacio de parámetros que produce una masa máxima elevada pero radios canónicos inconsistentes con las observaciones astrofísicas actuales.}
    \label{fig:mayor_masa_invalida}
\end{figure}

\subsection{Comparación de Regiones}

Las regiones de parámetros mostradas en las figuras \ref{fig:region_nuclear}, \ref{fig:props_estelares} y \ref{fig:max_mass} representan diferentes áreas del espacio de parámetros que satisfacen las restricciones nucleares y producen diversas propiedades estelares. Es relevante comparar estas regiones para entender cómo las variaciones en los parámetros afectan las predicciones del modelo. En la figura \ref{fig:comparacion_regiones} se presenta una comparación directa en el plano $A_\sigma$ - $A_\omega$ entre las regiones que satisfacen las restricciones nucleares y son consistentes con las estimaciones observacionales, y las masas máximas que producen. Como notamos anteriormente, el aumento en los acoplamientos $A_\sigma$ y $A_\omega$ conduce a un incremento en la masa máxima predicha. Más aún, para mantener las propiedades de simetría dentro de los rangos aceptables, es necesario reducir el acoplamiento isovectorial $A_\rho$ al aumentar $A_\sigma$ y $A_\omega$, y puede explicarse por la necesidad de compensar el efecto de los acoplamientos escalares y vectoriales que tienden a aumentar la energía, requiriendo una disminución en la contribución isovectorial. Esta comparación resalta la interdependencia entre los parámetros del modelo y las propiedades estelares, por lo que es importante considerar estas relaciones al ajustar el modelo a observaciones astrofísicas.

\begin{figure}[h]
    \centering
    \includegraphics[width=0.8\linewidth]{Figuras/new_comparacion_regiones_validas}
    \caption[Comparación de regiones en el plano $A_\sigma$ - $A_\omega$]{Comparación de las regiones del plano de parámetros $A_\sigma$ - $A_\omega$ que satisfacen las restricciones nucleares y son consistentes con las estimaciones observacionales, y las masas máximas que producen.}
    \label{fig:comparacion_regiones}
\end{figure}

% Introducimos y discutimos la comparación de las EoS y M-R obtenidas para la configuración de mayor masa máxima en cada región válida del espacio de parámetros.

Es interesante comparar las ecuaciones de estado y las curvas masa-radio obtenidas para la configuración de mayor masa máxima en cada región válida del espacio de parámetros. Esta comparación permite evaluar cómo las diferentes elecciones de parámetros afectan la rigidez de la ecuación de estado y las propiedades macroscópicas de las estrellas de neutrones. En la figura \ref{fig:comparacion_regiones_eos}, se presentan las ecuaciones de estado y las curvas masa-radio correspondientes a las configuraciones de mayor masa máxima en cada una de las regiones válidas discutidas anteriormente. Vale destacar que se comprobó que todas las ecuaciones de estado cumplen con el límite de velocidad de la luz, asegurando la causalidad del modelo en todas las configuraciones consideradas. Además, las ecuaciones de estado se presentan a partir de la densidad de saturación $n_0 = 0.157 \, \text{fm}^{-3}$ ($\approx 2.6\times 10^{14} \, \text{g cm}^{-3}$) y hasta $n_0 = 4.0 \, \text{fm}^{-3}$ ($\approx 6.7\times 10^{15} \, \text{g cm}^{-3}$), mostrando únicamente la materia del núcleo debido a que la ecuación de estado para la corteza, discutida en la sección \ref{sec:corteza}, es independiente de los parámetros del modelo y es igual para todas las regiones.

Observamos que las ecuaciones de estado varían entre las diferentes regiones, reflejando las distintas elecciones de parámetros. En particular, la rigidez de la ecuación de estado, que influye directamente en la masa máxima y el radio de las estrellas de neutrones, muestra diferencias apreciables aumentando con regiones que producen mayores masas máximas y radios canónicos. Las curvas masa-radio exhiben también variaciones considerables, con diferencias en la masa máxima alcanzada y en los radios correspondientes para estrellas de masa $1.4 \, \masasol$. Resaltamos el hecho de que todas las configuraciones presentadas cumplen con las restricciones nucleares y son consistentes con las observaciones astrofísicas actuales, lo que muestra la capacidad del modelo relativista de campo medio para describir la materia nuclear en estrellas de neutrones bajo diferentes elecciones de parámetros.

\begin{figure}[h]
    \centering
    \includegraphics[width=0.95\linewidth]{Figuras/comparacion_regiones_validas_eos}
    \caption[Comparación de ecuaciones de estado y curvas masa-radio de diferentes regiones]{Comparación de las ecuaciones de estado (izquierda) y las curvas masa-radio (derecha) obtenidas para la configuración de mayor masa máxima en cada región válida del espacio de parámetros, mostradas en la figura \ref{fig:comparacion_regiones}.}
    \label{fig:comparacion_regiones_eos}
\end{figure}

Adicionalmente, es pertinente analizar el comportamiento microscópico de la materia nuclear predicha por las configuraciones de mayor masa máxima en cada región. En la figura \ref{fig:comparacion_regiones_fraccion} se presentan la energía de enlace por nucleón ($B/A$) y la fracción de neutrones respecto a la densidad bariónica ($n_n/n_B$) para estas configuraciones. La energía de enlace se grafica en un rango de densidades desde $0.001 \, \text{fm}^{-3}$ ($\approx 1.7\times 10^{12} \, \text{g cm}^{-3}$) hasta $0.7 \, \text{fm}^{-3}$ ($\approx 1.2\times 10^{15} \, \text{g cm}^{-3}$), lo que permite observar el comportamiento de la materia nuclear simétrica en función de los parámetros de cada región en saturación y a densidades varias veces superior a la saturación. Por otro lado, la fracción de neutrones se presenta desde la densidad de saturación hasta $100 \, \text{fm}^{-3}$ ($\approx 1.7\times 10^{17} \, \text{g cm}^{-3}$), cubriendo las densidades supranucleares características del núcleo estelar. A partir de la gráfica de $B/A$, se confirma lo observado en la figura \ref{fig:comparacion_regiones_eos}: las ecuaciones de estado que producen mayores masas máximas son efectivamente más rígidas, ya que la energía de enlace aumenta más rápidamente con la densidad. Respecto a la composición, se observa que para una región de densidades intermedia, la fracción de neutrones es mayor para las configuraciones de menor masa máxima. Sin embargo, este comportamiento se invierte a mayores densidades, donde las configuraciones de mayor masa máxima pasan a tener la mayor fracción de neutrones. Esta inflexión sucede aproximadamente entre $0.55$ y $0.7 \, \text{fm}^{-3}$ ($\approx 9.2\times 10^{14}$ y $1.2\times 10^{15} \, \text{g cm}^{-3}$).

\begin{figure}[h]
    \centering
    \includegraphics[width=0.95\linewidth]{Figuras/comparacion_regiones_validas_fraccion}
    \caption[Energía de enlace y fracción de neutrones para diferentes regiones]{Energía de enlace por nucleón (izquierda) y fracción de neutrones (derecha) para la configuración de mayor masa máxima en las regiones de la figura \ref{fig:comparacion_regiones}. La facción de protones es entonces $n_p/n_B = 1 - n_n/n_B$.}
    \label{fig:comparacion_regiones_fraccion}
\end{figure}