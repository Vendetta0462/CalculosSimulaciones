\chapter{Conclusiones}

En este trabajo se implementó y analizó un modelo de materia nuclear basado en la Teoría Relativista de Campo Medio con autointeracciones no lineales del campo escalar, aplicado al estudio de estrellas de neutrones. A través de la resolución autoconsistente de las ecuaciones de campo y las ecuaciones de estructura estelar TOV, se demostró la viabilidad del formalismo para describir configuraciones estelares que satisfacen condiciones de aceptabilidad física, restricciones nucleares empíricas y reproducen observaciones astrofísicas. La exploración sistemática del espacio de parámetros permitió identificar regiones específicas donde el modelo reproduce satisfactoriamente las propiedades de saturación de la materia nuclear y, simultáneamente, genera predicciones macroscópicas de masa y radio consistentes con las observaciones astrofísicas más recientes, incluyendo eventos de ondas gravitacionales y mediciones de púlsares masivos.

\section{Principales hallazgos}

Los resultados obtenidos muestran la compleja interdependencia entre los parámetros microscópicos de la interacción fuerte y las propiedades observables de las estrellas de neutrones, validando la eficacia del formalismo RMF frente a modelos más simples.

El análisis de las condiciones de aceptabilidad física demostró que el modelo satisface los criterios de estabilidad mecánica, causalidad y regularidad métrica en el interior estelar. Se verificó que la velocidad del sonido se mantiene siempre inferior a la de la luz y que la condición de energía dominante se cumple en todo el rango de densidades. Aunque se observó una violación de la condición de estabilidad convectiva en las capas más externas de la estrella, este comportamiento parece ser consistente con otros modelos relativistas de campo medio. Esto implica que las ecuaciones de estado derivadas son físicamente admisibles para la descripción de objetos compactos en equilibrio hidrostático, fortaleciendo la validez del enfoque adoptado.

La caracterización del espacio de parámetros evidenció una jerarquía clara en la influencia de los acoplamientos mesónicos, validando la metodología de ajuste sistemático empleada. Se confirmó el desacople del mesón isovectorial a las propiedades de materia simétrica, donde el parámetro $A_\rho$ no afecta las propiedades $n_0$, $B/A$ y $K_0$, y se evidenció un impacto despreciable a la masa máxima de la estrella, limitando su influencia casi exclusivamente a la energía de simetría, su pendiente y al radio estelar. Por el contrario, los acoplamientos escalar ($A_\sigma$) y vectorial ($A_\omega$), en conjunto con los términos de autointeracción escalar ($b$ y $c$), resultaron determinantes para la rigidez de la ecuación de estado y el control del módulo de compresibilidad. En particular, se encontró que el parámetro $c$ favorece significativamente el aumento de la masa máxima y el radio, mientras que $b$ actúa como un regulador de las propiedades de saturación. Físicamente, esto refleja cómo la competencia entre la atracción escalar y la repulsión vectorial, analizada mediante la descomposición de la energía de enlace, gobierna la estructura, mientras que la interacción isovectorial ajusta la composición de isospín y el tamaño físico de la estrella sin alterar significativamente su capacidad de soporte de masa.

Finalmente, en contraste con el modelo de gas ideal degenerado cuya masa máxima se limita a $\sim 0.7 \, \masasol$, la inclusión de las interacciones fuertes permitió obtener configuraciones capaces de soportar hasta $2.79 \, \masasol$. Este resultado es compatible con los candidatos a estrellas de neutrones más masivos observados, como el secundario en el evento de ondas gravitacionales \textit{GW190814}. Más aún, se encontró que existen regiones del espacio de parámetros que permiten masas aún mayores ($> 3.0 \, \masasol$), pero generan radios canónicos ($R_{1.4}$) inconsistentes con las restricciones observacionales de NICER, subrayando la necesidad de validar los modelos teóricos frente a múltiples observables astrofísicos. Adicionalmente, la comparación con el gas ideal reveló que las interacciones nucleares reducen significativamente la fracción de neutrones a altas densidades, favoreciendo una composición más equilibrada de protones y neutrones.

Finalmente, se identificó una configuración capaz de soportar una masa máxima de $2.79 \, \masasol$, compatible con los candidatos a estrellas de neutrones más masivos observados, como el secundario en el evento de ondas gravitacionales \textit{GW190814}. Más aún, se encontró que existen regiones del espacio de parámetros que permiten masas aún mayores ($> 3.0 \, \masasol$), pero generan radios canónicos ($R_{1.4}$) inconsistentes con las restricciones observacionales actuales de NICER, lo que subraya la necesidad de validar los modelos teóricos frente a múltiples observables astrofísicos. Adicionalmente, la comparación con el gas ideal reveló que las interacciones nucleares reducen significativamente la fracción de neutrones a altas densidades, favoreciendo una composición más equilibrada de protones y neutrones.

\section{Limitaciones y recomendaciones}

Este estudio se realizó bajo ciertas aproximaciones que limitan el alcance de las conclusiones. Se asumió una composición de materia nuclear constituida únicamente por nucleones (neutrones y protones) y electrones en equilibrio beta, ignorando la inclusión de otros leptones (muones) y grados de libertad exóticos como hiperones, condensados de piones o fases de quarks desconfinados que podrían aparecer a altas densidades y modificar la ecuación de estado. Además, el análisis se restringió a estrellas estáticas y esféricamente simétricas a temperatura cero, dejando de lado efectos de rotación, campos magnéticos intensos y temperatura finita, factores no despreciables en escenarios astrofísicos dinámicos como fusiones de estrellas de neutrones.

Para futuras investigaciones, se recomienda extender el modelo incluyendo el octeto de bariones para estudiar el impacto de la aparición de hiperones en la masa máxima y la estabilidad estelar. Asimismo, sería pertinente explorar la inclusión de interacciones adicionales, como el mesón isovectorial-escalar $\delta$, para refinar el comportamiento de la energía de simetría a altas densidades, o términos extra de autointeracción para los mesones $\omega$ y $\rho$. Adicionalmente, se pueden aplicar otros métodos de ajuste de parámetros, como técnicas bayesianas o machine learning, para optimizar la búsqueda en el espacio de parámetros y cuantificar las incertidumbres asociadas a las predicciones del modelo. Finalmente, incorporar efectos de rotación y temperatura finita permitiría una descripción más realista de las estrellas de neutrones en contextos astrofísicos variados.